%%%%%%%%%%%%%%%%%%%%%%%%%%%%%%%%%%%%%%%%%%%%%%%%%%%%%%%%%%%%%%%%%%%%%%%%%%%%%%%%
% partII/appendix.tex
%
% Copyright 2012, Jeffrey Hellrung.
%%%%%%%%%%%%%%%%%%%%%%%%%%%%%%%%%%%%%%%%%%%%%%%%%%%%%%%%%%%%%%%%%%%%%%%%%%%%%%%%

\chapter{Quadrature} \label{chap:partII.appendix.quadrature}

In Chapter~\ref{chap:partII.poisson}, \S\ref{subsec:chap4.discretization.embedding} we mention the use of Gaussian quadrature rules to compute surface integrals over triangles of polynomial integrands. For convenience, we reproduce the triangle Gaussian quadrature rules of various orders from \cite{Cowper73} in Table~\ref{tab:partII.quadrature.rules}. For the quadrature points with multiplicity $3$, the coordinates should be permuted to give $3$ total symmetrically distributed quadrature points, all with the same given quadrature weight. For example, to integrate a cubic polynomial $p(\bfx) : \bbR^3 \to \bbR$ over a triangle $T$ with vertices $\{\bfx_1, \bfx_2, \bfx_3\} \subset \bbR^3$, one would use the order $3$ quadrature rule from Table~\ref{tab:partII.quadrature.rules}, which manifests itself as
\begin{equation} \label{eq:partII.quadrature.example}
\begin{split}
\int_T p(\bfx) d\bfx = {} & \operatorname{area}(T) \left( -\frac{27}{48} \tilde{p} \lrp{\frac{1}{3}, \frac{1}{3}, \frac{1}{3}} \right. \\ & \left. {} + \frac{25}{48} \lrp{\tilde{p} \lrp{\frac{1}{5}, \frac{1}{5}, \frac{3}{5}} + \tilde{p} \lrp{\frac{1}{5}, \frac{3}{5}, \frac{1}{5}} + \tilde{p} \lrp{\frac{3}{5}, \frac{1}{5}, \frac{1}{5}}} \right),
\end{split}
\end{equation}
where $\tilde{p} \lrp{\alpha_1, \alpha_2, \alpha_3} = p \lrp{\alpha_1 \bfx_1 + \alpha_2 \bfx_2 + \alpha_3 \bfx_3}$. Note how the multiplicity $3$ quadrature point with barycentric coordinates $(1/5, 1/5, 3/5)$ in Table~\ref{tab:partII.quadrature.rules} represents all of the latter $3$ quadrature points in \eqref{eq:partII.quadrature.example} via permutation of the coordinates.

\begin{table}[htbp]
\centering
\begin{tabular}{|@{$\,$}c@{$\,$}|@{$\,$}c@{$\,$}|c|c|}
\hline
order & mult. & weight & barycentric coordinates \\
\hline\hline
$1$ & $1$ & $1$ & $(1/3, 1/3, 1/3)$ \\
\hline
$2$ & $3$ & $1/3$ & $(1/6, 1/6, 2/3)$ \\
\hline
\multirow{2}{*}{$3$}
& $1$ & $-27/48$ & $(1/3, 1/3, 1/3)$ \\
& $3$ & $25/48$ & $(1/5, 1/5, 3/5)$ \\
\hline
\multirow{2}{*}{$4$}
& $3$ & $0.109951743655322$ & $(0.091576213509771, 0.0915\dotso, 0.816847572980459)$ \\
& $3$ & $0.223381589678011$ & $(0.108103018168070, 0.445948490915965, 0.4459\dotso)$ \\
\hline
\multirow{3}{*}{$5$}
& $1$ & $9/40$ & $(1/3, 1/3, 1/3)$ \\
& $3$ & $0.125939180544827$ & $(0.101286507323456, 0.1012\dotso, 0.797426985353087)$ \\
& $3$ & $0.132394152788506$ & $(0.059715871789770, 0.470142064105115, 0.4701\dotso)$ \\
\hline
\end{tabular}
\caption{Triangle Gaussian quadrature rules of order $1$ through $5$, as given in \cite{Cowper73}. [Some repeated barycentric coordinates have been abbreviated with ``$\dotso$'' for formatting purposes.]}
\label{tab:partII.quadrature.rules}
\end{table}

\chapter{Cell Averages} \label{chap:partII.appendix.cellaverages}

The Poisson discretizations described in Chapter~\ref{chap:partII.poisson}, \S\ref{sec:chap4.discretization} require computing cell averages $\overline{\beta}$, $\overline{f}$, and $\overline{q}$ of $\beta$, $f$, and $q$, respectively. One can use any of a variety of techniques to compute these averages, and one's choice would likely depend upon whether one has $\beta$, $f$, and $q$ immediately defined pointwise at grid vertices; pointwise at grid edge, face, or cell centers; or analytically throughout the domain, domain boundary, or interface.

We used evaluations of $\beta$ and $f$ at grid vertices to compute their cell averages. For non-boundary and non-interfacial grid cells, the cell average amounts to a straightforward, equal-weighted average of the values at the $8$ grid vertices of the cell. For boundary and interfacial grid cells, we used trilinear interpolation to compute the cell average. For example, in the domain discretizations, we compute the cell average of $\beta$ over grid cell $c_k \in \calC^h_{\dOmega}$ as
\begin{equation*}
\overline{\beta} := \frac{\int_{c_k \cap \Omega} \beta d\bfx}{\int_{c_k \cap \Omega} d\bfx} \approx \frac{\sum_{i \in \calN^h_{c_k}} \beta_i \int_{c_k \cap \Omega} N_i d\bfx}{\int_{c_k \cap \Omega} d\bfx},
\end{equation*}
where, as introduced in \S\ref{subsec:chap4.discretization.neumann}, $\set{ N_i : i \in \calN^h_{c_k} }$ denotes the set of trilinear basis functions associated to the $8$ grid vertices of $c_k$. Note that all integrands remaining in the right-most expression are polynomials, hence the integrals may be evaluated as described in \S\ref{subsec:chap4.discretization.embedding}. We compute $\overline{f}$, as well as cell averages in embedded interface discretizations, in a completely analogous fashion.

For embedded Neumann discretizations, to simplify implementation, we assume that $q \equiv \beta \nabla u \cdot \hatn$ is available everywhere along the polyhedral representation of $\dOmega$. We use the second order quadrature rule from Table~\ref{tab:partII.quadrature.rules} in Appendix~\ref{chap:partII.appendix.quadrature} over each polygon of $\calP^{c_k}_{\dOmega}$ (where $c_k \in \calC^h_{\dOmega}$) to approximate $\overline{q}$:
\begin{equation*}
\overline{q} := \frac{\int_{c_k \cap \dOmega} q d\bfS(\bfx)}{\int_{c_k \cap \dOmega} d\bfS(\bfx)} \approx \frac{\sum_{g \in \calP^{c_k}_{\dOmega}} \int_g q d\bfS(\bfx)}{\sum_{g \in \calP^{c_k}_{\dOmega}} \operatorname{area}(g)}.
\end{equation*}

\chapter{Double-Wide Constraint Conditioning} \label{chap:partII.appendix.constraintconditioning}

In the context of the constraint aggregation algorithm for the discretization of Dirichlet boundary conditions and interfacial jump conditions for Poisson's equation, as mentioned in Chapter~\ref{chap:partII.poisson}, \S\ref{subsubsec:chap4.constraintaggregation}, we found that the double-wide constraints introduced in \cite{Bedrossian10} present significant conditioning issues in $3$ dimensions that do not exist in $2$ dimensions. Table~\ref{tab:partII.constraintconditioning} shows the condition numbers and the number of conjugate gradient solve iterations for the $Z^tAZ$ matrices resulting from the discretization of a simple Dirichlet problem and from the discretization of a similarly simple interface problem using each of two alternate discretizations $\Lambda^h$ of the Lagrange multiplier space $\Lambda$: $\Lambda^h_2$, which corresponds to the double-wide constraints; and $\Lambda^h_a$, which corresponds to the aggregate constraints constructed via the algorithm described in \S\ref{subsubsec:chap4.constraintaggregation}. We calculated these statistics using PETSc in exactly the same way as for Table~\ref{tab:chap4.examples.interface}. This includes applying Jacobi preconditioning and then solving with (incomplete Cholesky-preconditioned) conjugate gradient to a relative residual norm of $2.3 \times 10^{-13}$ of the Jacobi preconditioned system. Clearly, the conditioning of the $Z^tAZ$ system arising from the double-wide constraints is several orders of magnitude worse than that arising from the constraint aggregation algorithm described in \S\ref{subsubsec:chap4.constraintaggregation}.

\begin{table}[htbp]
\centering
\begin{tabular}{|c|c|c|c|c|}
\hline
Test case & cond. \# (no ICC) & cond. \# (w/ICC) & \# CG iter. & \# PCG iter. \\
\hline
Dirichlet, $\Lambda^h = \Lambda^h_2$ & $3.7 \times 10^{12}$ & $1.1 \times 10^{12}$ & $59846$ & $61568$ \\
Interface, $\Lambda^h = \Lambda^h_2$ & $4.4 \times 10^{12}$ & $1.4 \times 10^{13}$ & $97061$ & $80225$ \\
Dirichlet, $\Lambda^h = \Lambda^h_a$ & $9.3 \times 10^{ 2}$ & $2.3 \times 10^{ 1}$ &   $200$ &    $44$ \\
Interface, $\Lambda^h = \Lambda^h_a$ & $3.9 \times 10^{ 3}$ & $4.1 \times 10^{ 1}$ &   $494$ &    $61$ \\
\hline
\end{tabular}
\caption{Condition numbers and (preconditioned) conjugate gradient ((P)CG) solve iterations, both with and without Incomplete Cholesky (ICC) preconditioning, for the $Z^tAZ$ system arising from the discretization of a Dirichlet and from the discretization of an interface problem at grid resolution $32 \times 32 \times 32$. The Dirichlet problem has $\Omega = \{\bfx : \abs{\bfx} \leq 0.8\}$ and $\beta \equiv 1$; the interface problem has $\Gamma = \{\bfx : \abs{\bfx} = 0.8\}$ and $(\beta^-, \beta^+) \equiv (1, 2)$.}
\label{tab:partII.constraintconditioning}
\end{table}
