%%%%%%%%%%%%%%%%%%%%%%%%%%%%%%%%%%%%%%%%%%%%%%%%%%%%%%%%%%%%%%%%%%%%%%%%%%%%%%%%
% chapters/part2.tex
%
% Copyright 2012, Jeffrey Hellrung.
%%%%%%%%%%%%%%%%%%%%%%%%%%%%%%%%%%%%%%%%%%%%%%%%%%%%%%%%%%%%%%%%%%%%%%%%%%%%%%%%

\part{Virtual Node Methods for Elliptic Problems}

\section{Introduction} \label{sec:pt2.Introduction}

An embedded framework lends itself nicely to solving partial differential equations on arbitrary, irregular domains with a discretization based on a regular Cartesian grid. We will consider elliptic problems in the subsequent chapters, specifically:

\begin{itemize}

\item Poisson's equation with interfacial jump conditions:
\begin{subequations} \label{eq:pt2.Poisson}
\begin{align}
-\nabla \cdot \lrp{\beta \nabla u} & = f, \quad \in \Omega \setminus \Gamma; \label{eq:pt2.Poisson.PDE} \\
\jump{u} & = a, \quad \in \Gamma; \label{eq:pt2.Poisson.DJ} \\
\jump{\beta \nabla u \cdot \hatn} & = b, \quad \in \Gamma; \label{eq:pt2.Poisson.NJ} \\
u & = p, \quad \in \dOmega_d; \label{eq:pt2.Poisson.D} \\
\beta \nabla u \cdot \hatn & = q, \quad \in \dOmega_n; \label{eq:pt2.Poisson.N}
\end{align}
\end{subequations}
where
\begin{itemize}
\item the domain $\Omega \subset \bbR^d$ is open ($d = 2 \text{ or } 3$, typically);
\item the interface $\Gamma$ is a co-dimension one closed curve ($d = 2$) or surface ($d = 3$) that divides $\Omega$ into the interior domain $\Omega^-$ and the exterior domain $\Omega^+$, such that $\Omega = \Omega^- \sqcup \Omega^+ \sqcup \Gamma$;
\item $u$ (unknown), $\beta$ (known), and $f$ (known) are scalar functions which can exhibit discontinuities across $\Gamma$ but otherwise are assumed to have smooth restrictions $u^{\sigma}$, $\beta^{\sigma}, f^{\sigma}$ to $\Omega^{\sigma}$, $\sigma \in \set{+,-}$;
\item $\hatn = \hatn(\bfx)$ denotes the outward unit normal, both to $\Omega^-$ for $\bfx \in \Gamma$ and to $\Omega$ for $\bfx \in \dOmega$; and
\item $\jump{v}(\bfx) := v^+(\bfx) - v^-(\bfx) := \lim_{\epsilon \to 0^+} v\lrp{\bfx + \epsilon \hatn(\bfx)} - \lim_{\epsilon \to 0^+} v\lrp{\bfx - \epsilon \hatn(\bfx)}$ denotes the \emph{jump} of the quantity $v$ across the interface $\Gamma$.
\end{itemize}
The relevant physics generally determine the jumps in the solution \eqref{eq:pt2.Poisson.DJ} and in the flux \eqref{eq:pt2.Poisson.NJ}, as well as the boundary conditions \eqref{eq:pt2.Poisson.D}, \eqref{eq:pt2.Poisson.N} on $\dOmega$.

\item The equilibrium equations of linear elasticity:
\begin{subequations} \label{eq:pt2.elasticity}
\begin{align}
-\nabla \cdot \bssigma(\bfu) & = \bff, \quad \in \Omega; \\
\bfu & = \bfu_0, \quad \in \dOmega_d; \\
\bssigma(\bfu) \cdot \hatn & = \bstau, \quad \in \dOmega_n;
\end{align}
\end{subequations}
where
\begin{itemize}
\item the domain $\Omega \subset \bbR^d$ is open (we consider $d = 2$ primarily);
\item $\bfu$ is the (unknown) material displacement map;
\item $\bssigma$ is the Cauchy stress tensor;
\item $\bff$ is the external force per unit volume;
\item $\bfu_0$ is the prescribed Dirichlet boundary displacements over $\dOmega_d \subset \dOmega$; and
\item $\bstau$ is the prescribed external surface traction over $\dOmega_n \subset \dOmega$.
\end{itemize}
In linear elasticity, the stress $\bssigma(\bfu)$ is linearly dependent on the Cauchy strain $\bsepsilon(\bfu)$ via
\begin{align*}
\bsepsilon(\bfu) & = \frac{1}{2} \lrp{\nabla \bfu + \lrp{\nabla \bfu}^T}, \\
\bssigma(\bfu)   & = 2 \mu \bsepsilon(\bfu) + \lambda \lrp{\tr \bsepsilon(\bfu)} \bfI \\
                 & = \mu \lrp{\nabla \bfu + \lrp{\nabla \bfu}^T} + \lambda \lrp{\nabla \cdot \bfu} \bfI.
\end{align*}
Therefore, the linear elasticity equations \eqref{eq:pt2.elasticity} are equivalent to
\begin{subequations} \label{eq:pt2.LE}
\begin{align}
-\lrp{\Delta \bfI + (\lambda + \mu) \nabla \nabla^T} \bfu & = \bff \quad \in \Omega \\
\bfu & = \bfu_0 \quad \in \dOmega_d \\
\mu \lrp{\bfu \cdot \hatn + \nabla \lrp{\bfu \cdot \hatn}} + \lambda \lrp{\nabla \cdot \bfu} \hatn & = \bstau \quad \in \dOmega_n.
\end{align}
\end{subequations}

\end{itemize}

In all the above, $\dOmega = \dOmega_d \sqcup \dOmega_n$, and we assume $\Gamma$ (if applicable) and $\dOmega$ to be sufficiently smooth.

Due to irregular geometry of the domain boundary $\dOmega$ and (if applicable) the interface $\Gamma$ in many physical phenomena, a natural approach to the numerical approximations of \eqref{eq:pt2.Poisson} and \eqref{eq:pt2.elasticity}/\eqref{eq:pt2.LE} is the \emph{finite element method} (FEM) with \emph{unstructured meshes} that conform to the geometry of $\dOmega$ and (if applicable) $\Gamma$ \cite{Babuska70, Bramble96, Chen.Zhiming96, Dryja05, Cockburn09, Wohlmuth99, Huang02, Lamichhane04}. However, meshing complex domain boundary and interface geometries can prove difficult and time-consuming when these surfaces frequently change shape, especially in $3$ dimensions. Indeed, many applications, such as shape optimization for elastic materials (\cite{Sethian00, Osher01, Allaire04, Duysinx06, Challis08, Wei.Peng08}) or crack propagation, require the geometry of the domain to change at each time step of the simulation. The task is further complicated when using the more elaborate element types seen in mixed FEM formulations, which are typically necessary for stability in the nearly incompressible regime of \eqref{eq:pt2.elasticity}/\eqref{eq:pt2.LE}. Also, many numerical methods, such as geometric multigrid methods, do not naturally apply to unstructured meshes. These concerns are largely circumvented with the use of \emph{embedded} (or \emph{immersed}) methods that approximate solutions to \eqref{eq:pt2.Poisson} and \eqref{eq:pt2.LE}/\eqref{eq:pt2.LE2} on regular Cartesian grids or structured meshes that do not conform to the domain boundary nor the interface. Retention of higher order accuracy in $L^{\infty}$ with such embedded strategies is an ongoing area of research. The difficulty typically lies in determining the numerical stencils near embedded features such as $\dOmega$ and $\Gamma$ that retain the accuracy achieved in the regions of the domain sufficiently distanced from said embedded features. Many present methods address these problems at the cost of complex implementation that sometimes require significant effort to adapt to general applications. But while the accuracy of the discretization is an issue, the ability to use a regular Cartesian grid greatly facilitates the implementation of efficient solution methods. Specifically for high resolution discretizations, efficient solution of the discrete systems can be challenging; direct methods become too slow and memory intensive. Geometric multigrid methods and domain decomposition approaches have been shown to provide very favorable performance in this setting. However, their application to embedded discretizations is not straightforward. Ultimately, special attention must be paid near embedded features for both discretization accuracy and efficient numerical linear algebra.

The methods presented in the following chapters build on the work of Bedrossian et al. \cite{Bedrossian10}, which introduced a second order virtual node method for solving the elliptic interface problem \eqref{eq:pt2.Poisson} in $2$ dimensions. The discretization presented in \cite{Bedrossian10} is easy to implement and yields a symmetric positive definite sparse linear system for both interface problems and boundary value problems on irregular domains. In summary, this virtual node method employs a uniform Cartesian grid with duplicated Cartesian bilinear elements along the interface. These duplicated elements introduce additional \emph{virtual} nodes or degrees of freedom to accurately capture the lack of regularity in the solution. The method is variational to define stencils symmetrically, and a different discretization is used depending on proximity to embedded features, allowing for the retention of the standard $5$-point finite difference stencil away from embedded boundaries and interfaces. Langrange multipliers are used to enforce embedded Dirichlet conditions \eqref{eq:pt2.Poisson.D} and embedded jump conditions \eqref{eq:pt2.Poisson.DJ}, and the choice of Lagrange multiplier space admits a symmetric positive definite discretization. In the special case when $\beta$ is smooth, a discontinuity removal technique allows the use of the standard $5$-point Poisson stencil even across the embedded interface.

The feature set of this virtual node approach is very powerful. In Chapter \ref{chap:Poisson}, we improve many aspects of \cite{Bedrossian10} and provide key modifications necessary to extend the method to $3$ dimensions.


Within the context of embedded Dirichlet and embedded interface discretizations, we present a novel and flexible algorithm to define the discrete Lagrange multiplier space. This algorithm gives more control on the conditioning of the resulting linear system and specifically addresses the conditioning issues (see Appendix \ref{???}) we found in the straightforward extension of \cite{Bedrossian10} to $3$ dimensions.

We also give an expanded treatment of the discontinuity removal technique, detailing an algorithm for the construction of a scalar function satisfying the jump conditions (\ref{eq:pt2.Poisson.DJ}, \ref{eq:pt2.Poisson.NJ}).

Specific to the $3$-dimensional implementation, we describe an algorithm for creating a polyhedral representation of cell-local interface/boundary geometry and quadrature rules suitable for these polyhedral surfaces. Finally, we present a family of multigrid algorithms that solve \eqref{eq:pt2.Poisson} with near-optimal multigrid efficiency.








