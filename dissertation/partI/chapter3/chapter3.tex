%%%%%%%%%%%%%%%%%%%%%%%%%%%%%%%%%%%%%%%%%%%%%%%%%%%%%%%%%%%%%%%%%%%%%%%%%%%%%%%%
% partI/chapter3/chapter3.tex
%
% Copyright 2012, Jeffrey Hellrung.
%%%%%%%%%%%%%%%%%%%%%%%%%%%%%%%%%%%%%%%%%%%%%%%%%%%%%%%%%%%%%%%%%%%%%%%%%%%%%%%%

\chapter{Geometric Fracture Modeling in Computer Animation} \label{chap:partI.fractureanimation}

\setlength{\figurewidth}{0.32\textwidth}
\begin{figure}[htbp]
\centering
\includegraphics[width=\figurewidth]{partI/chapter3/figures/1a}
\includegraphics[width=\figurewidth]{partI/chapter3/figures/1b}
\includegraphics[width=\figurewidth]{partI/chapter3/figures/1c}
\caption{Left, middle: Rhino�s ball is riddled with cracks as a metal gate crushes it down. Right: A roadway is torn up by Bolt�s ``superbark''.}
\label{fig:chap3.1}
\end{figure}

\section{Introduction}

\footnote{The content of this chapter is a version of \cite{Hellrung09} with minor revisions.}
Modeling the geometry of solid materials cracking and shattering
into elaborately shaped pieces is a painstaking task, which is often
impractical to tune by hand when a large number of fragments are
produced. In Walt Disney�s animated feature film \textit{Bolt}, cracking
and shattering objects were prominent visual elements in a number
of action sequences. We designed a system to facilitate the modeling
of cracked and shattered objects, enabling the automatic generation
of a large number of fragments while retaining the flexibility to
artistically control the density and complexity of the crack formation,
or even manually controlling the shape of the resulting pieces
where necessary. Our method resolves every fragment \emph{exactly} into
a separate triangulated surface mesh, producing pieces that line up
perfectly even upon close inspection, and allows straightforward
transfer of texture and look properties from the un-fractured model.

\section{Crack Geometry Generation}

The input to our system consists of a closed triangulated surface
defining the (uncut) solid object to be fractured and one or more additional
triangulated surfaces defining the geometry of the cracks.
The geometry of this ``crack surface'' is not constrained by the shape
of the material object itself; cracks are free to extend outside the
material into the empty space, and can have non-manifold shapes,
topological junctions, or even intersect themselves. Leveraging this
flexibility, we can automatically create a fracture surface which partitions
the ambient space into any given number of regions by simply
introducing the same number of seed points and computing the
3D Voronoi regions of those seed locations. The fracture surface is
then defined as the union of all boundaries between Voronoi cells
(see Figure~\ref{fig:chap3.2}, top right). In practice, we approximate these regions
by laying down a background, procedurally generated tetrahedral
mesh, and computing the Voronoi cells via a flood-fill on the tetrahedral
mesh, starting from the elements containing the seed points.

The shape of the fracture surface can be controlled by specifying the
number of seed points, or even by volumetric painting of seed point
densities, to control which regions will shatter into more, smaller
fragments. We also control the smoothness of the boundaries between
Voronoi regions by jittering the background tetrahedral mesh
prior to the flood-fill, and selectively smoothing the boundary surface
where smoother cracks are desired. Finally, in certain situations
(e.g. the cracks of the hamster ball in Figure~\ref{fig:chap3.1}), more specific
artist control of the fragment geometry is desired. For these cases,
we extruded sets of artist-drawn 3D curves in the direction normal
to the object surface to create a fracture surface that cuts through
the material and reflects the crack design intended by the artists.

\section{Automatic Fragment Mesh Generation}

We adapt the cutting algorithm of \cite{Sifakis07} to automatically
generate triangulated meshes for the material fragments defined
by the object and fracture geometries of the previous section.
In particular, their method begins with a tetrahedral volumetric representation
of the material to be fractured, as opposed to the triangulated
boundary geometry we assumed as input to our system. We
handle this representation discrepancy by first generating a tetrahedral
mesh that fully covers the object to be fractured . The triangulated
surface of the uncut object itself is used as the first cut in an
application of the algorithm of \cite{Sifakis07}, effectively sectioning
the background tetrahedral volume into the ``material'' and
``void'' regions. The fracture surface is then applied as the second
cut, resulting in the separation of the material volume into separate
fragments. The cutting algorithm computes the triangulated boundary
of every volumetric fragment in a way that every triangle of a
fragment is contained inside a triangle either of the uncut object,
or of the fracture surface (Figure~\ref{fig:chap3.2}, bottom). As a result, texture and
look properties can be remapped simply by embedding each resulting
triangle barycentrically into either the material or the fracture
surface respectively, and looking up the properties of the embedding
triangle. Finally, although our framework is currently used for
geometric modeling, we aim to employ it in conjunction with simulation
in the future, to model time-dependent crack propagation.

\setlength{\figurewidth}{0.49\textwidth}
\begin{figure}[htbp]
\centering
\includegraphics[width=\figurewidth]{partI/chapter3/figures/2a}
\includegraphics[width=\figurewidth]{partI/chapter3/figures/2b} \\
\includegraphics[width=\figurewidth]{partI/chapter3/figures/2c}
\includegraphics[width=\figurewidth]{partI/chapter3/figures/2d}
\caption{Top left: Simulation of the shattered fragments of Rhino�s ball. Top right: Fracture surfaces defined as the boundaries of Voronoi regions in 3D. Bottom: The fragments are fully resolved as independent surface meshes, and can be separately manipulated.}
\label{fig:chap3.2}
\end{figure}
