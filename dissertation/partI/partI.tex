%%%%%%%%%%%%%%%%%%%%%%%%%%%%%%%%%%%%%%%%%%%%%%%%%%%%%%%%%%%%%%%%%%%%%%%%%%%%%%%%
% partI/partI.tex
%
% Copyright 2012, Jeffrey Hellrung.
%%%%%%%%%%%%%%%%%%%%%%%%%%%%%%%%%%%%%%%%%%%%%%%%%%%%%%%%%%%%%%%%%%%%%%%%%%%%%%%%

\part{Applications of Arbitrary Lagrangian Mesh Cutting}

\renewcommand{\thechapter}{\thepart}

\section*{Introduction}

Many physical simulations necessitate the modeling of fracture or crack surfaces, and in the most demanding of these simulations, these surfaces may have a highly complex non-manifold topology, e.g., with many branches and open ``fronts''. In a dynamics scenario, this may be further complicated by the frequent extension of existing crack fronts and the introduction of new failure surfaces. Applying a traditional finite element method in this context is challenging: one must either constantly regenerate the simulation mesh to account for the fracture geometry, which is computationally expensive and easily introduces ill-conditioned ``sliver'' elements; or one must artificially and often severely limit the potential paths and resolution of the failure surface to lie along element boundaries. Additionally, in either case, handling a high resolution fracture surface necessitates a correspondingly high resolution simulation volume, at least locally, which in turn increases the computational cost of solving the relevant discrete continuum mechanics equations. Ideally, one should be able to decouple the resolution of the fracture surface (which may be highly detailed for visual effects purposes, for example; see Chapter~\ref{chap:partI.fractureanimation}) from the resolution of the simulation mesh (which may be limited by available computational resources; see Chapter~\ref{chap:partI.virtualsurgery}).

Given the above difficulties with traditional finite element methods, Belytschko, Black, Mo\"{e}s, Dolbow \cite{Belytschko99, Moes99} and others developed the \emph{eXtended Finite Element Method} (XFEM), specifically in the context of modeling cracks, which avoids the need to remesh to capture crack geometry. The basic idea of the XFEM is to enrich the usual finite element spaces with additional degrees of freedom which incorporate the near tip asymptotic solutions and allow the displacements around the crack surface to be discontinuous. We hold off a complete introduction to and history of the XFEM until Chapter~\ref{chap:partI.crackpropagation}, but do mention one of the main challenges with utilizing the XFEM: automating the determination of material connectivity and subsequent enrichment of the finite element spaces. In the following chapters, we apply the mesh cutting algorithm of Sifakis et al. \cite{Sifakis07} to resolve arbitrary Lagrangian cutting surfaces against the simulation volume mesh, to determine material connectivity, and to automatically duplicate mesh vertices to yield \emph{virtual nodes} which effect the requisite enrichment necessary for separation.

We conclude this prelude to Part I with a brief summary of the mesh cutting algorithm from \cite{Sifakis07}, as the main subject of Part I is the various applications of this algorithm. Chapter~\ref{chap:partI.crackpropagation} combines the XFEM with the mesh cutting algorithm to simulate propagating cracks in $2$ dimensions; we also introduce a novel quadrature scheme to accurately and straightforwardly integrate the nonlinear and singular finite element basis functions. In Chapter~\ref{chap:partI.virtualsurgery}, we consider the combination of this mesh cutting algorithm with nonlinear continuum mechanics in $3$ dimensions to create a virtual surgery simulator. Lastly, in Chapter~\ref{chap:partI.fractureanimation}, we discuss a framework to systematically create cracked and shattered models for visual effects and computer animation.

\section*{Mesh Cutting Algorithm Overview}

We now give a brief overview of the mesh cutting algorithm described by Sifakis et al.; see \cite{Sifakis07} for more details, specifically in resolving the intra-simplex geometry, the discussion of which we exclude here. The essence of the algorithm may be described by considering the resolution of a segmented curve cutting surface against a triangulated area as the volumetric mesh, although the following overview applies equally well to higher dimensions (as seen in Chapters~\ref{chap:partI.virtualsurgery} and \ref{chap:partI.fractureanimation}, for example). Ultimately, the algorithm produces a volumetric mesh geometrically coincident with the original (uncut) mesh with mesh elements along the cutting surface duplicated into topologically and materially disconnected counterparts; see Figure~\ref{fig:partI.cutting.example} for an example overview of the entire procedure.

\setlength{\figurewidth}{\textwidth}
\begin{figure}[htb]
\centering
\includegraphics[width=\figurewidth]{partI/figures/cutter_illustration}
\caption{This mesh is cut by two curves, one of which contains a branch (left). The cutting algorithm first treats each triangle individually, creating duplicates for each locally disjoint material region (center), and then uses the global mesh topology to join these duplicates on the proper degrees of freedom (right). }
\label{fig:partI.cutting.example}
\end{figure}

In the first phase, the algorithm processes each mesh triangle individually, identifying the disjoint \emph{material components} the triangle is divided into by the cutting curve and describing each as a closed polygonal region (depicted blue in Figure~\ref{fig:partI.cutting.triangle.2}). For each such material region, a duplicate copy of the triangle is created and assigned said material region. For example, in Figure~\ref{fig:partI.cutting.triangle.2}, the cutting curve divides the triangle into two distinct material regions, inducing the creation of two duplicates of the original triangle with each duplicate possessing one of the material regions. In the duplicated triangles, we identify vertices within material regions (solid blue circles in Figure~\ref{fig:partI.cutting.triangle.2}) with the original triangle vertices. We also furnish the duplicate triangles with \emph{virtual nodes} (hollow blue circles in Figure~\ref{fig:partI.cutting.triangle.2}). A less trivial example is given in Figure~\ref{fig:partI.cutting.triangle.3}.

\setlength{\figureheight}{0.25\textwidth}
\begin{figure}[htb]
\centering
\includegraphics[height=\figureheight]{partI/figures/cutting}
\caption{Simple example of an original mesh triangle (left) duplicated with its disjoint \emph{material regions} (blue) being distributed among the duplicates (middle, right). Likewise, the vertices in each duplicate are either identified with the vertices of the original triangle (\emph{material nodes}, sold blue) or duplicate copies (\emph{virtual nodes}, hollow blue), depending on whether they fall within a material region.}
\label{fig:partI.cutting.triangle.2}
\end{figure}

\begin{figure}[htbp]
\centering
\includegraphics[height=\figureheight]{partI/figures/complex_cutting}
\caption{A more complex example of the initial division of a triangle by a cutting curve. Since the original mesh triangle is divided into $3$ disjoint material regions (left), the algorithm generates $3$ duplicate triangles, each possessing a different material region (right). (Coloring is consistent with that used in Figure~\ref{fig:partI.cutting.triangle.2}.)}
\label{fig:partI.cutting.triangle.3}
\end{figure}

After processing all triangles in the original mesh $\calT$ individually, we obtain a duplicate mesh $\calT'$ composed of duplicated triangles and vertices. The second phase of the algorithm proceeds to determine global material connectivity within this duplicate mesh (see Figure~\ref{fig:partI.cutting.example}). For notational convenience, let $C(T)$ denote the set of triangles duplicated from an original triangle $T \in \calT$, and let $P(T')$ denote the original ``parent'' triangle of a duplicate triangle $T' \in \calT'$, so that $T' \in C(T)$ if and only if $T = P(T')$. Determining global material connectivity then proceeds as follows. Given a triangle $T' \in \calT'$, let $T = P(T')$. Then for each $U' \in C(U)$ where $U$ is face-adjacent to $T$, determine if $U'$ shares \emph{material connectivity} across the $T-U$ face with $T'$. If so, the vertices of the corresponding faces of $T'$ and $U'$ are identified as equivalent and collapsed, thus joining these duplicate triangles together.

Refer again to Figure~\ref{fig:partI.cutting.example} for an example overview of the entire algorithm. The original mesh, at left, consists of three triangles. This mesh is cut by two segmented curves (red); the geometry typifies some of the subtleties in the algorithm, as the center triangle contains a branch, a tip, and is cut into multiple pieces. In the first phase of the algorithm, each triangle is processed in isolation and duplicated based on the disjoint material regions created by the cutting curves, as shown in the center of Figure~\ref{fig:partI.cutting.example}. In the second phase, these duplicate triangles are joined along faces where they share material connectivity, with the final mesh on the right of Figure~\ref{fig:partI.cutting.example}.

Notice that, in particular, if mesh vertices are identified with degrees of freedom, the latter mesh in Figure~\ref{fig:partI.cutting.example} possesses the necessary richness in degrees of freedom to allow the upper crack to partially separate and the lower crack to separate entirely. This automatic generation of additional degrees of freedom is one of the primary motivations to combine this mesh cutting algorithm with an XFEM framework, and its fruits are demonstrated in the remainder of Part I.

\renewcommand{\thechapter}{\arabic{chapter}}
