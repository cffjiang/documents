\documentclass{article}

\usepackage[left=.75in,top=.75in,bottom=.75in,right=.75in,nohead,nofoot]{geometry}
\usepackage{amsmath}
\pagestyle{empty}

\begin{document}

\begin{center}

{\large \textbf{Jeffrey Lee Hellrung, Jr.}}

\vspace{-12pt}

{\small
2038 S Bentley Avenue \#302  \hfill \verb+jeffrey.hellrung@gmail.com+ \\
Los Angeles, CA \ 90025-5699 \hfill (801) 920-8773
}

\vspace{8pt}

\def\CWa{0.15}
\def\CWb{0.85}
\newlength\HI
\setlength\HI{16pt}
\newlength\VS
\setlength\VS{4pt}
\newlength\VSEX
\setlength\VSEX{2pt}


\begin{tabular}{@{}p{\CWa\columnwidth}@{}p{\CWb\columnwidth}@{}}
{\small OBJECTIVE} &
\begin{minipage}[t]{\CWb\columnwidth}
\par Obtain a postdoctoral research position studying scientific computing problems.
\end{minipage}
\end{tabular}

\vspace{\VS}

\begin{tabular}{@{}p{\CWa\columnwidth}@{}p{\CWb\columnwidth}@{}}
{\small RESEARCH} &
\begin{minipage}[t]{\CWb\columnwidth}
\par Elliptic problems on irregular domains; geometric multigrid; fracture simulation; mesh cutting.
\end{minipage}
\end{tabular}

\vspace{\VS}

\begin{tabular}{@{}p{\CWa\columnwidth}@{}p{\CWb\columnwidth}@{}}
{\small SKILLS} &
\begin{minipage}[t]{\CWb\columnwidth}
\everypar={\hangindent=\HI}
\par Solid computational mathematics and scientific computing background.
\par Proficient primarily in C{\tt ++}, but also familiar with a wide range of other programming languages, packages, and paradigms; including C, Java, \LaTeX, Matlab$^{\rm (R)}$, MIPS assembly, ML/SML, Perl, Python, Qt 4, and Z80 assembly.
\par Active member of the Boost C{\tt ++} open source community.
\par Adept at problem solving, as exemplified by placement in the ``Putnam exam'', Microsoft Imagine Cup Algorithm Invitational, ACM Programming Contest, and ICFP Programming Contest.
\par Mathematical modeling experience from clinic, research in reaction-diffusion PDEs, and the MCM.
\end{minipage}
\end{tabular}

\vspace{\VS}

\begin{tabular}{@{}p{\CWa\columnwidth}@{}p{\CWb\columnwidth}@{}}
{\small EDUCATION} &
\begin{minipage}[t]{\CWb\columnwidth}
\par \textbf{University of California Los Angeles, Los Angeles, CA} \hfill \textbf{GPA} : 3.948 \\
Master of Arts in Mathematics, June 2006 \\
pursuing a Ph.D. in Mathematics, June 2012 (expected) %\\
%\textbf{Coursework} : algebraic topology, algorithms, approximation algorithms, asymptotic methods (2 quar), combinatorics (2 quar), complex analysis (2 quar), computational fluid dynamics, computational linear algebra, continuum mechanics, control and optimization, differential equations (3 quar), differential geometry, differential topology, distributed algorithms, Fourier analysis, Lie groups and Lie algebras, numerical analysis (3 quar), real analysis (4 quar), scientific computing
\vspace{\VS}
\par \textbf{Harvey Mudd College, Claremont, CA} \hfill \textbf{Overall GPA} : 3.858 \\
Bachelor of Science in Mathematics, May 2005     \hfill   \textbf{Major GPA} : 4.000 \\
Graduated with High Distinction, Honors in Mathematics %\\
%\textbf{Coursework} : algebra, algorithms, complex analysis, data structures and program development, differential geometry (2 sem), discrete mathematics, dynamical systems, linear algebra (2 sem), logic, ODEs, PDEs (2 sem), probability theory (1.5 sem), real analysis (2 sem)
%\par \textbf{Mathematics Coursework} : Abstract Algebra, Advanced Linear Algebra, Advanced Probability Theory, Analysis of Partial Differential Equations, Complex Variables and Integral Transforms, Discrete Mathematics, Differential Equations I/II, Dynamical Systems, Elementary Differential Geometry, Intermediate Probability Theory, Introduction to Probability and Statistics, Linear Algebra I/II, Mathematical Analysis I/II, Multivariable Calculus I/II, Putnam Seminar (4 sem), Seminar in Differential Geometry
%\par \textbf{Computer Science Coursework} : Algorithms, Data Structures and Program Development, Logic for Computer Science, Principles of Computer Science, Programming Practicum (4 sem)
%\par \textbf{Other Relevant Coursework} : (Advanced) General Chemistry, Advanced Placement Physics - Mechanics and Wave Motion, Digital Electronics and Computer Engineering, Electromagnetic Theory and Optics, Introduction to Biology, Introduction to Engineering Systems
\end{minipage}
\end{tabular}

\vspace{\VS}

\begin{tabular}{@{}p{\CWa\columnwidth}@{}p{\CWb\columnwidth}@{}}
{\small EXPERIENCE} &
\begin{minipage}[t]{\CWb\columnwidth}
\par \textbf{Park City Mathematics Institute}, Park City, UT; summer 2010.  Teaching assistant for Professor Joseph Teran's week-long workshop on nonlinear elasticity.
\par \textbf{Walt Disney Animation Studios}, Burbank, CA; summer 2008.  Created a Maya plugin to volumetrically fracture nonvolumetric (i.e., quad surfaces) animation models.
\vspace{\VSEX}
\par \textbf{The Aerospace Corporation}, El Segundo, CA; summer 2005, 2006, 2007.  Supported various programming projects relating to SOAP (Satellite Orbit Analysis Program) and SRE (Software Reliability Engineering).  Created and modified existing code in C, C{\tt ++}, and Perl.  Used Qt 4 to develop cross-platform graphical user interfaces in C{\tt ++}.  Designed and prototyped several algorithms to solve satellite positioning problems.
\vspace{\VSEX}
\par \textbf{Teaching Fellow for Department of Mathematics}, UCLA, Los Angeles, CA; 2005 -- 2011.  Led discussion sections to review lecture material, tutored at the Student Math Center, prepared review sessions before exams, and held office hours to provide students with additional help.  %Classes include Math 32A, 32B, 33A, 33B, 132, 151A, 157, 269A (x3), 269B (x2); PIC 60.
\vspace{\VSEX}
\par \textbf{Clinic Project Manager serving Hewlett-Packard}, HMC, Claremont, CA; 2004 -- 2005.  Headed a team of four to deliver to HP Labs several tools written in Matlab$^{\rm (R)}$ to analyze and correct printer drift, as well as mid-year and final reports detailing our approaches.  Presented our findings at Harvey Mudd College and directly to Hewlett-Packard in Palo Alto, CA.
\vspace{\VSEX}
\par \textbf{Auditude$^{\rm TM}$, Inc.}, Los Angeles, CA; summer 2004.  Constructed several test scripts to automate the execution, collection, and compilation of tests and their results, with the aim of optimizing the BroadcastID infrastructure.  Wrote several technical summaries analyzing said results and recommending potential efficiency improvements.
\vspace{\VSEX}
\par \textbf{Research on Spatiotemporal Pattern Formation under Turing Instabilities}, HMC, Claremont, CA; summer 2003.  Programmed (and used) several programs in Matlab$^{\rm (R)}$ (totaling approximately 20,000 lines of code) to solve reaction-diffusion partial differential equations on growing 2-dimensional surfaces.  Implemented several different numerical integration schemes to solve these systems of PDEs.
\vspace{\VSEX}
\par \textbf{Academic Excellence Mathematics Tutor}, HMC, Claremont, CA; 2003 - 2005.  Tutored students in linear algebra, differential equations, multivariable calculus, discrete mathematics; conducted workshops and exam review sessions.
%\par \textbf{Computer Science 60 (Principles of Computer Science) ``Grutor''}, HMC, Claremont, CA; 2002 - 2005.  Graded course assignments; tutored students on the principles of computer science.
%\par \textbf{Mathematics 131 (Mathematical Analysis) Tutor}, HMC, Claremont, CA; spring 2004.  Tutored students in real analysis.
%\par \textbf{Mathematics 173 (Advanced Linear Algebra) Grader, HMC, Claremont, CA; spring 2005.  Graded course assignments.
%\par \textbf{Mathematics 055 (Discrete Mathematics) Grader/Tutor}, HMC, Claremont, CA; fall 2002.  Graded course assignments; tutored students in discrete mathematics.
\par \textbf{Park City Mathematics Institute}, Park City, UT; summer 2003.  Attended lectures on wavelets, partial differential equations, and Fourier analysis.
%\par \textbf{Carpet Installer Assistant for R.C. Willey}, Riverdale, UT, summer 2002.  Pulled up old carpet, padded, laid carpet, stretched, and trimmed.
%\par \textbf{Summer Mathematics Camp}, University of Utah, Salt Lake City, UT, summer 2000.  Studied number theory, graph theory, knot theory, geometry.
\end{minipage}
\end{tabular}

\vspace{\VS}

\begin{tabular}{@{}p{\CWa\columnwidth}@{}p{\CWb\columnwidth}@{}}
{\small PUBLICATIONS} &
\begin{minipage}[t]{\CWb\columnwidth}
\everypar={\hangindent=\HI}
\par ``A Second-Order Virtual Node Algorithm for Nearly Incompressible Linear Elasticity in Irregular Domains.''  Y.~Zhu, Y.~Wang, \textbf{J.~Hellrung}, A.~Cantarero, E.~Sifakis, J.~Teran.  2011.  Submitted.
\par ``A Second-Order Virtual Node Method for Elliptic Problems with Interfaces and Irregular Domains in Three Dimensions.''  \textbf{J.~Hellrung}, L.~Wang, E.~Sifakis, J.~Teran.  Journal of Computational Physics, 2011.  To appear.
\par ``Simulating Crack Propagation with XFEM and a Hybrid Mesh.''  C.~Richardson, J.~Hegemann, E.~Sifakis, \textbf{J.~Hellrung}, J.~Teran.  International Journal for Numerical Methods in Engineering, 2010.  In press.
\par ``Geometric Fracture Modeling in BOLT.'' \textbf{J.~Hellrung}, A.~Selle, A.~Shek, E.~Sifakis, J.~Teran.  ACM SIGGRAPH 2009 (sketch).
\par ``Local Flaps: A Real-Time Finite Element Based Solution to the Plastic Surgery Defect Puzzle.''  E.~Sifakis, \textbf{J.~Hellrung}, J.~Teran, A.~Oliker, and C.~Cutting.  Medicine Meets Virtual Reality 17, 2009.
\par ``Risk assessment of real time digital control systems.''  M.~Hecht, D.~Buettner, \textbf{J.~Hellrung}.  Proceedings of the RAMS '06.  Annual Reliability and Maintainability Symposium, 2006.  Pages 409 - 415.
\end{minipage}
\end{tabular}

\vspace{\VS}

\begin{tabular}{@{}p{\CWa\columnwidth}@{}p{\CWb\columnwidth}@{}}
{\small AWARDS} &
\begin{minipage}[t]{\CWb\columnwidth}
\everypar={\hangindent=\HI}
\par ICFP (International Conference on Functional Programming) Programming Contest - $80^{th}$ (of 215 with positive scores, 872 registered) place (2010) and $95^{th}$ (of 199) place (2011)
\par Google Games Santa Monica - $3^{rd}$ place (2011)
\par VIGRE Fellowship (2005 -- 2009) and Chancellor's Prize (2005 -- 2006) (awarded by UCLA)
\par Robert Borrelli Clinic Prize for Most Outstanding Clinic Team (awarded by HMC) (2005)
\par William Lowell Putnam Mathematical Competition - Top-200 (2003, 2002) and Top-500 (2004, 2001) Individual Placement and $11^{th}$ (2004) Team Placement
\par Microsoft Imagine Cup Algorithm Invitational - $18^{th}$ place internationally (2004)
\par MCM (Mathematical Contest in Modeling) - Meritorious Winner (2004)
\par ACM (Association for Computing Machinery) Programming Contest - $7^{th}$ / 63 place (2004) and $20^{th}$ / 59 place (2003)
%member of the MAA since 01/01/03; score of 10 on A.S.A. Actuarial Examination, Course 1, from the Society of Actuaries, 2003; National AP Scholar; Academic Olympiad at Utah State University, Mar 2001 - gold medal in Mathematics, bronze medal in Science; Utah State Math Contest, Mar 2000 - 4th place in state
\par Stavros Busenberg Prize in Applied Mathematics (awarded by HMC) (2004)
\par Coleman Prize in Mathematics (awarded by HMC) (2003)
\par HMC Dean's List, 2002 -- 2004
\end{minipage}
\end{tabular}

\end{center}

\end{document}
