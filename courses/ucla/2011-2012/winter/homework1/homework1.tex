\documentclass{article}

\usepackage{fullpage}

\usepackage{amsmath}
\usepackage{amssymb}
\usepackage{amsthm}

\providecommand{\abs}[1]{\left\lvert#1\right\rvert}
\providecommand{\norm}[1]{\left\lVert#1\right\rVert}

\begin{document}

\title{Math 269B, 2012 Winter, Homework 1}
\date{January 18, 2012}
\author{Professor Joseph Teran \and Jeffrey Lee Hellrung, Jr.}
\maketitle

\section{Theory}

\begin{itemize}

\item[1.] (Strikwerda 1.1.3) Solve the initial value problem for
\begin{equation*}
u_t + \frac{1}{1 + \frac{1}{2} \cos x} u_x = 0
\end{equation*}
Show that the solution is given by $u(t,x) = u_0(\xi)$, where $\xi$ is the unique solution of
\begin{equation*}
\xi + \frac{1}{2} \sin \xi = x + \frac{1}{2} \sin x - t.
\end{equation*}

\item[2.] Solve the initial value problem
\begin{equation*}
u_t + \left( \sin t \right) u_x = \frac{1}{1 + t^2}, \quad u(0,x) = u_0(x), \quad x \in \mathbb{R}, \quad t > 0.
\end{equation*}

\item[3.] Consider the first order system of PDEs of the form
\begin{equation*}
\vec{u}_t + A \vec{u}_x = 0, \quad \vec{u}(0,x) = \vec{u}_0(x), \quad x \in [0,1], \quad t > 0.
\end{equation*}
\begin{itemize}
\item[(a)] Give the solution to the initial value problem when
\begin{equation*}
A = \begin{pmatrix} 2 & 1 \\ 1 & 2 \end{pmatrix}.
\end{equation*}
\item[(b)] Describe appropriate boundary conditions at $x = 0$ and/or $x = 1$, if possible, which make the initial boundary value problem in (a) well-posed. Try to be as general as possible. How should such boundary conditions be presented to put the solution in a simple form?
\item[(c)] Give the solution to the initial value problem when
\begin{equation*}
A = \begin{pmatrix} 2 & 3 \\ 3 & 2 \end{pmatrix}.
\end{equation*}
\item[(d)] Describe appropriate boundary conditions at $x = 0$ and/or $x = 1$, if possible, which make the initial boundary value problem in (c) well-posed. Try to be as general as possible. How should such boundary conditions be presented to put the solution in a simple form?
\end{itemize}

\item[4.] Derive the leading term of the local truncation error for the following finite difference schemes used to approximate solutions to the equation $u_t + a u_x = 0$.
\begin{itemize}
\item[(a)]
\begin{equation*}
\frac{1}{k} \left( v^{n+1}_m - v^n_m \right) + a \frac{1}{2h} \left( v^n_{m+1} - v^n_{m-1} \right) = 0.
\end{equation*}
\item [(b)]
\begin{equation*}
\frac{1}{k} \left( v^{n+1}_m - \frac{1}{2} \left( v^n_{m+1} + v^n_{m-1} \right) \right) + a \frac{1}{2h} \left( v^n_{m+1} - v^n_{m-1} \right) = 0.
\end{equation*}
\end{itemize}

\item[5.] Determine the stability region $\Lambda$ for each of the finite difference schemes in Problem 4.

\end{itemize}

\section{Programming}

\begin{itemize}
\item[1.] Implement the finite difference schemes in Problem 4.\ in the Theory section for $x \in [0,1]$, $t \in [0,T]$ for some final time $T$, $u(x,0) = u_0(x)$, and \emph{periodic} boundary conditions.
\item[2.] Investigate the convergence of each scheme for $a = 1$ and $T = 1$. Set $k/h =: \lambda$ to be constant, and demonstrate which values of $\lambda$ cause the scheme to converge and which to diverge. If no such $\lambda$ gives convergence, find an alternate relation between $k$ and $h$ which does ensure convergence (if possible). Try using both a smooth initial condition (e.g., $u_0(x) = \sin(2 \pi x)$); a non-smooth initial condition (e.g., $u_0(x) = 1 - 2 \abs{x - 1/2}$); and a discontinuous initial condition (e.g., $u_0(x) = 0$ if $\abs{x - 1/2} > 1/4$ and $u_0(x) = 1$ if $\abs{x - 1/2} < 1/4$). Use the discrete $L^2$ norm to measure the error between your numerical solution and the true solution:
\begin{equation*}
\norm{w}_h = \left( h \sum_m \abs{w_m}^2 \right)^{1/2}.
\end{equation*}
[Note: Due to periodicity, be sure to avoid double-counting the contributions at $x = 0$ and $x = 1$!] Plot the numerical solutions from each scheme at $t = T$ when $h = 1/100$. Summarize your results. Which scheme do you think is better and why?
\end{itemize}

\end{document}
