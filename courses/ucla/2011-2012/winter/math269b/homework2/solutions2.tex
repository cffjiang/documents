\documentclass{article}

\usepackage{fullpage}

\usepackage{amsmath}
\usepackage{amssymb}
\usepackage{amsthm}

\providecommand{\abs}[1]{\left\lvert#1\right\rvert}
\providecommand{\norm}[1]{\left\lVert#1\right\rVert}

\begin{document}

\title{Math 269B, 2012 Winter, Homework 2 (Solutions)}
\date{February 13, 2012}
\author{Professor Joseph Teran \and Jeffrey Lee Hellrung, Jr.}
\maketitle

\section{Theory}

\begin{itemize}

\item[1.] (Strikwerda 2.1.9.) Finite Fourier Transforms. For a function $v_m$ defined on the integers, $m = 0, 1, \dotsc, M-1$, we can define the Fourier transform as
\begin{equation*}
\hat{v}_{\ell} = \sum_{m=0}^{M-1} e^{-2i\pi\ell m/M} v_m \quad \text{for } \ell = 0, \dotsc, M-1.
\end{equation*}
For this transform prove the Fourier inversion formula
\begin{equation*}
v_m = \frac{1}{M} \sum_{\ell=0}^{M-1} e^{2i\pi\ell m/M} \hat{v}_{\ell},
\end{equation*}
and the Parseval's relation
\begin{equation*}
\sum_{m=0}^{M-1} \abs{v_m}^2 = \frac{1}{M} \sum_{\ell=0}^{M-1} \abs{\hat{v}_{\ell}}^2.
\end{equation*}
Note that $v_m$ and $\hat{v}_{\ell}$ can be defined for all integers by making them periodic with period $M$.

\textbf{Solution}

Substituting in for $\hat{v}_{\ell}$ and swapping the order of summation yields
\begin{align*}
  & \frac{1}{M} \sum_{\ell=0}^{M-1} e^{2i\pi\ell m/M} \hat{v}_{\ell} \\
= & \frac{1}{M} \sum_{\ell=0}^{M-1} e^{2i\pi\ell m/M} \sum_{m'=0}^{M-1} e^{-2i\pi\ell m'/M} v_{m'} \\
= & \frac{1}{M} \sum_{m'=0}^{M-1} v_{m'} \sum_{\ell=0}^{M-1} e^{2i\pi\ell \left( m - m' \right) / M}.
\end{align*}
Now, if $m \neq m'$, then
\begin{equation*}
\sum_{\ell=0}^{M-1} e^{2i\pi\ell \left( m - m' \right) /M} = \frac{1 - e^{2i\pi M \left( m - m' \right) /M}}{1 - e^{2i\pi \left( m - m' \right) /M}} = 0;
\end{equation*}
on the other hand, if $m = m'$, then the summand is identically $1$ and hence the sum is $M$. The result follows immediately:
\begin{equation*}
\frac{1}{M} \sum_{\ell=0}^{M-1} e^{2i\pi\ell m/M} \hat{v}_{\ell} = v_m.
\end{equation*}
Regarding the Parseval's relation,
\begin{align*}
\sum_{m=0}^{M-1} \abs{v_m}^2
& = \frac{1}{M^2} \sum_{m=0}^{M-1} \abs{\sum_{\ell=0}^{M-1} e^{2i\pi\ell m/M} \hat{v}_{\ell}}^2 \\
& = \frac{1}{M^2} \sum_{m=0}^{M-1} \left( \sum_{\ell=0}^{M-1} e^{2i\pi\ell m/M} \hat{v}_{\ell} \right) \overline{\left( \sum_{\ell'=0}^{M-1} e^{2i\pi\ell'm/M} \hat{v}_{\ell'} \right)} \\
& = \frac{1}{M^2} \sum_{\ell=0}^{M-1} \sum_{\ell'=0}^{M-1} \sum_{m=0}^{M-1} e^{2i\pi \left( \ell - \ell' \right) m/M} \hat{v}_{\ell} \overline{\hat{v}_{\ell'}} \\
& = \frac{1}{M} \sum_{\ell=0}^{M-1} \abs{\hat{v}_{\ell}}^2,
\end{align*}
as desired (since, when $\ell \neq \ell'$, the inner sum over $m$ sums out to $0$). Alternatively, we my derive Parseval's relation via the inversion formula:
\begin{align*}
\sum_{m=0}^{M-1} \abs{v_m}^2
& = \sum_{m=0}^{M-1} v_m \overline{v_m} \\
& = \frac{1}{M} \sum_{m=0}^{M-1} \sum_{\ell=0}^{M-1} e^{2\pi\ell m/M} \hat{v}_{\ell} \overline{v_m} \\
& = \frac{1}{M} \sum_{\ell=0}^{M-1} \hat{v}_{\ell} \overline{\sum_{m=0}^{M-1} e^{-2\pi\ell m/M} v_m} \\
& = \frac{1}{M} \sum_{\ell=0}^{M-1} \hat{v}_{\ell} \overline{\hat{v}_{\ell}} \\
& = \frac{1}{M} \sum_{\ell=0}^{M-1} \abs{\hat{v}_{\ell}}^2.
\end{align*}

\item[2.] Prove convergence for the Beam-Warming scheme
\begin{equation*}
u^{n+1}_m = u^n_m - \frac{ak}{2h} \left( 3 u^n_m - 4 u^n_{m-1} + u^n_{m-2} \right) + \frac{a^2 k^2}{2h^2} \left( u^n_m - 2 u^n_{m-1} + u^n_{m-2} \right)
\end{equation*}
used to approximate solutions to $u_t + au_x = 0$ for $a > 0$.

\textbf{Solution}

We rewrite the difference operator as
\begin{equation*}
P_{k,h} u^n_m = \frac{1}{k} \left( u^{n+1}_m - u^n_m \right) + \frac{a}{2h} \left( 3 u^n_m - 4 u^n_{m-1} + u^n_{m-2} \right) - \frac{a^2k}{2h^2} \left( u^n_m - 2 u^n_{m-1} + u^n_{m-2} \right).
\end{equation*}
Its symbol is thus
\begin{align*}
p_{k,h}(s,\xi) & = P_{k,h} \left( e^{skn + imh\xi} \right) / e^{skn + imh\xi} \\
               & = \frac{1}{k} \left( e^{sk} - 1 \right) + \frac{a}{2h} \left( 3 - 4 e^{-ih\xi} + e^{-2ih\xi} \right) - \frac{a^2k}{2h^2} \left( 1 - 2 e^{-ih\xi} + e^{-2ih\xi} \right) \\
               & = \frac{1}{k} \left( sk + \frac{1}{2} s^2 k^2 + O \left( k^3 \right) \right) + \frac{a}{2h} \left( 2ih\xi + O \left( h^3 \right) \right) - \frac{a^2k}{2h^2} \left( -h^2 \xi^2 + O \left( h^3 \right) \right) \\
               & = \left( 1 + \frac{k}{2} \left( s - ia\xi \right) \right) \left( s + ia\xi \right) + O \left( k^2 + kh + h^2 \right).
\end{align*}
Since the symbol of the differential operator $P = \partial_t + a \partial_x$ is $p = s + ia\xi$, this demonstrates that $P_{k,h}$ is consistent with $P$ to second order in $k,h$. Incidentally, it also suggests that the symbol $r_{k,h}$ of the difference operator $R_{k,h}$ should be (for example)
\begin{align*}
r_{k,h}(s,\xi) & = 1 + \frac{k}{2} \left( s - ia\xi \right) + O \left( k^2 + kh + h^2 \right) \\
               & = \frac{1}{2} \left( e^{sk} + 1 \right) - \frac{ak}{2h} \left( 1 - e^{-ih\xi} \right),
\end{align*}
giving
\begin{equation*}
R_{k,h} f^n_m = \frac{1}{2} \left( f^{n+1}_m + f^n_m \right) - \frac{ak}{2h} \left( f^n_m - f^n_{m-1} \right).
\end{equation*}

Regarding stability, we substitute $g := e^{sk}$ and solve $p_{k,h} = 0$ for $g$, giving
\begin{align*}
g & = 1 - \frac{ak}{2h} \left( 3 - 4 e^{-i\theta} + e^{-2i\theta} \right) + \frac{a^2k^2}{2h^2} \left( 1 - 2 e^{-i\theta} + e^{-2i\theta} \right) \\
  & = 1 - e^{-i\theta} \frac{ak}{2h} \left( \left( 3 - \frac{ak}{h} \right) e^{i\theta} - 2 \left( 2 - \frac{ak}{h} \right) + \left( 1 - \frac{ak}{h} \right) e^{-i\theta} \right) \\
  & = 1 + e^{-i\theta} \frac{ak}{h} \left( \left( 2 - \frac{ak}{h} \right) \left( 1 - \cos \theta \right) - i \sin \theta \right) \\
  & = e^{-i\theta} \left( \cos \theta + \frac{ak}{h} \left( 2 - \frac{ak}{h} \right) \left( 1 - \cos \theta \right) + i \left( 1 - \frac{ak}{h} \right) \sin \theta \right) \\
  & = e^{-i\theta} \left( 1 - \left( 1 - \frac{ak}{h} \right)^2 \left( 1 - \cos \theta \right) + i \left( 1 - \frac{ak}{h} \right) \sin \theta \right).
\end{align*}
If we let $\alpha = 1 - ak/h$, then
\begin{align*}
\abs{g}^2 & = \abs{1 - \alpha^2 \left( 1 - \cos \theta \right) + i \alpha \sin \theta}^2 \\
          & = 1 + \alpha^4 \left( 1 - \cos \theta \right)^2 + \alpha^2 \sin^2 \theta - 2 \alpha^2 \left( 1 - \cos \theta \right) \\
          & = 1 - \alpha^2 \left( 1 - \alpha^2 \right) \left( 1 - \cos \theta \right)^2,
\end{align*}
hence $\abs{g} \leq 1$ if and only if $\abs{\alpha} \leq 1$, which is equivalent to $0 \leq ak/h \leq 2$.

\item[3.] (Strikwerda 2.2.4.) Show that the box scheme
\begin{equation*}
\frac{1}{2k} \left( \left( v^{n+1}_m + v^{n+1}_{m+1} \right) - \left( v^n_m + v^n_{m+1} \right) \right) + \frac{a}{2h} \left( \left( v^{n+1}_{m+1} - v^{n+1}_m \right) + \left( v^n_{m+1} - v^n_m \right) \right) = f^n_m
\end{equation*}
is consistent with the one-way wave equation $u_t + au_x = f$ and is stable for all values of $\lambda$.

\textbf{Solution}

The symbols corresponding to the difference operators $P_{k,h}$ and $R_{k,h}$ are
\begin{align*}
p_{k,h}(s,\xi) & = P_{k,h} \left( e^{skn + imh\xi} \right) / e^{skn + imh\xi} \\
               & = \frac{1}{2k} \left( e^{sk} - 1 \right) \left( e^{ih\xi} + 1 \right) + \frac{a}{2h} \left( e^{sk} + 1 \right) \left( e^{ih\xi} - 1 \right) \\
               & = \frac{1}{2k} \left( sk + \frac{1}{2} s^2 k^2 + O \left( k^3 \right) \right) \left( 2 + i h \xi + O \left( h^2 \right) \right) \\
               & \quad {} + \frac{a}{2h} \left( 2 + sk + O \left( k^2 \right) \right) \left( ih\xi - \frac{1}{2} h^2 \xi^2 + O \left( h^3 \right) \right) \\
               & = \left( 1 + \frac{1}{2} \left( sk + ih\xi \right) \right) \left( s + ia\xi \right) + O \left( h^2 + h k + k^2 \right); \\
r_{k,h}(s,\xi) & \equiv 1.
\end{align*}
Since the symbol of the differential operator $P = \partial_t + a \partial_x$ is $p = s + ia\xi$, this shows that the box scheme as given by the difference operators $P_{k,h}$ and $R_{k,h}$ is consistent with $P$ to first order in $k,h$. It also shows that we can actually achieve second order accuracy by modifying $r_{k,h}$ to
\begin{align*}
r'_{k,h}(s,\xi) & = 1 + \frac{1}{2} \left( sk + ih\xi \right) + O \left( k^2 + k h + h^2 \right) \\
                & = \frac{1}{4} \left( 1 + e^{sk} \right) \left( 1 + e^{ih\xi} \right),
\end{align*}
giving
\begin{equation*}
R'_{k,h} f^n_m = \frac{1}{4} \left( f^n_m + f^{n+1}_m + f^n_{m+1} + f^{n+1}_{m+1} \right).
\end{equation*}

Regarding stability, we substitute $g := e^{sk}$ and solve $p_{k,h} = 0$ for $g$, giving
\begin{align*}
g & = \frac{e^{i \theta} + 1 - a \lambda \left( e^{i\theta} - 1 \right)}{e^{i\theta} + 1 + a\lambda \left( e^{i\theta} - 1 \right)} \\
  & = \frac{e^{i\theta/2} + e^{-i\theta/2} - a \lambda \left( e^{i\theta/2} - e^{-i\theta/2} \right)}{e^{i\theta/2} + e^{-i\theta/2} + a \lambda \left( e^{i\theta/2} - e^{-i\theta/2} \right)},
\end{align*}
which clearly shows that $\abs{g} \equiv 1$ (the numerator and denominator are complex conjugates).

\item[4.] (Strikwerda 2.2.6.) Determine the stability of the following scheme, sometimes called the Euler backward scheme, for $u_t + au_x = f$:
\begin{align*}
v^{n+1/2}_m & = v^n_m - \frac{a\lambda}{2} \left( v^n_{m+1} - v^n_{m-1} \right) + k f^n_m, \\
v^{n+1}_m & = v^n_m - \frac{a\lambda}{2} \left( v^{n+1/2}_{m+1} - v^{n+1/2}_{m-1} \right) + k f^{n+1}_m.
\end{align*}
The variable $v^{n+1/2}$ is a temporary variable, as is $\tilde{v}$ in Example 2.2.5.

\textbf{Solution}

The symbol $p^{1/2}_{k,h}$ of the difference operator $P^{1/2}_{k,h}$ of the first half-step is
\begin{align*}
p^{1/2}_{k,h}(s,\xi) & = P^{1/2}_{k,h} \left( e^{skn + imh\xi} \right) / e^{skn + imh\xi} \\
                      & = \frac{1}{k} \left( e^{sk} - 1 \right) + \frac{a}{2h} \left( e^{ih\xi} - e^{-ih\xi} \right);
\end{align*}
the symbol $p_{k,h}$ of the full difference operator $P_{k,h}$ is then
\begin{align*}
p_{k,h}(s,\xi) & = P_{k,h} \left( e^{skn + imh\xi} \right) / e^{skn + imh\xi} \\
               & = \frac{1}{k} \left( e^{sk} - 1 \right) + \frac{a}{2h} \left( e^{ih\xi} - e^{-ih\xi} \right) v^{n+1/2} \left( e^{imh\xi} \right) \\
               & = \frac{1}{k} \left( e^{sk} - 1 \right) + \frac{a}{2h}\left( e^{ih\xi} - e^{-ih\xi} \right) \left( 1 - \frac{a\lambda}{2} \left( e^{ih\xi} - e^{-ih\xi} \right) \right).
\end{align*}
Substituting $g := e^{sk}$ and solving $p_{k,h} = 0$ for $g$ yields
\begin{align*}
g & = 1 - \frac{a\lambda}{2} \left( e^{i\theta} - e^{-i\theta} \right) \left( 1 - \frac{a\lambda}{2} \left( e^{i\theta} - e^{-i\theta} \right) \right) \\
  & = 1 - a \lambda i \sin \theta - a^2 \lambda^2 \sin^2 \theta,
\end{align*}
hence
\begin{align*}
\abs{g}^2 & = \left( 1 - a^2 \lambda^2 \sin^2 \theta \right)^2 + a^2 \lambda^2 \sin^2 \theta \\
          & = 1 - a^2 \lambda^2 \sin^2 \theta \left( 1 - a^2 \lambda^2 \sin^2 \theta \right)
\end{align*}
from which we conclude that $\abs{g} \leq 1$ (independent of $\theta$) if and only if $\abs{a\lambda} \leq 1$.

\end{itemize}

\section{Programming}

\begin{itemize}

\item[1.] (Strikwerda 2.3.3.) Solve the initial value problem for equation
\begin{equation*}
u_t + \left( 1 + \frac{1}{4} \left( 3 - x \right) \left( 1 + x \right) \right) u_x = 0
\end{equation*}
on the interval $[-1,3]$ with the Lax-Friedrichs scheme (2.3.1) with $\lambda$ equal to $0.8$. Demonstrate that the instability phenomena occur where $\abs{a(t,x) \lambda}$ is greater than $1$ and where there are discontinuities in the solution. Use the same initial data as in Exercise 2.3.1. Specify the solution to be $0$ at both boundaries. Compute up to the time of $0.2$ and use successively smaller values of $h$ to show the location of the instability.

\textbf{Solution}

\item[2.] Investigate (via numerical evidence) the convergence (or lack thereof) of the forward-time central-space scheme
\begin{equation*}
\frac{1}{k} \left( u^{n+1}_m - u^n_m \right) + \frac{a}{2h} \left( u^n_{m+1} - u^n_{m-1} \right) = 0
\end{equation*}
in the $L^{\infty}$-norm. Use the same scenarios from Homework 1, e.g., compare your results using smooth, continuous-but-non-smooth, and discontinuous initial conditions. Be sure to restrict the relation between $k$ and $h$ appropriately. Compare convergence in the $L^{\infty}$-norm with convergence in the $L^2$-norm.

\textbf{Solution}

We need only run the previous convergence tests for the forward-time central-space scheme with use of the $L^{\infty}$-norm rather than the $L^2$-norm:

\begin{verbatim}
test_convergence( ...
    1, 1, @(x) sin(2*pi*x), ...
    @ftcs, "forward-time central-space", ...
    2.^(-(5:0.5:8)), @(h) h*h, "inf");
test_convergence( ...
    1, 1, @(x) 1 - 2*abs(x - 1/2), ...
    @ftcs, "forward-time central-space", ...
    2.^(-(5:0.5:8)), @(h) h*h, "inf");
test_convergence( ...
    1, 1, @(x) (1 + sign(1/4 - abs(x - 1/2)))/2, ...
    @ftcs, "forward-time central-space", ...
    2.^(-(5:0.5:8)), @(h) h*h, "inf");
\end{verbatim}

The numerical convergence rates are found to be $2.51$, $1.19$, and $0.36$, respectively. (Note that the last convergence test with discontinuous $u_0$ does not give a strong linear relationship in the log-log plot of $h$ vs. $L^2$-error.) Interestingly, these numerical convergence rates are all non-negligibly better than those for the $L^2$-error.

\end{itemize}

\end{document}
