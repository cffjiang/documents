\documentclass{article}

\usepackage{fullpage}

\usepackage{amsmath}
\usepackage{amssymb}
\usepackage{amsthm}

\providecommand{\abs}[1]{\left\lvert#1\right\rvert}
\providecommand{\norm}[1]{\left\lVert#1\right\rVert}

\begin{document}

\title{Math 269B, 2012 Winter, Final}
\date{March 14, 2012}
\author{Professor Joseph Teran \and Jeffrey Lee Hellrung, Jr.}
\maketitle

\section{Theory}

\begin{itemize}

\item[1.] Suppose $u \colon [0,\infty) \times \mathbb{R} \to \mathbb{R}$ satisfies the inviscid Burger's equation,
\begin{equation}\label{eq:burgers.strong}
0 = u_t + \frac{1}{2} \left( u^2 \right)_x = u_t + u u_x, \quad u(0,x) = u_0(x).
\end{equation}
Use the method of characteristics to show that $u$ must then satisfy the implicit relation
\begin{equation}\label{eq:burgers.implicit}
u(t,x) = u_0 \left( x - t u(t,x) \right).
\end{equation}
[Hint: Begin by defining $\tilde{u}(t,X) := u \left( t, \varphi(t,X) \right)$ for some to-be-determined change of variables $\varphi \colon [0,\infty) \times \{X\} \to \{x\}$, and choose $\varphi$ such that $\tilde{u}_t \equiv 0$.]

\item[2.] Suppose $u_0'$ in \eqref{eq:burgers.strong} is bounded below, i.e., $u_0' \geq c$ for some constant $c$. Determine the maximal $T$ such that a solution to \eqref{eq:burgers.implicit} is guaranteed to exist for $t \in [0,T)$ (possibly with $T = \infty$). [Hint: Determine when one can guarantee that the function $u \mapsto u - u_0(x - tu)$ has a unique root.)

\item[3.] Solve \eqref{eq:burgers.strong} for $u_0(x) = ax + b$, where $a,b$ are constants. [Hint: Use \eqref{eq:burgers.implicit}.]

\item[4.] Denote the solution to \eqref{eq:burgers.strong} by $u = F \left[ u_0 \right]$. Express $F \left[ x \mapsto a u_0(x) + b \right]$ in terms of $F \left[ u_0 \right]$.

In other words, given $u$ satisfying \eqref{eq:burgers.strong} for some $u_0$, determine the solution $v$ (in terms of the aforementioned $u$) to
\begin{equation*}
v_t + v v_x = 0, \quad v(0,x) = v_0(x) := a u_0(x) + b.
\end{equation*}

\item[5.] Suppose $u_0$ is given as
\begin{equation*}
u_0(x) := \begin{cases} u^L_0(x) := a_L x + b_L, & x < 0 \\ u^R_0(x) := a_R x + b_R, & x > 0 \end{cases}.
\end{equation*}
Determine the path $t \mapsto \left( t, x_S(t) \right)$ of the (physically correct) shock in the solution $u$ to \eqref{eq:burgers.strong} eminating from $(t,x) = (0,0)$ (assume $b_L \geq b_R$). You may use the fact that
\begin{equation*}
\frac{1}{2} \int \frac{b_L + b_R + \left( a_L b_R + a_R b_L \right) t}{\left( \left( 1 + a_L t \right) \left( 1 + a_R t \right) \right)^{3/2}} dt = \frac{\left( a_L b_R - a_R b_L \right) t + \left( b_R - b_L \right)}{\left( a_L - a_R \right) \sqrt{\left( 1 + a_L t \right) \left( 1 + a_R t \right)}} \quad \left[ a_L \neq a_R \right].
\end{equation*}
[Hint: Recall that the shock speed $x_S'(t) = \frac{1}{2} \left( u^L + u^R \right) \left( t, x_S(t) \right)$, thus allowing you to set up an ordinary differential equation for $x_S$.] Consider and explain the physical significance of the special cases $a_L = a_R$ and $b_L = b_R$.

\item[6.] Solve the weak form of \eqref{eq:burgers.strong} (i.e., give the entropy solution with rarefaction, and with any shocks propagating at the physically correct speed) on the \emph{periodic} domain $[0,4]$ with the ``pulse'' initial condition
\begin{equation}\label{eq:pulse}
u_0(x) := \begin{cases} 0, & 0 \leq x < 1 \\ 2, & 1 < x < 2 \\ 0, & 2 < x \leq 4 \end{cases}.
\end{equation}
Identify key points in time $t$ when the character of the solution changes. (It will be natural to express the solution $u(t,x)$ piecewise with respect to $x$ \emph{and} $t$.) Confirm that $\int u(t,x) dx$ is conserved (i.e., $\int u(t,x) dx = \text{constant}$ for all $t$), and determine $\lim_{t \to \infty} u(t,x)$.

\item[7.] (Strikwerda 6.3.9.) Consider a scheme for (6.1.1), $u_t = b u_{xx}$, of the form
\begin{equation*}
v^{n+1}_m = \left( 1 - 2 \alpha_1 - 2 \alpha_2 \right) v^n_m + \alpha_1 \left( v^n_{m+1} + v^n_{m-1} \right) + \alpha_2 \left( v^n_{m+2} + v^n_{m-2} \right).
\end{equation*}
Show that when $\mu$ is constant, as $k$ and $h$ tend to zero, the scheme is inconsistent unless
\begin{equation*}
\alpha_1 + 4 \alpha_2 = b \mu.
\end{equation*}
Show that the scheme is fourth-order accurate in $x$ if $\alpha_2 = -\alpha_1 / 16$.

\end{itemize}

\section{Programming}

\begin{itemize}

\item[1.] Implement the following numerical schemes to solve \eqref{eq:burgers.strong} on the \emph{periodic} domain $[0,4]$:
\begin{itemize}
\item Godunov's method. At time level $n$, solve the Riemann problem assuming a piecewise constant initial condition $v^n$, then resample to determine $v^{n+1}$.
\item (Backward) Semi-Lagrangian. At time level $n+1$ and grid vertex $m$, trace the characteristic $t \mapsto x_m + v^n_m \left( t - t_{n+1} \right)$ \emph{backward} to time level $n$ and linearly interpolate $v^n$ to determine $v^{n+1}_m$.
\item (Forward) Semi-Lagrangian. Trace the characteristics $t \mapsto x_m + v^n_m \left( t - t_n \right)$ \emph{forward} to time level $n+1$ and linearly interpolate the nearest characteristics at a given grid vertex $m$ to determine $v^{n+1}_m$.
\item (Conservative) Lax-Friedrichs. Discretize the conservative form of (inviscid) Burger's equation:
\begin{equation*}
\frac{1}{k} \left( v^{n+1}_m - \frac{1}{2} \left( v^n_{m+1} - v^n_{m-1} \right) \right) + \frac{1}{2} \cdot \frac{1}{2h} \left( \left( v^n_{m+1} \right)^2 - \left( v^n_{m-1} \right)^2 \right) = 0
\end{equation*}
\item (Advective) Lax-Friedrichs. Discretize the advective form of (inviscid) Burger's equation:
\begin{equation*}
\frac{1}{k} \left( v^{n+1}_m - \frac{1}{2} \left( v^n_{m+1} - v^n_{m-1} \right) \right) + v^n_m \frac{1}{2h} \left( v^n_{m+1} - v^n_{m-1} \right) = 0
\end{equation*}
\end{itemize}
Use the initial condition \eqref{eq:pulse}. For those schemes that appear to converge to the exact solution (derived previously), compute a numerical convergence rate. For those schemes that don't appear to converge to the exact solution, explain the discrepancy (e.g., incorrect rarefaction, non-physical shock speed, unstable). Which scheme do you think performs best for the given initial condition?

\item[2.] Use your implementation of the Thomas algorithm from Homework 4 to solve \emph{periodic} tridiagonal systems:
\begin{equation*}
a_i w_{i-1} + b_i w_i + c_i w_{i+1}, \quad i = 1, \dotsc, m,
\end{equation*}
with $w_0 = w_m$ and $w_{m+1} = w_1$. The following algorithm is described in Strikwerda. First, solve the following (non-periodic) tridiagonal systems:
\begin{align*}
a_i x_{i-1} + b_i x_i + c_i x_{i+1} & = d_i, \quad x_0 = 0 \text{ and } x_{m+1} = 0; \\
a_i y_{i-1} + b_i y_i + c_i y_{i+1} & = 0,   \quad y_0 = 1 \text{ and } y_{m+1} = 0; \\
a_i z_{i-1} + b_i z_i + c_i z_{i+1} & = 0,   \quad z_0 = 0 \text{ and } z_{m+1} = 1;
\end{align*}
for $i = 1, \dotsc, m$. Then $w_i$ is given by
\begin{equation*}
w_i = x_i + r y_i + s z_i
\end{equation*}
where
\begin{align*}
r := {} & \frac{1}{D} \left( x_m \left( 1 - z_1 \right) + x_1 z_m \right), \\
s := {} &  \frac{1}{D} \left( x_m y_1 + x_1 \left( 1 - y_m \right) \right), \\
D := {} & \left( 1 - y_m \right) \left( 1 - z_1 \right) - y_1 z_m.
\end{align*}

\end{itemize}

\end{document}
