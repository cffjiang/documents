\documentclass{article}

\usepackage{fullpage}

\usepackage{amsmath}
\usepackage{amssymb}
\usepackage{amsthm}

\providecommand{\abs}[1]{\left\lvert#1\right\rvert}
\providecommand{\norm}[1]{\left\lVert#1\right\rVert}

\begin{document}

\title{Math 269B, 2012 Winter, Homework 3 (Solutions)}
\date{February 17, 2012}
\author{Professor Joseph Teran \and Jeffrey Lee Hellrung, Jr.}
\maketitle

\section{Theory}

\begin{itemize}

\item[1.] (Strikwerda 5.1.2.) Show that the modified leapfrog scheme (5.1.6) is stable for $\epsilon$ satisfying
\begin{equation*}
0 < \epsilon \leq 1 \quad \text{if} \quad 0 < a^2 \lambda^2 \leq \frac{1}{2}
\end{equation*}
and
\begin{equation*}
0 < \epsilon \leq 4 a^2 \lambda^2 \left( 1 - a^2 \lambda^2 \right) \quad \text{if} \quad \frac{1}{2} \leq a^2 \lambda^2 < 1.
\end{equation*}
Note that these limits are not sharp. It is possible to choose $\epsilon$ larger than these limits and still have the scheme be stable.

\textbf{Solution}

Continuing from the text, we find the amplification factors to be
\begin{equation*}
g_{\pm}(\theta) = -i a \lambda \sin \theta \pm \sqrt{1 - a^2 \lambda^2 \sin^2 \theta - \epsilon \sin^4 \frac{1}{2} \theta}.
\end{equation*}
If the expression under the $\sqrt{\cdot}$ is nonnegative, then
\begin{equation*}
\abs{g_{\pm}(\theta)}^2 = 1 - \epsilon \sin^4 \frac{1}{2} \theta \leq 1,
\end{equation*}
hence the scheme is stable. We thus wish to satisfy
\begin{equation*}
0 \leq 1 - a^2 \lambda^2 \sin^2 \theta - \epsilon \sin^4 \frac{1}{2} \theta =: \alpha(\theta).
\end{equation*}
We compute that
\begin{equation*}
\alpha'(\theta) = -\frac{1}{2} \sin \theta \left( \left( 4 a^2 \lambda^2 - \epsilon \right) \cos \theta + \epsilon \right),
\end{equation*}
and hence the extrema of $\alpha$ occur when $\sin \theta = 0$ or $\cos \theta = \epsilon / \left( \epsilon - 4 a^2 \lambda^2 \right)$. Values of $\theta$ satisfying $\sin \theta = 0$ give $\alpha = 1$ or $\alpha = 1 - \epsilon$, requiring that $\epsilon \leq 1$. Values of $\theta$ satisfying $\cos \theta = \epsilon / \left( \epsilon - 4 a^2 \lambda^2 \right)$ exist if and only if $\abs{\epsilon / \left( \epsilon - 4 a^2 \lambda^2 \right)} \leq 1$, which is equivalent to $\epsilon \leq 2 a^2 \lambda^2$. For such $\theta$, we get $\alpha = 1 - 4 a^4 \lambda^4 / \left( 4 a^2 \lambda^2 - \epsilon \right)$, and for this to be nonnegative, we must have $\epsilon \leq 4 a^2 \lambda^2 \left( 1 - a^2 \lambda^2 \right)$. In particular, we must have $\abs{a \lambda} < 1$.

So far, we have deduced that, at a minimum, $0 < \epsilon \leq 1$. Furthermore, if $\epsilon \leq 2 a^2 \lambda^2$, then we must additionally satisfy $\epsilon \leq 4 a^2 \lambda^2 \left( 1 - a^2 \lambda^2 \right)$. Now, in the instance that $2 a^2 \lambda^2 \leq 4 a^2 \lambda^2 \left( 1 - a^2 \lambda^2 \right)$, we would automatically satisfy the second condition, and this latter inequality is equivalent to $a^2 \lambda^2 \leq \frac{1}{2}$. It follows that
\begin{itemize}
\item If $0 < a^2 \lambda^2 \leq \frac{1}{2}$, it is sufficient to take $0 < \epsilon \leq 1$.
\item If $\frac{1}{2} \leq a^2 \lambda^2 < 1$, it is sufficient to take $0 < \epsilon \leq 4 a^2 \lambda^2 \left( 1 - a^2 \lambda^2 \right)$.
\end{itemize}

\item[2.] Derive the stability condition for the backward-time forward-space scheme
\begin{equation*}
\frac{1}{k} \left( v^{n+1}_m - v^n_m \right) + \frac{a}{h} \left( v^{n+1}_{m+1} - v^{n+1}_m \right) = 0
\end{equation*}
used to approximate solutions to $u_t + a u_x = 0$ with, say, $x \in [0,1]$ and periodic boundary conditions. Give an example of an initial condition $v^0_m$ and an explicit expression for $v^n_m$ that demonstrate unstable behavior for a particular $\lambda$ (your choice) which fails to satisfy the stability condition. Does the growth in your example agree with your theoretical amplification factor?

\textbf{Solution}

Denoting the difference operator of the backward-time forward-space scheme by $P_{k,h}$, we find its corresponding symbol to be
\begin{align*}
p_{k,h}(s,\xi) & = P_{k,h} \left( e^{skn + imh\xi} \right) / e^{skn + imh\xi} \\
               & = \frac{1}{k} \left( e^{sk} - 1 \right) + \frac{a}{h} e^{sk} \left( e^{ih\xi} - 1 \right).
\end{align*}
We determine stability by finding the roots of the symbol as a function of $g := e^{sk}$, yielding
\begin{equation*}
g = \frac{1}{1 + a \lambda \left( e^{i\theta} - 1 \right)}
\end{equation*}
where $\lambda := k/h$ and $\theta := h \xi$. We find that
\begin{equation*}
\abs{g}^{-2} = 1 + 2 a \lambda \left( a \lambda - 1 \right) \left( 1 - \cos \theta \right),
\end{equation*}
hence the scheme is stable ($\abs{g} \leq 1$) if and only if $a \leq 0$ or $a \lambda \geq 1$.

If, for example, $a \lambda = \frac{1}{4}$, then
\begin{equation*}
\abs{g}^{-2} = 1 - \frac{3}{8} \left( 1 - \cos \theta \right) = \frac{5}{8} + \frac{3}{8} \cos \theta.
\end{equation*}
Choosing, for example, $\theta = \pi$ ought to give an amplication factor of exactly $g = 2$ of the pure mode $v_m = e^{i \theta m} = (-1)^m$. Indeed, one can quickly verify that $v^n_m = 2^n (-1)^m$ satisfies the difference equation:
\begin{align*}
k P_{k,h} v^n_m
 & = v^{n+1}_m - v^n_m + a \lambda \left( v^{n+1}_{m+1} - v^{n+1}_m \right) \\
 & = 2^{n+1} (-1)^m - 2^n (-1)^m + \frac{1}{4} \left( 2^{n+1} (-1)^{m+1} - 2^{n+1} (-1)^m \right) \\
 & = 2^n (-1)^m \left( 2 - 1 + \frac{1}{4} \left( -2 -2 \right) \right) \\
 & = 0.
\end{align*}
One final remark: Notice that if $a \lambda = \frac{1}{2}$, $\abs{g}$ is \emph{unbounded} near $\theta = \pi$. This corresponds to a null space in the resulting system of equations for $v^{n+1}$ induced by the difference operator, and this null space is spanned precisely by the mode corresponding to $\theta = \pi$, $v_m = (-1)^m$.

\item[3.] Prove that numerical solutions to the Lax-Friedrichs scheme
\begin{equation*}
\frac{1}{k} \left( v^{n+1}_m - \frac{1}{2} \left( v^n_{m+1} + v^n_{m-1} \right) \right) + \frac{a}{2h} \left( v^n_{m+1} - v^n_{m-1} \right) = 0
\end{equation*}
converge to solutions to the corresponding modified equation
\begin{equation*}
u_t + a u_x = \frac{h^2}{2k} \left( 1 - \left( \frac{a k}{h} \right)^2 \right) u_{xx}
\end{equation*}
to second order accuracy in $L^{\infty}$. I.e., show that $\norm{v^n - u_{k,h} \left( t_n, \cdot \right)}_{\infty} \to 0$ (with $t_n = T$) as $h,k \to 0$ (according to the stability criterion), where the subscripts on $u_{k,h}$ only indicate that the solution to the modified equation is parameterized by $k,h$.

\textbf{Solution}

Denoting the difference operator for the Lax-Friedrichs scheme by $P_{k,h}$, we find its corresponding symbol to be
\begin{align*}
p_{k,h}(s,\xi) & = P_{k,h} \left( e^{skn + imh\xi} \right) / e^{skn + imh\xi} \\
               & = \frac{1}{k} \left( e^{sk} - \cos h \xi \right) + i \frac{a}{h} \sin h \xi \\
               & = \frac{1}{k} \left( 1 + sk + \frac{1}{2} s^2 k^2 - 1 + \frac{1}{2} h^2 \xi^2 \right) + i \frac{a}{h} \left( h \xi \right) + O \left( k^2 + h^2 + h^4 k^{-1} \right) \\
               & = s + i a \xi + \frac{k}{2} s^2 + \frac{h^2}{2k} \xi^2 + O \left( k^2 + h^2 + h^4 k^{-1} \right).
\end{align*}
Thus, $p_{k,h}$ is within $O \left( h^2 \right)$ (assuming $k \sim h$) of the symbol
\begin{equation*}
\tilde{p}^{k,h}(s,\xi) := s + i a \xi + \frac{k}{2} s^2 + \frac{h^2}{2k} \xi^2,
\end{equation*}
which corresponds to the differential operator
\begin{equation*}
\tilde{P}^{k,h} \tilde{u} := \tilde{u}_t + a \tilde{u}_x + \frac{k}{2} \tilde{u}_{tt} - \frac{h^2}{2k} \tilde{u}_{xx}.
\end{equation*}
Observe also that the Lax-Friedrichs scheme $P_{k,h} v^n_m = 0$ is stable in $L^{\infty}$ (so long as $\abs{a\lambda} \leq 1$ with $\lambda := k/h$), hence $\norm{v^n - \tilde{u}_{k,h} \left( t_n, \cdot \right)}_{\infty} \to 0$ (with $t_n = T$) by the Lax-Richtmyer Equivalence Theorem, where $\tilde{u}_{k,h}$ satisfies $\tilde{P}^{k,h} \tilde{u}_{k,h} = 0$.

It remains to compare $\tilde{u}_{k,h}$ from above with $u_{k,h}$ satisfying $P^{k,h} u_{k,h} = 0$, where
\begin{equation*}
P^{k,h} u := u_t + a u_x - \frac{1}{2} \left( \frac{h^2}{k} - a^2 k \right) u_{xx}.
\end{equation*}
Without going into too much detail, one can infer that $P^{k,h} \tilde{u}_{k,h} = O \left( h^2 \right)$ based on $\left( \tilde{u}_{k,h} \right)_{tt} = a^2 \left( \tilde{u}_{k,h} \right)_{xx} + O \left( h \right)$ and well-posedness of $\tilde{P}^{k,h} \tilde{u} = 0$ with respect to the coefficients of $\tilde{P}^{k,h}$. It follows then that $\norm{u^{k,h} \left( t^n, \cdot \right) - \tilde{u}^{k,h} \left( t^n, \cdot \right)}_{\infty} = O \left( h^2 \right)$ based on well-posedness of $P^{k,h} u = f$ with respect to $f$.

\item[4.] (Strikwerda 4.1.2.) Show that the $(2,2)$ leapfrog scheme for $u_t + a u_{xxx} = f$ (see (2.2.15)) given by
\begin{equation*}
\frac{v^{n+1}_m - v^{n-1}_m}{2k} + a \delta^2 \delta_0 v^n_m = f^n_m,
\end{equation*}
with $\nu = k / h^3$ constant, is stable if and only if
\begin{equation*}
\abs{a \nu} < \frac{2}{3^{3/2}}.
\end{equation*}

\textbf{Solution}

Denoting the difference operator of the given leapfrog scheme by $P_{k,h}$, we find its corresponding symbol to be
\begin{align*}
p_{k,h}(s,\xi) & = P_{k,h} \left( e^{skn + imh\xi} \right) / e^{skn + imh\xi} \\
               & = \frac{1}{2k} \left( e^{sk} - e^{-sk} \right) + i \frac{2a}{h^3} \left( \cos \theta - 1 \right) \sin \theta \\
               & = \frac{1}{2k} e^{-sk} \left( e^{2sk} - 4 i a \nu \sin \theta \left( 1 - \cos \theta \right) e^{sk} - 1 \right),
\end{align*}
where $\nu := k/h^3$. We determine stability by finding the roots of the symbol as a function of $g := e^{sk}$, yielding
\begin{equation*}
g_{\pm} = 2 i a \nu \sin \theta \left( 1 - \cos \theta \right) \pm \sqrt{1 - 4 a^2 \nu^2 \sin^2 \theta \left( 1 - \cos \theta \right)^2}.
\end{equation*}
Clearly, if the expression under the $\sqrt{\cdot}$ is nonnegative, then $\abs{g_{\pm}} \equiv 1$, so this would certainly be sufficient for stability. Furthermore, the expression under the $\sqrt{\cdot}$ should never be zero, as that would give $g_+(\theta) = g_-(\theta)$ (for some $\theta$), leading to linear growth. Lastly, if the expression under the $\sqrt{\cdot}$ is ever \emph{negative} (for some $\theta$), then it is easy to see that at least one of $\abs{g_+}$ or $\abs{g_-}$ will be larger than $1$, since, in that case, $\abs{2 a \nu \sin \theta \left( 1 - \cos \theta \right)} > 1$. Stability is thus equivalent to
\begin{equation*}
0 < 1 - 4 a^2 \nu^2 \sin^2 \theta \left( 1 - \cos \theta \right)^2 =: \alpha(\theta)
\end{equation*}
for all $\theta$. We compute that
\begin{equation*}
\alpha'(\theta) = 2 \sin \theta \left( 1 - \cos \theta \right)^2 \left( 1 + 2 \cos \theta \right),
\end{equation*}
and hence the extrema of $\alpha$ occur when $\sin \theta = 0$, $\cos \theta = 1$, or $\cos \theta = -\frac{1}{2}$. The former two cases give $\alpha = 1$, while the latter case gives $\alpha = 1 - \frac{27}{4} a^2 \nu^2$. It follows that $\alpha > 0$ is equivalent to $\abs{a \nu} < 2/3^{3/2}$, as desired.

\item[5.] (Strikwerda 3.2.1.) Show that the (forward-backward) MacCormack scheme
\begin{align*}
\tilde{v}^{n+1}_m & = v^n_m - a \lambda \left( v^n_{m+1} - v^n_m \right) + k f^n_m, \\
v^{n+1}_m & = \frac{1}{2} \left( v^n_m + \tilde{v}^{n+1}_m - a \lambda \left( \tilde{v}^{n+1}_m - \tilde{v}^{n+1}_{m-1} \right) + k f^{n+1}_m \right)
\end{align*}
is a second-order accurate scheme for the one-way wave equation (1.1.1). Show that for $f = 0$ it is identical to the Lax-Wendroff scheme (3.1.1).

\textbf{Solution}

Denoting the difference operators of the given MacCormack scheme by $P_{k,h}$ and $R_{k,h}$, we find the corresponding symbols to be
\begin{align*}
p_{k,h}(s,\xi) & = P_{k,h} \left( e^{skn + imh\xi} \right) / e^{skn + imh\xi} \\
               & = \frac{1}{k} \left( e^{sk} - \frac{1}{2} \left( 1 + \tilde{v}^{n+1} \left( e^{imh\xi}, f^n \equiv 0 \right) / e^{imh\xi} \right) \right) \\
               & \quad {} + \frac{a}{2h} \left( 1 - e^{-ih\xi} \right) \tilde{v}^{n+1} \left( e^{imh\xi}, f^n \equiv 0 \right) / e^{imh\xi} \\
               & = \frac{1}{k} \left( e^{sk} - \frac{1}{2} \left( 2 - a \lambda \left( e^{ih\xi} - 1 \right) \right) \right) + \frac{a}{2h} \left( 1 - e^{-ih\xi} \right) \left( 1 - a \lambda \left( e^{ih\xi} - 1 \right) \right) \\
               & = \frac{1}{k} \left( e^{sk} - 1 \right) + i \frac{a}{h} \sin h \xi + \frac{a^2k}{h^2} \left( 1 - \cos h \xi \right) \\
               & = s + \frac{1}{2} k s^2 + i a \xi + \frac{1}{2} a^2 k \xi^2 + O \left( k^2 + h^2 \right) \\
               & = \left( 1 + \frac{1}{2} k s - \frac{1}{2} i k a \xi \right) \left( s + i a \xi \right) + O \left( k^2 + h^2 \right); \\
r_{k,h}(s,\xi) & = R_{k,h} \left( e^{skn + imh\xi} \right) / e^{skn + imh\xi} \\
               & = \frac{1}{2} \left( \left( \frac{1}{k} - \frac{a}{h} \left( 1 - e^{-ih\xi} \right) \right) \tilde{v}^{n+1} \left( v^n \equiv 0, e^{imh\xi} \right) / e^{imh\xi} + e^{sk} \right) \\
               & = \frac{1}{2} \left( 1 - a \lambda \left( 1 - e^{-ih\xi} \right) + e^{sk} \right) \\
               & = 1 + \frac{1}{2} k s - \frac{1}{2} i k a \xi + O \left( k^2 + kh + h^2 \right).
\end{align*}
Given that the differential operator $P := \partial_t + a \partial_x$ for the one-way wave equation $P u = f$ has symbol $p(s,\xi) = s + i a \xi$, we see that $p_{k,h} - r_{k,h} p = O \left( k^2 + kh + h^2 \right)$, showing second order accuracy. Additionally, $P_{k,h}$ for this MacCormack scheme is identical to Lax-Wendroff, hence the stability condition is identical to that for Lax-Wendroff, $\abs{a\lambda} \leq 1$. By the Lax-Richtmyer Equivalence Theorem, for $\abs{a\lambda} \leq 1$, this MacCormack scheme is second order convergent.

\end{itemize}

\section{Programming}

\begin{itemize}

\item[1.] For the one-way wave equation $u_t + a u_x = 0$, investigate how close the numerical solution to a finite difference scheme is to the solution to the corresponding modified equation. To be concrete, suppose a convenient initial condition for which you can solve the modified equation explicitly with periodic boundary conditions. Take $a = 1$, $k/h = 0.5$, and final time $T = 0.5$. Compare the following finite difference schemes: forward-time backward-space, Lax-Friedrichs, and Lax-Wendroff. Also, include a derivation of the respective corresponding modified equations.

\textbf{Solution}

\begin{itemize}

\item The forward-time backward-space scheme is given by
\begin{equation*}
P_{k,h} v^n_m := \frac{1}{k} \left( v^{n+1}_m - v^n_m \right) + a \frac{1}{h} \left( v^n_m - v^n_{m-1} \right) = 0.
\end{equation*}
Its corresponding symbol is
\begin{align*}
p_{k,h}(s,\xi) := {} & P_{k,h} \left( e^{skn + imh\xi} \right) / e^{skn + imh\xi} \\
                = {} & \frac{1}{k} \left( e^{sk} - 1 \right) + \frac{a}{h} \left( 1 - e^{-ih\xi} \right) \\
                = {} & s + i a \xi + \frac{1}{2} s^2 k + \frac{1}{2} a \xi^2 h + O \left( k^2 + h^2 \right),
\end{align*}
suggesting a modified equation of
\begin{equation*}
u_t + a u_x + \frac{1}{2} k u_{tt} - \frac{1}{2} ah u_{xx} = 0.
\end{equation*}
Of course, if $u$ satisfies the above, then
\begin{equation*}
u_{tt} = a^2 u_{xx} + O \left( k + h \right),
\end{equation*}
giving the modified equation
\begin{equation*}
u_t + a u_x - \frac{1}{2} a \left( h - ak \right) u_{xx} = 0.
\end{equation*}
To determine convenient initial conditions, let us suppose that $u$ takes the form $u_K(t,x) = \alpha(t) e^{2\pi iKx}$ for some integer $K$ (the $e^{2\pi iKx}$ term ensures we satisfy the periodic boundary conditions). Then $\alpha$ must satisfy the ordinary differential equation
\begin{equation*}
\alpha' + \left( 2 \pi i K a + 2 \pi^2 K^2 a \left( h - ak \right) \right) \alpha = 0,
\end{equation*}
giving
\begin{equation*}
u_K(t,x) = e^{-2 \pi^2 K^2 a \left( h - ak \right) t} e^{2 \pi i K \left( x - a t \right)}.
\end{equation*}

The included code tests the forward-time backward-space scheme against the following solution to the modified equation (where the parameters $k,h$ have been suppressed for notational convenience):
\begin{equation*}
u(t,x) = u_{K=1}(t,x) - 2 u_{K=2}(t,x) + 3 u_{K=3}(t,x).
\end{equation*}
The results of the following statement
\begin{verbatim}
test_convergence_ftbs(1, 0.5, 2.^(-(9:0.5:12)), 0.5);
\end{verbatim}
give a numerical convergence rate of $2.97$, which is an order of accuracy better than the theoretical convergence rate of $2$. This is (most likely) due to a fortuitous choice of $\lambda$. For example, if $\lambda = 0.6$, the numerical convergence rate drops to $1.97$, which is more in line with expectations.

\item The Lax-Friedrichs scheme is given by
\begin{equation*}
P_{k,h} v^n_m := \frac{1}{k} \left( v^{n+1}_m - \frac{1}{2} \left( v^n_{m+1} + v^n_{m-1} \right) \right) + a \frac{1}{2h} \left( v^n_{m+1} - v^n_{m-1} \right) = 0.
\end{equation*}
Its corresponding symbol is
\begin{align*}
p_{k,h}(s,\xi) := {} & P_{k,h} \left( e^{skn + imh\xi} \right) / e^{skn + imh\xi} \\
                = {} & \frac{1}{k} \left( e^{sk} - \frac{1}{2} \left( e^{ih\xi} + e^{-ih\xi} \right) \right) + \frac{a}{2h} \left( e^{ih\xi} - e^{-ih\xi} \right) \\
                = {} & \frac{1}{k} \left( e^{sk} - \cos h\xi \right) + i \frac{a}{h} \sin h\xi \\
                = {} & s + i a \xi + \frac{1}{2} s^2 k + \frac{1}{2} \xi^2 \frac{h^2}{k} + O \left( k^2 + h^2 \right),
\end{align*}
suggesting a modified equation of
\begin{equation*}
u_t + a u_x + \frac{1}{2} k u_{tt} - \frac{1}{2} \frac{h^2}{k} u_{xx} = 0.
\end{equation*}
Again, if $u$ satisfies the above, then
\begin{equation*}
u_{tt} = a^2 u_{xx} + O \left( k + h \right),
\end{equation*}
giving the modified equation
\begin{equation*}
u_t + a u_x - \frac{1}{2} \left( \frac{h^2}{k} - a^2 k \right) u_{xx} = 0.
\end{equation*}
To determine convenient initial conditions, we proceed as before and find that
\begin{equation*}
u_K(t,x) = e^{-2 \pi^2 K^2 \left( \frac{h^2}{k} - a^2 k \right) t} e^{2 \pi i K \left( x - a t \right)}
\end{equation*}
satisfies the modified equation for each integer $K$.

The included code tests the Lax-Friedrichs scheme against the following solution to the modified equation (where the parameters $k,h$ have been suppressed for notational convenience):
\begin{equation*}
u(t,x) = u_{K=1}(t,x) - 2 u_{K=2}(t,x) + 3 u_{K=3}(t,x).
\end{equation*}
The results of the following statement
\begin{verbatim}
test_convergence_lax_friedrichs(1, 0.5, 2.^(-(9:0.5:12)), 0.5);
\end{verbatim}
give a numerical convergence rate of $1.90$, which is consistent with the theoretical convergence rate of $2$.

\item The Lax-Wendroff scheme is given by
\begin{equation*}
P_{k,h} v^n_m := \frac{1}{k} \left( v^{n+1}_m - v^n_m \right) + a \frac{1}{2h} \left( v^n_{m+1} - v^n_{m-1} \right) - \frac{a^2k}{2h^2} \left( v^n_{m+1} - 2v^n_m + v^n_{m-1} \right) = 0.
\end{equation*}
Its corresponding symbol is
\begin{align*}
p_{k,h}(s,\xi) := {} & P_{k,h} \left( e^{skn + imh\xi} \right) / e^{skn + imh\xi} \\
                = {} & \frac{1}{k} \left( e^{sk} - 1 \right) + \frac{a}{2h} \left( e^{ih\xi} - e^{-ih\xi} \right) - \frac{a^2k}{2h^2} \left( e^{ih\xi} - 2 + e^{-ih\xi} \right) \\
                = {} & \frac{1}{k} \left( e^{sk} - 1 \right) + i \frac{a}{h} \sin h\xi + \frac{a^2k}{h^2} \left( 1 - \cos h\xi \right) \\
                = {} & \left( 1 + \frac{1}{2} k \left( s - i a \xi \right) \right) \left( s + i a \xi \right) + \frac{1}{6} s^3 k^2 - i \frac{1}{6} a \xi^3 h^2 + O \left( k^3 + h^3 \right),
\end{align*}
suggesting a modified equation of
\begin{equation*}
u_t + a u_x + \frac{1}{6} k^2 u_{ttt} + \frac{1}{6} a h^2 u_{xxx} = 0.
\end{equation*}
Similar to before, if $u$ satisfies the above, then
\begin{equation*}
u_{ttt} = -a^3 u_{xxx} + O \left( k + h \right),
\end{equation*}
giving the modified equation
\begin{equation*}
u_t + a u_x + \frac{1}{6} a \left( h^2 - a^2 k^2 \right) u_{xxx} = 0.
\end{equation*}
To determine convenient initial conditions, we proceed as before and find that
\begin{equation*}
u_K(t,x) = e^{2 \pi i K \left( x - a_K t \right)} \quad \left[ a_K := a \left( 1 - \frac{2}{3} \pi^3 K^2 \left( h^2 - a^2 k^2 \right) \right) \right]
\end{equation*}
satisfies the modified equation for each integer $K$.

The included code tests the Lax-Wendroff scheme against the following solution to the modified equation (where the parameters $k,h$ have been suppressed for notational convenience):
\begin{equation*}
u(t,x) = u_{K=1}(t,x) - 2 u_{K=2}(t,x) + 3 u_{K=3}(t,x).
\end{equation*}
The results of the following statement
\begin{verbatim}
test_convergence_lax_wendroff(1, 0.5, 2.^(-(9:0.5:12)), 0.5);
\end{verbatim}
give a numerical convergence rate of $3.00$, which is identical (to the displayed precision) with the theoretical convergence rate of $3$.

\end{itemize}

\end{itemize}

\end{document}
