\documentclass{article}

\usepackage{fullpage}

\usepackage{amsmath}
\usepackage{amssymb}
\usepackage{amsthm}

\providecommand{\abs}[1]{\left\lvert#1\right\rvert}
\providecommand{\norm}[1]{\left\lVert#1\right\rVert}

\begin{document}

\title{Math 269B, 2012 Winter, Homework 3}
\date{February 17, 2012}
\author{Professor Joseph Teran \and Jeffrey Lee Hellrung, Jr.}
\maketitle

\section{Theory}

\begin{itemize}

\item[1.] (Strikwerda 5.1.2.) Show that the modified leapfrog scheme (5.1.6) is stable for $\epsilon$ satisfying
\begin{equation*}
0 < \epsilon \leq 1 \quad \text{if} \quad 0 < a^2 \lambda^2 \leq \frac{1}{2}
\end{equation*}
and
\begin{equation*}
0 < \epsilon \leq 4 a^2 \lambda^2 \left( 1 - a^2 \lambda^2 \right) \quad \text{if} \quad \frac{1}{2} \leq a^2 \lambda^2 < 1.
\end{equation*}
Note that these limits are not sharp. It is possible to choose $\epsilon$ larger than these limits and still have the scheme be stable.

\item[2.] Derive the stability condition for the backward-time forward-space scheme
\begin{equation*}
\frac{1}{k} \left( v^{n+1}_m - v^n_m \right) + \frac{a}{h} \left( v^{n+1}_{m+1} - v^{n+1}_m \right) = 0
\end{equation*}
used to approximate solutions to $u_t + a u_x = 0$ with, say, $x \in [0,1]$ and periodic boundary conditions. Give an example of an initial condition $v^0_m$ and an explicit expression for $v^n_m$ that demonstrate unstable behavior for a particular $\lambda$ (your choice) which fails to satisfy the stability condition. Does the growth in your example agree with your theoretical amplification factor?

\item[3.] Prove that numerical solutions to the Lax-Friedrichs scheme
\begin{equation*}
\frac{1}{k} \left( v^{n+1}_m - \frac{1}{2} \left( v^n_{m+1} + v^n_{m-1} \right) \right) + \frac{a}{2h} \left( v^n_{m+1} - v^n_{m-1} \right) = 0
\end{equation*}
converge to solutions to the corresponding modified equation
\begin{equation*}
u_t + a u_x = \frac{h^2}{2k} \left( 1 - \left( \frac{a k}{h} \right)^2 \right) u_{xx}
\end{equation*}
to second order accuracy in $\ell^{\infty}$. I.e., show that $\abs{v^n_m - u_{k,h} \left( t_n,x_m \right)} \to 0$ as $h,k \to 0$ (according to the stability criterion), where the subscripts on $u_{k,h}$ only indicate that the solution to the modified equation is parameterized by $k,h$.

\item[4.] (Strikwerda 4.1.2.) Show that the $(2,2)$ leapfrog scheme for $u_t + a u_{xxx} = f$ (see (2.2.15)) given by
\begin{equation*}
\frac{v^{n+1}_m - v^{n-1}_m}{2k} + a \delta^2 \delta_0 v^n_m = f^n_m,
\end{equation*}
with $\nu = k / h^3$ constant, is stable if and only if
\begin{equation*}
\abs{a \nu} < \frac{2}{3^{3/2}}.
\end{equation*}

\item[5.] (Strikwerda 3.2.1.) Show that the (forward-backward) MacCormack scheme
\begin{align*}
\tilde{v}^{n+1}_m & = v^n_m - a \lambda \left( v^n_{m+1} - v^n_m \right) + k f^n_m, \\
v^{n+1}_m & = \frac{1}{2} \left( v^n_m + \tilde{v}^{n+1}_m - a \lambda \left( \tilde{v}^{n+1}_m - \tilde{v}^{n+1}_{m-1} \right) + k f^{n+1}_m \right)
\end{align*}
is a second-order accurate scheme for the one-way wave equation (1.1.1). Show that for $f = 0$ it is identical to the Lax-Wendroff scheme (3.1.1).

\end{itemize}

\section{Programming}

\begin{itemize}

\item[1.] Solve $u_t + a u_x = 0$ numerically using the Lax-Friedrichs scheme. Take $a = 1$, $T = 1$, $x \in [0,1]$ with periodic boundary conditions, and $u_0(x) = \sin 2 \pi x$. Demonstrate convergence for various decreasing values of $\lambda := k/h$ (satisfying the stability criterion) by plotting the logarithm of the $L^2$-norm of the error (between the analytic solution and the numerical solution) versus the logarithm of $h$. Verify that the slope suggested by your plot agrees with theory, and estimate the error constant $C_{\lambda}$ in the relation $\text{error} = C_{\lambda} h^p$. Use enough values of $\lambda$ to estimate the relation between $C_{\lambda}$ and $\lambda$. What appears to happen to $C_{\lambda}$ as $\lambda \to 0+$, i.e., as you shrink $k$ relative to $h$? What happens if, instead of taking $k = \lambda h$, you take $k = h^2$? Explain your numerical results in the context of the theoretical convergence analysis of the Lax-Friedrichs scheme.

\item[2.] For the one-way wave equation $u_t + a u_x = 0$, investigate how close the numerical solution to a finite difference scheme is to the solution to the corresponding modified equation. To be concrete, suppose a pulse initial condition $u_0(x) = \frac{1}{2} \left( 1 + \abs{x}/x \right)$, $x \in [-1,1]$, and periodic boundary conditions. Take $a = 1$, $k/h = 0.5$, and final time $T = 0.5$. Compare the following finite difference schemes: upwinding, Lax-Friedrichs, and Lax-Wendroff. Also, include a derivation of the respective corresponding modified equations. You may find solutions to the modified equations using any appropriate method (i.e., analytically or to a sufficiently high accuracy numerically).

\end{itemize}

\end{document}
