\documentclass{article}

%\usepackage[left=1in,top=1in,bottom=1in,right=1in,nohead,nofoot]{geometry}
\usepackage{fullpage}
\usepackage{amsmath}
\usepackage{amsfonts}
\usepackage{graphicx}


\def\dist{\mathop{\rm dist}\nolimits}
\def\dim{\mathop{\rm dim}\nolimits}
\def\im{\mathop{\rm im}\nolimits}
\def\ker{\mathop{\rm ker}\nolimits}
\def\span{\mathop{\rm span}\nolimits}
\def\rowrank{\mathop{\rm rowrank}\nolimits}
\def\colrank{\mathop{\rm colrank}\nolimits}
\def\nullity{\mathop{\rm nullity}\nolimits}
\newcommand{\matrixiibyii}[4]{\left( \begin{array}{cc} #1 & #2 \\ #3 & #4 \end{array} \right)}


\begin{document}


\begin{flushright}
Jeffrey Hellrung \\
Basic Examination, F03 \\
\end{flushright}


\begin{enumerate}

\item Prove that \(\mathbb{R}\) is uncountable.  If you like to use the Baire Category Theorem, you have to prove it.

{\bf Solution}

It suffices to show that the interval \((0,1]\) is uncountable.  Consider the following isomorphism between \((0,1]\) and a subset of \(S_{0,1}\), the set of sequences composed only of \(0\)'s and \(1\)'s.  Let \(x \in (0,1]\) and express
\[x = \sum_{n = 1}^{\infty} \frac{d_n}{2^n},\]
where each \(d_n \in \{0,1\}\) and \(d_n = 1\) infinitely often.  Note this is always possible, as it is an expression of the following process.  Subdivide \((0,1]\) into \(2^m\) half-open equal-lengthed subintervals of length \(2^{-m}\); thus each subinterval has the form \((x_i, x_{i + 1}]\) with each
\[x_i = \sum_{n = 1}^m \frac{d_n^i}{2^n} = \frac{i}{2^m},\]
\(i = 0, \ldots, 2^m\).  Now note the left endpoint of the subinterval within which \(x\) falls; this is the \(m^{th}\) partial sum of the above infinite sum, which is always bounded above by \(x\) and within \(2^{-m}\) of \(x\).  Further, in choosing the left endpoint of the half-open interval, we guarantee that \(d_n = 1\) infinitely often (if we could represent \(x\) such that \(d_n = 0\) after some point, then \(x\) would be an endpoint of some fine subinterval and \(x\) could never be a partial sum of the above infinite sum, since we always choose the {\em left} endpoint, which is not included in the half-open interval).

This thus provides an injective mapping \(\phi\) from \((0,1]\) into \(S_{0,1}\) by mapping \(x\) to \((d_1, d_2, \ldots)\).  A diagonalization argument shows that \(S_{0,1}\) is itself uncountable, so we have only to show that \(S_{0,1} \backslash \phi((0,1])\) is countable to establish that \((0,1]\) is uncountable.  Note that \(S_{0,1} \backslash \phi((0,1])\) consists of those sequences of \(0\)'s and \(1\)'s which are constantly \(0\) after some point, for all other sequences may be mapped to an \(x \in (0,1]\) by the infinite summation.  Therefore, let \(T_n \subset S_{0,1}\) be the set of sequences which are constantly \(0\) from the \(n^{th}\) term on.  Then each \(T_n\) is finite (indeed, it has cardinality \(2^{n - 1}\)).  Thus
\[S_{0,1} \backslash \phi((0,1]) = \bigcup_{n = 1}^{\infty} T_n,\]
which is certainly at most countable.  This establishes the claim.



\item Let \(f : \mathbb{R} \to \mathbb{R}\) be an infinitely often differentiable function.  Assume that for each element \(x \in [0,1]\) there is a positive integer \(m\) such that the \(m^{th}\) derivative of \(f\) at \(x\) is not zero.

Prove that there exists an integer \(M\) such that the following stronger statement holds:

For each element \(x \in [0,1]\), there is a positive integer \(m\) with \(m \leq M\) such that the \(m^{th}\) derivative of \(f\) at \(x\) is not zero.

{\bf Solution}

As given, for each \(x \in [0,1]\), there exists an \(m_x\) such that the \(m_x^{th}\) derivative of \(f\) at \(x\) is nonzero; by the continuity of \(f^{(m_x)}\), there exists a \(\delta_x\) such that \(f^{(m_x)}\) is nonzero in the neighborhood \(B(x; \delta_x)\) of \(x\).  The family of open sets \(\{B(x; \delta_x)\}_{x \in [0,1]}\) is an open cover of \([0,1]\), which is compact, hence there exists a finite subcover \(\left\{ B \left( x_i; \delta_{x_i} \right) \right\}_{i = 1}^n\).  Set \(M = \max_i m_{x_i}\).  Then each \(x \in [0,1]\) lies in some \(B \left( x_i; \delta_{x_i} \right)\), hence the \(m_{x_i}^{th}\) derivative of \(f\) at \(x\) is nonzero, and \(m_{x_i} \leq M\) by construction.



\item Prove that the sequence \(a_1, a_2, \ldots\) with
\[a_n = \left( 1 + \frac{1}{n} \right)^n\]
converges as \(n \to \infty\).

{\bf Solution}

\[\begin{array}{rcl}
  a_n & = & \left( 1 + \frac{1}{n} \right)^n \\
      & = & \sum_{i = 0}^n \binom{n}{i} \frac{1}{n^i} \\
      & = & \sum_{i = 0}^n \frac{n!}{(n - i)!i!} \frac{1}{n^i} \\
      & = & \sum_{i = 0}^n \frac{n (n - 1) (n - 2) \cdots (n - i + 1)}{(n) (n) (n) \cdots (n)} \frac{1}{i!} \\
      & < & \sum_{i = 0}^n \frac{1}{i!} \\
      & < & \sum_{i = 0}^{\infty} \frac{1}{i!} \\
      & = & e
  \end{array},\]
hence the \(a_n\)'s are a sequence of positive real numbers bounded above by \(e\).



\item Let \(f : \mathbb{R} \to \mathbb{R}\) be a continuous function.  State the definition of the Riemann integral
\[\int_0^1 f(x) dx\]
and prove that it exists.

{\bf Solution}

Let \(P = \{0 = x_0 \leq x_1 \leq \cdots \leq x_n = 1\}\) be a partition of \([0,1]\) and define \(\Delta x_i = x_i - x_{i - 1}\), \(m_i = \inf_{[x_{i - 1}, x_i]} f\), and \(M_i = \sup_{[x_{i - 1}, x_i]} f\), \(i = 1, \ldots, n\).  Then we denote
\[U(P,f) = \sum_{i = 1}^n M_i \Delta x_i,\]
\[L(P,f) = \sum_{i = 1}^n m_i \Delta x_i,\]
and set
\[\overline{\int}_0^1 f dx = \inf_{P \subset [0,1]} U(P,f),\]
\[\underline{\int}_0^1 f dx = \sup_{P \subset [0,1]} L(P,f).\]
If the upper Riemann integral and lower Riemann integral are equal, then \(f\) is Riemann integrable and
\[\int_0^1 f dx\]
is their common value.

Now if \(f\) is continuous on \([0,1]\), then \(f\) is uniformly continuous.  Thus, given an \(\epsilon > 0\), there exists a \(\delta > 0\) such that \(|f(x) - f(y)| < \epsilon\) whenever \(x,y \in [0,1]\) with \(|x - y| < \delta\).  Let \(P\) be a partition of \([0,1]\) such that \(\Delta x_i < \delta/2\).  Then \(M_i - m_i < \epsilon\) and
\[U(P,f) - L(P,f) = \sum_{i = 1}^n (M_i - m_i) \Delta x_i
                  < \epsilon \sum_{i = 1}^n \Delta x_i
                  = \epsilon,\]
and since \(\epsilon\) was aribtrary, we conclude that \(f\) is Riemann integrable on \([0,1]\).



\item Assume \(f : \mathbb{R}^2 \to \mathbb{R}\) is a function such that all partial derivatives of order \(3\) exist and are continuous.  Write down (explicitly in terms of partial derivatives of \(f\)) a quadratic polynomial \(P(x,y)\) in \(x\) and \(y\) such that
\[|f(x,y) - P(x,y)| \leq C(x^2 + y^2)^{3/2}\]
for all \((x,y)\) in some small neighborhood of \((0,0)\), where \(C\) is a number that may depend on \(f\) but not on \(x\) and \(y\).  Then prove the above estimate.

{\bf Solution}

Let \(U\) be some open ball of \((0,0)\), and let \(M\) be a bound for each of the order \(3\) partial derivatives of \(f\) on \(U\) (which exists by the continuity of these partials).  Let
\[P(x,y) = f(0,0) + (D_1f)x + (D_2f)y + (D_{11}f) \frac{x^2}{2} + (D_{12}f)xy + (D_{22}f) \frac{y^2}{2},\]
where each partial derivative is evaluated at \((0,0)\).  Then
\[g(x,y) = f(x,y) - P(x,y)\]
has vanishing partial derivatives of order up to \(2\) at \((0,0)\).  Fix \((x,y) \in U\) and define \(h : [0,1] \to \mathbb{R}\) by
\[h(t) = g(tx,ty).\]
Then
\[h(0) = h'(0) = h''(0) = 0\]
and
\[h^{(3)}(t) = (D_{111}g)x^3 + 3(D_{112}g)x^2y + 3(D_{122}g)xy^2 + (D_{222}g)y^3,\]
where each partial derivative is evaluated at \((tx,ty) \in U\).  Thus
\[\left| h^{(3)}(t) \right| \leq M \left( \max \{x,y\} \right)^3
                            \leq M \left( x^2 + y^2 \right)^{3/2}.\]
By Taylor's Theorem,
\[h(1) = h(0) + h'(0) + h''(0)/2 + h^{(3)}(t)/6\]
for some \(0 \leq t \leq 1\), hence
\[|f(x,y) - P(x,y)| = |g(x,y)| = |h(1)| \leq M/6 \left( x^2 + y^2 \right)^{3/2}\]
for all \((x,y) \in U\).



\item Let \(U = \{(x,y) : x^2 + y^2 < 1\}\) be the standard unit ball in \(\mathbb{R}^2\) and let \(\partial U\) denotes its boundary.

Suppose \(F : \mathbb{R}^2 \to \mathbb{R}^2\) is continuously differentiable and that the Jacobian determinant of \(F\) is everywhere non-zero.  Suppose also that \(F(x,y) \in U\) for some \((x,y) \in U\) and \(F(x,y) \notin U \cup \partial U\) for all \((x,y) \in \partial U\).  Prove that \(U \subset F(U)\).

{\bf Solution}

(W02.7)



\item Prove that the space of continuous functions on the closed interval \([0,1]\) with the metric
\[\dist(f,g) = \sup_{x \in [0,1]} |f(x) - g(x)| = \|f - g\|_{\infty}\]
is complete.  You do not need to show that this is a metric space.

{\bf Solution}

Let \(\{f_n\}_{n = 1}^{\infty}\) be a Cauchy sequence of continuous functions on \([0,1]\) with the metric \(\dist\).  For each \(x \in [0,1]\), it follows that \(\{f_n(x)\}_{n = 1}^{\infty}\) is a Cauchy sequence in \(\mathbb{R}\), hence define \(f : [0,1] \to \mathbb{R}\) by
\[f(x) = \lim_{n \to \infty} f_n(x).\]
We show now that \(f\) is uniformly continuous on \([0,1]\), hence continuous.  To this end, let \(\epsilon > 0\) be given; then there exists an \(N\) such that \(\|f_n - f_m\|_{\infty} < \epsilon\) for all \(n,m \geq N\), that is, \(|f_n(x) - f_m(x)| < \epsilon\) for all \(x \in [0,1]\) and \(n,m \geq N\).  Letting \(m \to \infty\), we see that \(|f_n(x) - f(x)| \leq \epsilon\) for all \(x \in [0,1]\) and \(n \geq N\).  \(f_N\) is continuous on \([0,1]\), hence uniformly continuous, so there exists a \(\delta\) such that \(|f_N(x) - f_N(y)| < \epsilon\) whenever \(|x - y| < \delta\).  It follows that
\[|f(x) - f(y)| \leq |f(x) - f_N(x)| + |f_N(x) - f_N(y)| + |f_N(y) - f(y)| < 3\epsilon,\]
proving that \(f\) is uniformly continuous.



\item Prove the following three statements.  You certainly may choose an order of these statements and then use the earlier statements to prove the later statements.

\begin{enumerate}
\item If \(T : V \to W\) is a linear transformation between two finite dimensional real vector spaces \(V,W\), then
\[\dim(\im(T)) = \dim(V) - \dim(\ker(T)).\]

\item If \(T : V \to V\) is a linear transformation on a finite dimensional real inner product space and \(T^*\) denotes its adjoint, then \(\im(T^*)\) is the orthogonal complement of \(\ker(T)\) in \(V\).

\item Let \(A\) be an \(n\) by \(n\) real matrix; then the maximal number of linearly independent rows (row rank) in the matrix equals the maximal number of linearly independent columns (column rank).

\end{enumerate}

{\bf Solution}

\begin{enumerate}
\item Let \(U\) be any subspace of \(V\), and let \(\{u_1, \ldots, u_k\}\) be a basis for \(U\).  Extend this to a basis of \(V\) with \(\{u_{k + 1}, \ldots, u_n\}\).  Then \(\{\{u_{k + 1}\}, \ldots, \{u_n\}\}\) is evidently a basis for \(V/U\), hence
\[\dim(U) + \dim(V/U) = k + (n - k) = n = \dim(V).\]
Let \(U = \ker(T)\), and define \(T\) acting on \(V/U\) by setting
\[T\{x\} = Tx.\]
Then \(T\) provides an isomorphism between \(V/U\) and \(\im(T)\), hence
\[\dim(V) = \dim(U) + \dim(V/U) = \dim(\ker(T)) + \dim(\im(T)).\]

\item We show first that \(\im(T^*)^{\perp} = \ker(T)\):
\[\begin{array}{*{5}{c}}
  v \in \im(T^*)^{\perp}
  & \Leftrightarrow & (v,x) = 0 \ \forall x \in \im(T^*) & & \\
  & \Leftrightarrow & (v,T^*y) = 0 \ \forall y \in V^* & & \\
  & \Leftrightarrow & (Tv,y) = 0 \ \forall y \in V^* & & \\
  & \Leftrightarrow & Tv = 0
  & \Leftrightarrow & v \in \ker(T)
  \end{array}.\]
We next show that, for a subspace \(W\) of \(V*\), \(W^{\perp\perp} = W\).  Indeed,
\[x \in W \ \Leftrightarrow \ (v,x) = 0 \ \forall v \in W^{\perp}
          \ \Leftrightarrow \ x \in W^{\perp\perp},\]
hence
\[\im(T^*) = \im(T^*)^{\perp\perp} = \ker(T)^{\perp}.\]

\item We have that
\[n = \dim(\im(A^t)) + \dim(\ker(A^t)) = \rowrank(A^t) + \nullity(A^t).\]
But
\[\rowrank(A^t) = \colrank(A)\]
and
\[\nullity(A^t) = n - \dim(\ker(A^t)^{\perp}) = n - \dim(\im(A)) = n - \rowrank(A),\]
from which it follows that \(\colrank(A) = \rowrank(A)\).

\end{enumerate}



\item Consider a \(3\) by \(3\) real symmetric matrix with determinant \(6\).  Assume \((1,2,3)\) and \((0,3,-2)\) are eigenvectors with eigenvalues \(1\) and \(2\).  Give answers to (a) and (b) below and justify the answers.

\begin{enumerate}
\item Give an eigenvector of the form \((1,x,y)\) for some real \(x,y\) which is linearly independent of the two vectors above.

\item What is the eigenvalue of this eigenvector.

\end{enumerate}

{\bf Solution}

Since the product of all (real and complex) eigenvalues of a matrix \(A\) is equal to the determinant \(A\), it follows that if \(A \in M_{3 \times 3}(\mathbb{R})\) with eigenvalues \(1\) and \(2\) and \(\det(A) = 6\), then the third eigenvalue is \(\lambda = 3\).  If, additionally, \(A\) is symmetric, then the eigenspace associated with the eigenvalue \(3\) is orthogonal to the other \(2\) eigenspaces; it follows that \((1,x,y)\) is an eigenvector with eigenvalue \(3\) if and only if
\[0 = (1,x,y) \cdot (1,2,3) = 1 + 2x + 3y,\]
\[0 = (1,x,y) \cdot (0,3,-2) = 3x - 2y,\]
from which we solve \(x = -2/13\), \(y = -3/13\).



\item

\begin{enumerate}
\item Let \(t \in \mathbb{R}\) such that \(t\) is not an integer multiple of \(\pi\).  For the matrix
\[A = \matrixiibyii{\cos(t)}{\sin(t)}{-\sin(t)}{\cos(t)}\]
prove that there does not exist a real valued matrix \(B\) such that \(BAB^{-1}\) is a diagonal matrix.

\item Do the same for the matrix
\[A = \matrixiibyii{1}{\lambda}{0}{1}\]
where \(\lambda \in \mathbb{R} \backslash \{0\}\).

\end{enumerate}

{\bf Solution}

\begin{enumerate}
\item Suppose \(BAB^{-1} = D\) is diagonal for some real-valued matrix \(B\), and let its diagonal entries be \(\lambda_1\) and \(\lambda_2\).  Then each \(\lambda_i\) is an eigenvalue of \(D\), hence an eigenvalue of \(A\) since
\[\begin{array}{rcl}
  0 & = & \det(\lambda_i I - D) \\
    & = & \det(\lambda_i BB^{-1} - BAB^{-1}) \\
    & = & \det(B(\lambda_i I - A)B^{-1}) \\
    & = & \det(\lambda_i I - A)
  \end{array}.\]
But any \(\lambda\) satisfying
\[\det(\lambda I - A) = 0\]
is not real, since
\[0 = \det(\lambda I - A)
    = (\lambda - \cos(t))^2 + \sin(t)^2\]
implies that
\[\lambda = \cos(t) \pm i \sin(t),\]
and \(\sin(t) \neq 0\) as \(t\) is not a multiple of \(\pi\).  But \(D = BAB^{-1}\) must be real-valued since it is a product of real-valued matrices, a contradiction.  This proves there can exist no such \(B\).

\item Suppose \(BAB^{-1} = D\) is diagonal for some real-valued matrix \(B\).  Again, the diagonal entries of \(D\) correspond to eigenvalues of \(A\), so we compute
\[0 = \det(\mu I - A)
    = (\mu - 1)^2,\]
thus \(\mu = 1\) is the only eigenvalue of \(A\), so we conclude that \(D = I\), and \(A = B^{-1}DB = B^{-1}B = I\), which is contradiction to the given \(A\) for \(\lambda \neq 0\).  This proves there can exist no such \(B\).

\end{enumerate}



\end{enumerate}

\end{document}
