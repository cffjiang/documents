\documentclass{article}

%\usepackage[left=1in,top=1in,bottom=1in,right=1in,nohead,nofoot]{geometry}
\usepackage{fullpage}
\usepackage{amsmath}
\usepackage{amsfonts}
\usepackage{graphicx}


\def\tr{\mathop{\rm tr}\nolimits}
\newcommand{\matrixiibyii}[4]{\left( \begin{array}{cc} #1 & #2 \\ #3 & #4 \end{array} \right)}
\newcommand{\matrixiibyi}[2]{\left( \begin{array}{c} #1 \\ #2 \end{array} \right)}


\begin{document}


\begin{flushright}
Jeffrey Hellrung \\
Basic Examination, S03 \\
\end{flushright}


\begin{enumerate}

\item

\begin{enumerate}
\item Suppose \(f : (0,1) \to \mathbb{R}\) is a continuous function.  Define what it means for \(f\) to be {\em uniformly continuous}.

\item Show that if \(f : (0,1) \to \mathbb{R}\) is uniformly continuous, then there is a continuous function \(F : [0,1] \to \mathbb{R}\) with \(F(x) = f(x)\) for all \(x \in (0,1)\).

\end{enumerate}

{\bf Solution}

(S02.4)



\item Prove:  If \(a_1, a_2, a_3, \ldots\) is a sequence of real numbers with
\[\sum_{j = 1}^{\infty} |a_j| < +\infty,\]
then \(\lim_{N \to +\infty} \sum_{j = 1}^N a_j\) exists.

{\bf Solution}

Let
\[S_n = \sum_{j = 1}^n |a_j|,\]
\[T_n = \sum_{j = 1}^n a_j.\]
Given \(\epsilon > 0\), let \(N\) be such that \(|S_n - S_m| < \epsilon\) for all \(n,m > N\).  Then for \(N < n < m\),
\[|T_n - T_m| = \left| \sum_{j = n + 1}^m a_j \right| \leq \sum_{j = n + 1}^m |a_j| < \epsilon,\]
hence \(\{T_n\}_{n = 1}^{\infty}\) is a Cauchy sequence, so has a limit since \(\mathbb{R}\) is complete.



\item Find a subset \(S\) of the real numbers \(\mathbb{R}\) such that both (a) and (b) hold for \(S\):
\begin{enumerate}
\item \(S\) is not the countable union of closed sets.
\item \(S\) is not the countable intersection of open sets.
\end{enumerate}

{\bf Solution}

Let \(S = (\mathbb{Q} \cap (0,\infty)) \cup ((-\infty,0) \backslash \mathbb{Q})\).  Then \(S^{C} = (\mathbb{Q} \cap (-\infty,0]) \cup ((0,\infty) \backslash \mathbb{Q})\).  If \(S\) is a countable intersection of open sets, then \(S \cap (0,\infty) = \mathbb{Q} \cap (0,\infty)\) would be a countable intersection of open sets, which is impossible, following from the Baire Category Theorem.  Similarly, if \(S\) is the countable union of closed sets, \(S^{C}\) is the countable intersection of open sets, which again is impossible for the same reasons.  Hence \(S\) satisfies the given conditions.



\item Consider the following equations for a function \(F(x,y)\) on \(\mathbb{R}^2\):
\[\frac{\partial^2 F}{\partial x^2} = \frac{\partial^2 F}{\partial y^2} \ \ (*)\]

\begin{enumerate}
\item Show that if a function \(F\) has the form \(F(x,y) = f(x + y) + g(x - y)\) where \(f : \mathbb{R} \to \mathbb{R}\) and \(g : \mathbb{R} \to \mathbb{R}\) are twice differentiable, then \(F\) satisfies the equation \((*)\).

\item Show that if \(F(x,y) = ax^2 + bxy + cy^2\), \(a,b,c \in \mathbb{R}\), satisfies \((*)\), then \(F(x,y) = f(x + y) + g(x - y)\) for some polynomials \(f\) and \(g\) in one variable.

\end{enumerate}

{\bf Solution}

\begin{enumerate}
\item Given \(F\) as above, then
\[\frac{\partial^2 F}{\partial x^2} = f''(x + y) + g''(x - y),\]
\[\frac{\partial^2 F}{\partial y^2} = f''(x + y) + g''(x - y),\]
hence \(F\) satisfies the differential equation.

\item Given \(F\) as above, then
\[\frac{\partial^2 F}{\partial x^2} = 2a,\]
\[\frac{\partial^2 F}{\partial y^2} = 2c,\]
hence \(a = c\) and
\[F(x,y) = \frac{a}{2} \left( (x + y)^2 + (x - y)^2 \right)
         + \frac{b}{4} \left( (x + y)^2 - (x - y)^2 \right)
         = f(x + y) + g(x - y)\]
for
\[f(z) = \frac{a}{2} z^2 + \frac{b}{4} z^2,\]
\[g(z) = \frac{a}{2} z^2 - \frac{b}{4} z^2.\]

\end{enumerate}



\item Consider the function \(F(x,y) = ax^2 + 2bxy + cy^2\) on the set \(A = \{(x,y) : x^2 + y^2 = 1\}\).

\begin{enumerate}
\item Show that \(F\) has a maximum and minimum on \(A\).

\item Use Lagrange multipliers to show that if the maximum of \(F\) on \(A\) occurs at a point \((x_0, y_0)\), then the vector \((x_0, y_0)\) is an eigenvector of the matrix \(\matrixiibyii{a}{b}{b}{c}\).

\end{enumerate}

{\bf Solution}

\begin{enumerate}
\item \(F\) is continuous on \(A\), which is compact, hence \(F\) achieves its maximum and minimum.

\item Let
\[g(x,y) = x^2 + y^2 - 1.\]
Then \(F\) achieves its maximum and minimum values whenever \((x,y) \in \mathbb{R}^2\) simultaneously satisfy
\[\nabla F(x,y) = \lambda \nabla g(x,y),\]
\[g(x,y) = 0\]
for some \(\lambda \in \mathbb{R}\).  Thus we compute
\[\nabla F(x,y) = (2ax + 2by, 2bx + 2cy),\]
\[\nabla g(x,y) = (2x, 2y),\]
and substituting into the first condition gives the system
\[2ax + 2by = 2\lambda x,\]
\[2bx + 2cy = 2\lambda y,\]
which is equivalent to
\[\matrixiibyii{a}{b}{b}{c} \matrixiibyi{x}{y} = \lambda \matrixiibyi{x}{y},\]
hence \(\matrixiibyi{x}{y}\) is an eigenvector of \(\matrixiibyii{a}{b}{b}{c}\).
\end{enumerate}



\item Formulate some reasonably general conditions on a function \(f : \mathbb{R}^2 \to \mathbb{R}\) which guarantee that
\[  \frac{\partial}{\partial x} \left( \frac{\partial f}{\partial y} \right)
  = \frac{\partial}{\partial y} \left( \frac{\partial f}{\partial x} \right)\]
and prove that your conditions do in fact guarantee that this equality holds.

{\bf Solution}

(F01.5)



\item Let \(V\) be a finite dimensional real vector space.  If \(W \subset V\) is a subspace, let \(W^{\circ} = \{f : V \to \mathbb{F} \text{ linear} \ | \ f = 0 \text{ on } W\}\).  Let \(W_i \subset V\) be subspaces for \(i = 1, 2\).  Prove that
\[W_1^{\circ} \cap W_2^{\circ} = (W_1 + W_2)^{\circ}.\]

{\bf Solution}

Suppose \(f \in W_1^{\circ} \cap W_2^{\circ}\).  Let \(v \in W_1 + W_2\).  Then \(v = w_1 + w_2\) for some \(w_i \in W_i\).  Thus
\[f(v) = f(w_1 + w_2) = f(w_1) + f(w_2) = 0 + 0 = 0\]
since \(f \in W_1^{\circ}\) as well as \(f \in W_2^{\circ}\).  It follows that \(f \in (W_1 + W_2)^{\circ}\) and \(W_1^{\circ} \cap W_2^{\circ} \subset (W_1 + W_2)^{\circ}\).

Now suppose \(f \in (W_1 + W_2)^{\circ}\).  Then any \(w \in W_1\) can be expressed as \(w_1 + 0 \in W_1 + W_2\) (\(0 \in W_2\) since \(W_2\) is a subspace), hence \(f(w) = 0\) and \(f \in W_1^{\circ}\).  Similarly, \(f \in W_2^{\circ}\) as well, so \(f \in W_1^{\circ} \cap W_2^{\circ}\) and \((W_1 + W_2)^{\circ} \subset W_1^{\circ} \cap W_2^{\circ}\).  This completes the proof of the claim.



\item Let \(V\) be an \(n\)-dimensional complex vector space and \(T : V \to V\) a linear operator.  Suppose that the characteristic polynomial of \(T\) has \(n\) distinct roots.  Show that there is a basis \(B\) for \(V\) such that the matrix representation of \(T\) in the basis \(B\) is diagonal.  (Make sure that you prove your choice of \(B\) is in fact a basis.)

{\bf Solution}

Let the roots of the characteristic polynomial of \(T\) be \(\lambda_1, \ldots, \lambda_n \in \mathbb{R}\), \(\lambda_i \neq \lambda_j\) for \(i \neq j\).  Let \(x_1, \ldots, x_n \in \mathbb{R}^n\) be corresponding eigenvectors, respectively; i.e., \(T x_i = \lambda_i x_i\).  We first show by induction that \(\{x_1, \ldots, x_n\}\) is linearly independent.  For suppose \(\{x_1, \ldots, x_{n-1}\}\) is linearly independent, and suppose some trivial linear relation
\[\sum_{i = 1}^n c_i x_i = 0,\]
\(c_i \in \mathbb{R}\).  Then
\[0 = T \sum_i c_i x_i = \sum_i c_i T x_i = \sum_i c_i \lambda_i x_i.\]
If we multiply the first equation by \(\lambda_n\) and subtract the second, we obtain
\[\sum_{i = 1}^{n - 1} c_i (\lambda_n - \lambda_i) x_i = 0.\]
Since \(\lambda_n \neq \lambda_i\) for \(i = 1, \ldots, n - 1\) and \(\{x_1, \ldots, x_{n - 1}\}\) is linearly independent, it follows that \(c_i = 0\) for each \(i = 1, \ldots, n - 1\), hence \(c_n = 0\) as well.  Thus \(\{x_1, \ldots, x_n\}\) is linearly independent as well, as claimed.

Let \(B = \{x_1, \ldots, x_n\}\).  Then as \(B\) is a linearly independent set of vectors within an \(n\)-dimensional vector space, \(B\) must be a basis for that vector space.  It is evident as well that the matrix representation of \(T\) in the basis \(B\) is diagonal, with diagonal terms \([T]_{ii} = \lambda_i\).



\item Let \(A \in M_3(\mathbb{R})\) satisfy \(\det(A) = 1\) and \(A^tA = I = AA^t\) where \(I\) is the \(3 \times 3\) identity matrix.  Prove that the characteristic polynomial of \(A\) has \(1\) as a root.

{\bf Solution}

Let \(\lambda_i\), \(i = 1,2,3\) be the roots of the characteristic polynomial.  Then \(\lambda_1 \lambda_2 \lambda_3 = \det A = 1\), hence at least one of the \(\lambda_i\)'s is real.  Without loss of generality, suppose \(\lambda_1 \in \mathbb{R}\).

Let \(x_i\) be an associated eigenvector for \(\lambda_i\).  Then
\[(x_i, x_i) = (A^tAx_i, x_i)
             = (Ax_i, Ax_i)
             = (\lambda_i x_i, \lambda_i x_i)
             = |\lambda_i|^2 (x_i, x_i),\]
so \(|\lambda_i|^2 = 1\), hence \(|\lambda_i| = 1\).

Now if \(\lambda_2 \notin \mathbb{R}\), then \(\lambda_3 = \overline{\lambda_2}\) and
\[1 = \lambda_1 \lambda_2 \lambda_3
    = \lambda_1 \lambda_2 \overline{\lambda_2}
    = \lambda_1 |\lambda_2|^2
    = \lambda_1,\]
which proves the claim.  On the other hand, if \(\lambda_2 \in \mathbb{R}\), \(\lambda_3 \in \mathbb{R}\) as well, hence \(\lambda_i \in \{-1,1\}\) and either \(1\) or \(3\) of the \(\lambda_i\)'s will be equal to \(1\) (since their product is \(1\)).



\item Let \(V\) be a finite dimensional real inner product space and \(T : V \to V\) a hermitian linear operator.  Suppose the matrix representation of \(T^2\) in the standard basis has trace zero.  Prove that \(T\) is the zero operator.

{\bf Solution}

In the standard basis \(\{e_1, \ldots, e_n\}\), we have that
\[0 = \tr T^2 = \sum_{i = 1}^n (T^2 e_i, e_i)
              = \sum_{i = 1}^n (Te_i, Te_i)
              = \sum_{i = 1}^n \|Te_i\|^2,\]
hence \(Te_i = 0\) for \(i = 1, \ldots, n\), from which it follows that \(T = 0\).


\end{enumerate}

\end{document}
