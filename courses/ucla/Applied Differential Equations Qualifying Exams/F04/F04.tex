\documentclass{article}

%\usepackage[left=1in,top=1in,bottom=1in,right=1in,nohead,nofoot]{geometry}
\usepackage{fullpage}
\usepackage{amsmath}
\usepackage{amsfonts}
\usepackage{graphicx}



\def\div{\mathop{\rm div}\nolimits}
\def\supp{\mathop{\rm supp}\nolimits}
\def\Res{\mathop{\rm Res}\nolimits}



\begin{document}


\begin{flushright}
Jeffrey Hellrung \\
Applied Differential Equations Qualifying Exam, Fall 2004 \\
\end{flushright}


\begin{enumerate}

\item Solve the following initial-boundary value problem for the wave equation with a potential term,
\[\begin{array}{l}
  (\partial_t^2 - \partial_x^2) u + u = 0, \ 0 < x < \pi, \ t > 0, \\
  u(0,t) = u(\pi,t) = 0, \ t > 0, \\
  u(x,0) = f(x), \ \partial_t u(x,0) = 0, \ 0 < x < \pi,
  \end{array}\]
where
\[f(x) = \begin{cases} x, & \text{if \(x \in (0,\pi/2)\)} \\ \pi - x, & \text{if \(x \in (\pi/2,\pi)\)} \end{cases}.\]
The answer should be given in terms of an infinite series of explicitly given functions.

{\bf Solution}

Supposing \(u(x,t) = X(x) T(t)\), we separate variables:
\[X T'' - X'' T + X T = 0 \ \Rightarrow \ \frac{X''}{X} = \frac{T'' + T}{T} = \lambda\]
for some constant \(\lambda\).  The boundary conditions on \(X\) are \(X(0) = X(\pi) = 0\), hence we find that \(X = \sin(k x)\) and \(\lambda = \lambda_k = -k^2\) for \(k \geq 1\) integral.  Using the boundary condition \(T'(0) = 0\), we then solve \(T\) to be \(T = \cos \left( \sqrt{1 + k^2} t \right)\), so, by linearity,
\[u(x,t) = \sum_{k \geq 1} c_k \cos \left( \sqrt{1 + k^2} t \right) \sin(k x),\]
where, by orthogonality of the \(c_k\)'s,
\[c_k = \frac{2}{\pi} \int_0^{\pi} u(x,0) \sin(k x) dx.\]
Since \(f\) is even about \(\pi/2\), we find that \(c_k = 0\) for even \(k\) since \(\sin(k x)\) is odd about \(\pi/2\).  For odd \(k\), \(\sin(k x)\) is even about \(\pi/2\), so
\begin{eqnarray*}
c_k & = & \frac{2}{\pi} \int_0^{\pi} f(x) \sin(k x) dx \\
    & = & \frac{4}{\pi} \int_0^{\pi} x \sin(k x) dx \\
    & = & \left. -\frac{4}{k \pi} x \cos(k x) \right|_0^{\pi/2} + \frac{4}{k \pi} \int_0^{\pi/2} \cos(k x) dx \\
    & = & \left. \frac{4}{k^2 \pi} \sin(k x) \right|_0^{\pi/2} \\
    & = & (-1)^{(k - 1)/2} \frac{4}{\pi k^2}.
\end{eqnarray*}
Therefore,
\[u(x,t) = \sum_{k \geq 1, \ k \ \text{odd}} (-1)^{(k - 1)/2} \frac{4}{\pi k^2} \cos \left( \sqrt{1 + k^2} t \right) \sin(k x).\]



\item Let \(u(x,t)\) be a bounded solution to the Cauchy problem for the heat equation
\[\left\{ \begin{array}{l} \partial_t u = a^2 \partial_x^2 u, \ t > 0, \ x \in \mathbb{R}, \ a > 0, \\
                           u(x,0) = \phi(x). \end{array} \right.\]
Here \(\phi(x) \in C(\mathbb{R})\) satisfies
\[\lim_{x \to \infty} \phi(x) = b, \ \lim_{x \to -\infty} \phi(x) = c.\]
Compute the limit of \(u(x,t)\) as \(t \to \infty\), \(x \in \mathbb{R}\).  Justify your argument carefully.

{\bf Solution}

We have that \(u\) is given by
\[u(x,t) = \frac{1}{\sqrt{4 \pi t}} \int_{-\infty}^{\infty} e^{-(a x - y)^2 / 4 t} \phi(y) dy
         = \frac{1}{\sqrt{\pi}} \int_{-\infty}^{\infty} e^{-z^2} \phi(y) dz,\]
where \(y = a x + \sqrt{4 t} z\).  Given \(\epsilon > 0\), let \(B > 0\), \(C < 0\) be large enough (in absolute value) such that \(|\phi(x) - b| < \epsilon\) for \(x > B\) and \(|\phi(x) - c| < \epsilon\) for \(x < C\).  Let
\[\beta = \frac{B - a x}{\sqrt{4 t}}, \ \gamma = \frac{-C - a x}{\sqrt{4 t}},\]
such that \(z > \beta\) if and only if \(y > B\), and \(z < \gamma\) if and only if \(y < C\).  In preparing to make some estimations, we decompose the above integral as follows, and estimate each part separately:
\[\int_{-\infty}^{\infty} e^{-z^2} \phi(y) dz
  = \int_{-\infty}^{\gamma} e^{-z^2} \phi(y) dz + \int_{\gamma}^{\beta} e^{-z^2} \phi(y) dz + \int_{\beta}^{\infty} e^{-z^2} \phi(y) dz.\]
We estimate the first integral as
\[\int_{-\infty}^{\gamma} e^{-z^2} \phi(y) dz
  = \int_{-\infty}^{\gamma} c e^{-z^2} dz + \int_{-\infty}^{\gamma} e^{-z^2} (\phi(y) - c) dz.\]
Since \(\gamma \to 0\) as \(t \to \infty\),
\[\left| \int_{-\infty}^{\gamma} c e^{-z^2} dz - \frac{\sqrt{\pi}}{2} c \right| \leq \epsilon\]
for large enough \(t\), while, since \(|\phi(y) - c| < \epsilon\) for \(z < \gamma\),
\[\left| \int_{-\infty}^{\gamma} e^{-z^2} (\phi(y) - c) dz \right| \leq \sqrt{\pi} \epsilon.\]
It follows that
\[\left| \int_{-\infty}^{\gamma} e^{-z^2} \phi(y) dz - \frac{\sqrt{\pi}}{2} c \right| \leq \left( \sqrt{\pi} + 1 \right) \epsilon.\]
Similarly, the third integral admits the estimate
\[\left| \int_{\beta}^{\infty} e^{-z^2} \phi(y) dz - \frac{\sqrt{\pi}}{2} b \right| \leq \left( \sqrt{\pi} + 1 \right) \epsilon\]
for large enough \(t\).  Finally, the middle integral vanishes as \(t \to \infty\), since \(\beta, \gamma \to 0\):
\[\left| \int_{\gamma}^{\beta} e^{-z^2} \phi(y) dy \right| < \epsilon\]
for large enough \(t\).  We thus obtain the estimate
\[\left| \int_{-\infty}^{\infty} e^{-z^2} \phi(y) dz - \frac{\sqrt{\pi}}{2} (b + c) \right| \leq \left( 2 \sqrt{\pi} + 3 \right) \epsilon\]
for large enough \(t\).  It follows that
\[u(x,t) = \frac{1}{\sqrt{\pi}} \int_{-\infty}^{\infty} e^{-z^2} \phi(y) dz \to \frac{1}{2} (b + c)\]
as \(t \to \infty\).



\item Let us consider a damped wave equation,
\[\left\{ \begin{array}{l} (\partial_t^2 - \Delta + a(x) \partial_t) u = 0, \ (x,t) \in \mathbb{R}^3 \times \mathbb{R}, \\
                           u|_{t = 0} = u_0, \ \partial_t u|_{t = 0} = u_1. \end{array} \right.\]
Here the damping coefficient \(a \in C_0^{\infty}(\mathbb{R}^3)\) is a non-negative function with \(u_0, u_1 \in C_0^{\infty}(\mathbb{R}^3)\).  Show that the energy of the solution \(u(x,t)\) at time \(t\),
\[E(t) = \frac{1}{2} \int_{\mathbb{R}^3} \left( |\nabla_x u|^2 + |\partial_t u|^2 \right) dx,\]
is a decreasing function of \(t \geq 0\).

{\bf Solution}

Since \(u_0,u_1 \in C_0^{\infty}(\mathbb{R}^3)\), we also have \(u \in C_0^{\infty}(\mathbb{R}^3)\) by finite propagation speed.  Thus, when integrating by parts in what follows, boundary terms vanish.  With this in mind, we simply compute
\begin{eqnarray*}
E'(t) &   =  & \frac{d}{dt} \left( \frac{1}{2} \int_{\mathbb{R}^3} \left( |\nabla u|^2 + u_t^2 \right) dx \right) \\
      &   =  & \int_{\mathbb{R}^3} \left( \nabla u \cdot \nabla u_t + u_t u_{tt} \right) dx \\
      &   =  & -\int_{\mathbb{R}^3} (\Delta u) u_t dx + \int_{\mathbb{R}^3} u_t \left( \Delta u - a u_t \right) dx \\
      &   =  & -\int_{\mathbb{R}^3} a (u_t)^2 dx \\
      & \leq & 0,
\end{eqnarray*}
hence \(E\) is a decreasing function of \(t\).



\item Prove that each solution (except \(x_1 = x_2 = 0\)) of the autonomous system
\[\left\{ \begin{array}{l} x_1' = x_2 + x_1 \left( x_1^2 + x_2^2 \right) \\ x_2' = -x_1 + x_2 (x_1^2 + x_2^2) \end{array} \right.\]
blows up in finite time.  What is the blow-up time for the solution which starts at the point \((1,0)\) when \(t = 0\)?

{\bf Solution}

Let \(r^2 = x_1^2 + x_2^2\).  Then
\[r r' = x_1 x_1' + x_2 x_2' = r^4 \ \Rightarrow \ r' = r^3,\]
which solves to give
\[r(t) = \frac{r_0}{\sqrt{1 - 2 r_0^2 t}},\]
where \(r_0 = r(0)\).  Thus, solutions will blow up at \(t = 1 / 2 r_0^2\).  For the initial point \((1,0)\), \(r_0 = 1\), so the blow-up time is \(t = 1/2\).



\item Let us consider a generalized Volterra-Lotka system in the plane, given by
\[x'(t) = f(x(t)), \ x(t) \in \mathbb{R}^2, \ \ \ \ (1)\]
where \(f(x) = (f_1(x),f_2(x)) = (a x_1 - b x_1 x_2 - e x_1^2, -c x_2 + d x_1 x_2 - f x_2^2)\), and \(a, b, c, d, e, f\) are positive constants.  Show that
\[\div(\phi f) \neq 0, \ x_1 > 0, \ x_2 > 0,\]
where \(\phi(x_1,x_2) = 1/(x_1 x_2)\).  Using this observation, prove that the autonomous system (1) has no closed orbits in the first quadrant.

{\bf Solution}

Since
\[(\phi f)(x_1,x_2) = \left( \frac{a}{x_2} - b - e \frac{x_1}{x_2}, -\frac{c}{x_1} + d - f \frac{x_2}{x_1} \right),\]
we have simply
\[\div(\phi f)(x_1,x_2) = -\frac{e}{x_2} - \frac{f}{x_2} < 0.\]
Now let \(C \subset \mathbb{R}^2\) be a simple closed \(C^1\)-curve in the first quadrant enclosing a region \(\Omega\), such that \(C = \partial\Omega\) as subsets of \(\mathbb{R}^2\).  From the above, we have that
\[0 > \int_{\Omega} \div(\phi f) dx = \int_C \phi f \cdot \nu ds.\]
Along trajectories of (1), \(f \cdot \nu = 0\), so it follows that \(C\) cannot be a trajectory.  But since \(C\) is arbitrary, we conclude that (1) has no closed orbits.



\item Let \(q \in C_0^1(\mathbb{R}^3)\).  Prove that the vector field
\[u(x) = \frac{1}{4 \pi} \int_{\mathbb{R}^3} \frac{q(y) (x - y)}{|x - y|^3} dy\]
enjoys the following properties:

\begin{enumerate}
\item \(u(x)\) is conservative.

\item \(\div u(x) = q(x)\) for all \(x \in \mathbb{R}^3\).

\item \(|u(x)| = \mathcal{O}(|x|^{-2})\) for large \(x\).

\end{enumerate}

Furthermore, prove that the properties (a), (b), and (c) above determine the vector field \(u(x)\) uniquely.

{\bf Solution}

Define
\[f(x) = -\frac{1}{4 \pi} \int_{\mathbb{R}^3} \frac{q(y)}{|x - y|} dy.\]
Then it is easy to see that \(\nabla f = u\), hence \(u\) is conservative.

To compute the divergence, we first make a change of variables, expressing
\[u(x) = \frac{1}{4 \pi} \int_{\mathbb{R}^3} q(x - z) \frac{z}{|z|^3} dz,\]
so that
\[\div u(x) = \frac{1}{4 \pi} \int_{\mathbb{R}^3} \nabla_x q(x - z) \cdot \frac{z}{|z|^3} dz
            = -\frac{1}{4 \pi} \int_{\mathbb{R}^3} \nabla_z q(x - z) \cdot \frac{z}{|z|^3} dz.\]
Since \(q \in C^1_0(\mathbb{R}^3)\) and \(z / |z|^3\) is integrable near \(0\), we can write
\[\div u(x) = \lim_{\epsilon \searrow 0} -\frac{1}{4 \pi} \int_{|z| \geq \epsilon} \nabla_z q(x - z) \cdot \frac{z}{|z|^3} dz.\]
Integrating by parts, and using the fact that \(q\) vanishes for large enough \(|z|\), we get
\begin{eqnarray*}
\lefteqn{-\frac{1}{4 \pi} \int_{|z| \geq \epsilon} \nabla_x q(x - z) \cdot \frac{z}{|z|^3} dz} \\
& = & -\frac{1}{4 \pi} \int_{|z| = \epsilon} q(x - z) \frac{z}{|z|^3} \cdot \nu dS(z)
    + \frac{1}{4 \pi} \int_{|z| \geq \epsilon} q(x - z) \nabla_z \cdot \left( \frac{z}{|z|^3} \right) dz.
\end{eqnarray*}
It is not hard to show that \(\nabla_z \cdot \left( z / |z|^3 \right)\) vanishes (for \(z\) away from \(0\)):
\begin{eqnarray*}
\nabla_z \cdot \left( \frac{z}{|z|^3} \right)
& = & \frac{\nabla_z(z)}{|z|^3} + z \cdot \nabla_z \left( |z|^{-3} \right) \\
& = & \frac{3}{|z|^3} + z \cdot \left( \frac{-3 z}{|z|^5} \right) \\
& = & 0.
\end{eqnarray*}
This takes care of the second integral.  We evaluate the first integral by noticing that \(\nu = -z / |z|\) (the inward pointing normal) and \(|z| = \epsilon\) on the domain of integration:
\[-\frac{1}{4 \pi} \int_{|z| = \epsilon} q(x - z) \frac{z}{|z|^3} \cdot \nu dS(z)
  = \frac{1}{4 \pi \epsilon^2} \int_{|z| = \epsilon} q(x - z) dS(x)
  \to q(x)\]
as \(\epsilon \searrow 0\), by continuity of \(q\).  The claim then immediately follows:
\[\div u(x) = \lim_{\epsilon \searrow 0} -\frac{1}{4 \pi} \int_{|z| \geq \epsilon} \nabla_z q(x - z) \cdot \frac{z}{|z|^3} dz = q(x).\]

For the decay claim, let \(R > 0\) be large enough such that \(B_R(0) \supset \supp q\), and let \(M > 0\) be such that \(q < M\).  Then for large \(x\), \(|x - y| \geq |x| - R\) for \(y \in \supp q\), so
\[|u(x)| = \left| \frac{1}{4 \pi} \int_{\mathbb{R}^3} \frac{q(y) (x - y)}{|x - y|^3} dy \right|
         < \frac{M}{4 \pi} \int_{\mathbb{R}^3} \frac{|x| + R}{(|x| - R)^3} dy
         = \mathcal{O}(|x|^{-2}).\]

We now address the uniqueness claim.  First, \(u\) being conservative implies that \(u = \nabla f\) for some \(f\), and hence \(q = \div u = \Delta f\), i.e., \(f\) satisfies a Poisson equation.  The decay of \(u\) guarantees the solution \(f\) to be unique, and we know that \(f\) is given by convolution with the fundamental solution:
\[f(x) = (K * q)(x) = -\frac{1}{4 \pi} \int_{\mathbb{R}^3} \frac{q(y)}{|x - y|} dy.\]
It follows that \(u\) is as given.



\item Consider the partial differential equation
\[u u_x + u_t + u = 0, \ (z,t) \in \mathbb{R}^2.\]

\begin{itemize}
\item Find the particular solution that satisfies the condition \(u(0,t) = e^{-2 t}\).

\item Show that at the point \((z,t) = (1/9, \log 2)\), \(u = 1/3\).

\end{itemize}

{\bf Solution}

\begin{itemize}
\item We let \(x = z\) and \(y = t\), for notational convenience in applying the method of characteristics, giving the PDE \(u u_x + u_y + u = 0\).  The initial condition curve may be parametrized by \(s \mapsto (0, s, e^{-2 s}) = (x_0,y_0,z_0)\).  The method of characteristics yields the system of ODEs
\begin{eqnarray*}
x' & = & z; \\
y' & = & 1; \\
z' & = & -z.
\end{eqnarray*}
\(y\) and \(z\) may be solved immediately:
\begin{eqnarray*}
y & = & t + y_0 = t + s; \\
z & = & z_0 e^{-t} = e^{-2 s} e^{-t}.
\end{eqnarray*}
\(x\) may now be found:
\[x = e^{-2 s} \left( 1 - e^{-t} \right) + x_0 = e^{-2 s} \left( 1 - e^{-t} \right).\]
We can solve for \(s,t\) in terms of \(x,y\), giving the relations
\[t = y - s, \ e^{-s} = \frac{1}{2} e^{-y} \left( 1 + \sqrt{1 + 4 x e^{2 y}} \right).\]
The solution is thus
\[u(x,y) = z = e^{-y} e^{-s} = \frac{1}{2} e^{-2 y} \left( 1 + \sqrt{1 + 4 x e^{2 y}} \right).\]

\item We compute
\[u(1/9, \log 2) = \frac{1}{2} \left( \frac{1}{4} \right) \left( 1 + \sqrt{1 + 4 \left( \frac{1}{9} \right) (4)} \right) = \frac{1}{3}.\]

\end{itemize}



\item The function \(y(x,t)\) satisfies the partial differential equation
\[x \frac{\partial y}{\partial x} + \frac{\partial^2 y}{\partial x \partial t} + 2 y = 0,\]
and the boundary conditions
\[y(x,0) = 1, \ y(0,t) = e^{-a t},\]
where \(a > 0\).  Find the Laplace transform, \(\overline{y}(x,s)\), of the solution, and hence derive an expression for \(y(x,t)\) in the domain \(x \geq 0\), \(t \geq 0\).

{\bf Solution}

Recall that the Laplace transform is given by
\[\mathcal{L}_t(f(t))(s) = \int_0^{\infty} e^{-s t} f(t) dt.\]
It is easy to derive that
\[\mathcal{L}_t(f'(t))(s) = f(0+) + s \mathcal{L}_t(f(t))(s),\]
and so applying the Laplace transform to the PDE results in
\[x \overline{y}_x + s \overline{y}_x + 2 \overline{y} = 0,\]
where \(\overline{y} = \overline{y}(x,s) = \mathcal{L}_t(y(x,t))(s)\).  The boundary condition transforms to
\[\overline{y}(0,s) = \frac{1}{s + a},\]
and so this solves easily to
\[\overline{y}(x,s) = \frac{s^2}{(x + s)^2 (s + a)}.\]
We can recover \(y\) by the inverse Laplace transform via contour integration:
\begin{eqnarray*}
y(x,t) & = & \frac{1}{2 \pi i} \int_{\gamma - i \infty}^{\gamma + i \infty} \overline{y}(x,s) e^{s t} ds \\
       & = & \Res \left( \overline{y}(x,s) e^{s t}; s = -a \right) + \Res \left( \overline{y}(x,s) e^{s t}; s = -x \right) \\
       & = & \left( \lim_{s \to -a} (s + a) \overline{y}(x,s) e^{s t} \right) + \left( \lim_{s \to -x} \frac{\partial}{\partial s} (s + x)^2 \overline{y}(x,s) e^{s t} \right) \\
       & = & \left( \lim_{s \to -a} \frac{s^2}{(x + s)^2} e^{s t} \right) + \left( \lim_{s \to -x} \frac{\partial}{\partial s} \frac{s^2}{s + a} e^{s t} \right) \\
       & = & \frac{a^2}{(x - a)^2} e^{-a t} + \left( \lim_{s \to -x} \frac{s + 2 a + s^2 t + a s t}{(s + a)^2} s e^{s t} \right) \\
       & = & \frac{a^2}{(x - a)^2} e^{-a t} + \frac{-x + 2 a + x^2 t - a x t}{(x - a)^2} (-x) e^{-x t} \\
       & = & \frac{1}{(x - a)^2} \left( a^2 e^{-a t} + \left( x - 2 a - x^2 t + a x t \right) x e^{-x t} \right).
\end{eqnarray*}



\end{enumerate}

\end{document}
