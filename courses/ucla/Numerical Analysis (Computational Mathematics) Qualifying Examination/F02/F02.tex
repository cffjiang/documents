\documentclass{article}

%\usepackage[left=1in,top=1in,bottom=1in,right=1in,nohead,nofoot]{geometry}
\usepackage{fullpage}
\usepackage{amsmath}
\usepackage{amsfonts}
\usepackage{graphicx}


\begin{document}


\begin{flushright}
Jeffrey Hellrung \\
Numerical Analysis Qualifying Exam, Fall 2002 \\
\end{flushright}


\begin{enumerate}

\item Consider using the composite trapezoidal method to numerically evalute the following integral:
\[(I) \ \int_0^1 \frac{\sin t}{\sqrt{t}} dt.\]
Two different methods are employed:
\begin{enumerate}
\item The composite trapezoidal method is directly applied to the integral \((I)\) and the avlue of the integrand at \(t = 0\) is taken to be \(0\).
\item The composite trapezoidal method is applied to
\[(I') \ \int_0^1 2 \sin s^2 ds.\]
(This latter integral is obtained from \((I)\) by using the change of variables \(s = \sqrt{t}\).)
\end{enumerate}

The errors in the numerical approximation for these computations are given in the following table.

\begin{tabular}{c|c|c|}
\(\Delta x\) & error with computation (a) & error with computation (b) \\
\hline
 & \(-5.840e-04\) & \(7.204e-05\) \\
 & \(-2.068e-04\) & \(1.800e-05\) \\
 & \(-7.325e-05\) & \(4.500e-06\) \\
\end{tabular}

\begin{enumerate}
\item What is the expected rate of convergence for the composite trapezoidal method?

\item Give an estimate, based on the results in the above table, of the rate of convergence for each of the computational procedures.

\item If your estimate of convergence does not agree with the expected rate of convergence for either of these procedures, explain this discrepancy.

\end{enumerate}

{\bf Solution}

\begin{enumerate}
\item The expected rate of convergence for the composite trapezoidal method is second-order (i.e., the error is \(O(h^2)\)).

\item The errors in computation (a) seem to be \(O(h^{1 + \epsilon})\) for some relatively small \(\epsilon > 0\); the errors for computation (b) seem to be \(O(h^2)\).

\item For computation (a), the integrand fails to be differentiable at \(t = 0\), which would explain the less-than-expected rate of convergence.

\end{enumerate}



\item Consider the two-point boundary-value problem over the interval \([0,1]\):
\[\frac{d}{dx} \left( p(x) \frac{du}{dx}(x) \right) = f(x), \ u(0) = u(1) = 0;\]
with \(p(x) > 0\).

\begin{enumerate}
\item Assuming you are using an equispaced set of grid points in \([0,1]\), give a finite difference discretization of this equation that results in a {\em symmetric} linear system of equations.

\item Derive the leading term of the truncation error for the discretization in (a).

\end{enumerate}

{\bf Solution}

\begin{enumerate}
\item We use
\[p_{m-1/2} u_{m-1} - \left( p_{m-1/2} + p_{m+1/2} \right) u_m + p_{m+1/2} u_{m+1} = h^2 f_m, \ m = 1, \ldots, N - 1;\]
where \(p_{m\pm1/2} = p((m\pm1/2)h)\), \(f_m = f(mh)\), \(u_m\) is the approximation to \(u(mh)\), and \(h = 1/N\).  This may be rewritten in matrix form as
\[\left( \begin{array}{*{5}{c@{\ }}}
         -(p_{\frac{1}{2}} + p_{\frac{3}{2}}) & p_{\frac{3}{2}} \\
         p_{\frac{3}{2}} & -(p_{\frac{3}{2}} + p_{\frac{5}{2}}) & p_{\frac{5}{2}} \\
         & p_{\frac{5}{2}} & -(p_{\frac{5}{2}} + p_{\frac{7}{2}}) & p_{\frac{7}{2}} \\
         & & \ddots & \ddots & \ddots \\
         & & & p_{N-\frac{3}{2}} & -(p_{N-\frac{3}{2}} + p_{N-\frac{1}{2}}) \\
         \end{array} \right)
  \left( \begin{array}{c} u_1 \\ u_2 \\ u_3 \\ \vdots \\ u_{N-1} \end{array} \right)
  = h^2 \left( \begin{array}{c} f_1 \\ f_2 \\ f_3 \\ \vdots \\ f_{N-1} \end{array} \right).\]

\item We rewrite the system, where \(x = mh\), as
\[\frac{1}{h} \left( p \left( x + \frac{h}{2} \right) \frac{u(x + h) - u(x)}{h} - p \left( x - \frac{h}{2} \right) \frac{u(x) - u(x - h)}{h} \right) = f(x).\]
By Taylor's Theorem,
\begin{eqnarray*}
\frac{u(x + h) - u(x)}{h} & = & u' \left( x + \frac{h}{2} \right) + \frac{1}{24} u^{(3)} \left( x + \frac{h}{2} \right) h^2 + O(h^4), \\
\frac{u(x) - u(x - h)}{h} & = & u' \left( x - \frac{h}{2} \right) + \frac{1}{24} u^{(3)} \left( x - \frac{h}{2} \right) h^2 + O(h^4),
\end{eqnarray*}
\begin{eqnarray*}
\frac{1}{h} \left( p \left( x + \frac{h}{2} \right) u' \left( x + \frac{h}{2} \right) - p \left( x - \frac{h}{2} \right) u' \left( x - \frac{h}{2} \right) \right) & = & (pu')'(x) + O(h^2), \\
\frac{1}{h} \left( p \left( x + \frac{h}{2} \right) u^{(3)} \left( x + \frac{h}{2} \right) - p \left( x - \frac{h}{2} \right) u^{(3)} \left( x - \frac{h}{2} \right) \right) & = & \left( pu^{(3)} \right)'(x) + O(h^2),
\end{eqnarray*}
so we finally get
\[\frac{1}{h} \left( p \left( x + \frac{h}{2} \right) \frac{u(x + h) - u(x)}{h} - p \left( x - \frac{h}{2} \right) \frac{u(x) - u(x - h)}{h} \right) = (pu')'(x) + O(h^2),\]
giving a truncation error \(\in O(h^2)\).

\end{enumerate}




\end{enumerate}

\end{document}
