\documentclass{article}

%\usepackage[left=1in,top=1in,bottom=1in,right=1in,nohead,nofoot]{geometry}
\usepackage{fullpage}
\usepackage{amsmath}
\usepackage{amsfonts}
\usepackage{graphicx}


\begin{document}


\begin{flushright}
Jeffrey Hellrung \\
Numerical Analysis Qualifying Exam, Fall 2005 \\
\end{flushright}


\begin{enumerate}

\item (5 Pts.) Let \(\{x_n\}\) be a sequence such that \(x_n \geq \overline{x}\) for all \(n\) and \(\lim_{n \to \infty} x_n = \overline{x}\).  Assume there exist constants \(\alpha\) and \(p > 0\) such that, for sufficiently large \(n\),
\[x_{n+1} - \overline{x} \approx \alpha \left( x_n - \overline{x} \right)^p.\]

\begin{enumerate}
\item Assuming \(\overline{x}\) is known, give a derivation of a formula that estimates \(p\) in terms of \(\overline{x}\) and some number of consecutive iterates of the sequence \(\{x_n\}\).

\item Assuming \(\overline{x}\) is {\em unknown}, give a derivation of a formula that estimates \(p\) in terms of some number of consecutive iterates of the sequence \(\{x_n\}\).

\end{enumerate}

{\bf Solution}

\begin{enumerate}
\item Denote \(e_n = x_n - \overline{x}\).  Then the givens stipulate that, for large enough \(n\),
\[e_{n+1} \approx \alpha e_n^p,\]
in which case we also have
\[e_{n+2} \approx \alpha e_{n+1}^p,\]
so
\[\frac{e_{n+1}}{e_{n+2}} \approx \left( \frac{e_n}{e_{n+1}} \right)^p \ 
  \Rightarrow \ p \approx \frac{\ln e_{n+1} - \ln e_{n+2}}{\ln e_n - \ln e_{n+1}}.\]

\item We suppose that \(e_n = x_n - \overline{x} \approx x_n - x_{n+1}\), and similarly for \(e_{n+1}\) and \(e_{n+2}\), obtaining
\[p \approx \frac{\ln(x_{n+1} - x_{n+2}) - \ln(x_{n+2} - x_{n+3})}
                 {\ln(x_n - x_{n+1}) - \ln(x_{n+1} - x_{n+2})}.\]

\end{enumerate}



\item (5 Pts.) Consider the forward and backward difference operators \(D^+\) and \(D^-\) defined by
\[D^+f(x) = \frac{f(x + h) - f(x)}{h} \ D^-f(x) = \frac{f(x) - f(x - h)}{h}.\]

\begin{enumerate}
\item Assuming \(f\) is smooth, derive asymptotic error expansions for each of these operators.

\item What combination of \(D^+f(x)\) and \(D^-f(x)\) gives a second order accurate approximation to the derivative \(f'(x)\)?  Justify your answer.

\end{enumerate}

{\bf Solution}

\begin{enumerate}
\item By Taylor's Theorem,
\begin{eqnarray*}
f(x + h) & = & f(x) + f'(x)h + \frac{1}{2} f''(\alpha_1) h^2, \\
f(x - h) & = & f(x) - f'(x)h + \frac{1}{2} f''(\alpha_2) h^2,
\end{eqnarray*}
where \(\alpha_1\) and \(\alpha_2\) are between \(x,x+h\) and \(x,x-h\), respectively.  It follows that
\begin{eqnarray*}
D^+f(x) & = & f'(x) + \frac{1}{2} f''(\alpha_1) h \\
D^-f(x) & = & f'(x) + \frac{1}{2} f''(\alpha_2) h.
\end{eqnarray*}

\item Further using Taylor's Theorem,
\begin{eqnarray*}
f(x + h) & = & f(x) + f'(x) h + \frac{1}{2} f''(x) h^2 + \frac{1}{6} f^{(3)}(\alpha_3) h^3, \\
f(x - h) & = & f(x) - f'(x) h + \frac{1}{2} f''(x) h^2 - \frac{1}{6} f^{(3)}(\alpha_4) h^3,
\end{eqnarray*}
so that
\[\frac{1}{2} \left( D^+ + D^- \right) f(x)
  = \frac{f(x + h) - f(x - h)}{2h}
  = f'(x) + \frac{1}{12} \left( f^{(3)}(\alpha_3) + f^{(3)}(\alpha_4) \right) h^2.\]

\end{enumerate}



\item (5 Pts.) Consider the following factorization of a tri-diagonal matrix \(A\):
\[A = \left( \begin{array}{*{5}{c}}
             a_1 & c_1 \\
             b_2 & a_2 & c_2 \\
             & * & * & * \\
             & & * & * & c_{n-1} \\
             & & & b_n & a_n \\
             \end{array} \right)
    = \left( \begin{array}{*{5}{c}}
             1 \\
             d_2 & 1 \\
             & * & * \\
             & & * & * \\
             & & & d_n & 1 \\
             \end{array} \right)
      \left( \begin{array}{*{5}{c}}
             e_1 & c_1 \\
             & e_2 & c_2 \\
             & & * & * \\
             & & & * & c_{n-1} \\
             & & & & e_n \\
             \end{array}
      \right).\]

\begin{enumerate}
\item Derive the recurrence relations that determine the values of the \(d_k\)'s and \(e_k\)'s in terms of the values of the \(a_k\)'s, \(b_k\)'s, and \(c_k\)'s.

\item Give a condition on the matrix \(A\) which ensures your recurrence relations won't break down.

\end{enumerate}

{\bf Solution}

\begin{enumerate}
\item By simply multiplying out the matrices on the right,
\begin{eqnarray*}
e_1 & = & a_1 \\
d_k & = & b_k / e_{k - 1}, \ 2 \leq k \leq n \\
e_k & = & a_k - c_{k - 1} d_k, \ 2 \leq k \leq n.
\end{eqnarray*}

\item Errors using machine-precision approximations won't propagate if, e.g., \(A\) is diagonally dominant, i.e., \(|a_i| \geq |b_i| + |c_i|\).

\end{enumerate}



\item (10 Pts.)

\begin{enumerate}
\item Find conditions on the coefficients \(a_1,a_2,p_1,p_2\) so that the following Runge-Kutta method for \(y'(t) = f(t,y(t))\) is of order \(m \geq 2\):
\[y_{n + 1} = y_n + h \left( a_1 f(t_n,y_n) + a_2 f(t_n + p_1 h, y_n + p_2 h f(t_n,y_n)) \right).\]

\item Show by an example that the order cannot exceed two.

\item Analyze the linear stability of the scheme when \(a_1 = 0\), \(a_2 = 1\), \(p_1 = \frac{1}{2}\), \(p_2 = \frac{1}{2}\).

\end{enumerate}

{\bf Solution}

\begin{enumerate}
\item Suppose \(y(t_n) = y_n\).  By Taylor's Theorem,
\begin{eqnarray*}
y(t_n + h) & = & y_n + y'(t_n) h + \frac{1}{2} y''(t_n) h^2 + O(h^3) \\
           & = & y_n + f(t_n,y_n) h + \frac{1}{2} \left( f_t(t_n,y_n) + f_y(t_n,y_n) f(t_n,y_n) \right) h^2 + O(h^3).
\end{eqnarray*}
Also by Taylor's Theorem,
\[f(t_n + p_1h, y_n + p_2 h f(t_n,y_n))
  = f(t_n,y_n) + p_1 f_t(t_n,y_n) h + p_2 f_y(t_n,y_n) f(t_n,y_n) h + O(h^2),\]
hence
\begin{eqnarray*}
y_{n + 1} & = & y_n + h \left( a_1 f(t_n,y_n) + a_2 f(t_n + p_1 h, y_n + p_2 h f(t_n,y_n)) \right) \\
          & = & y_n + h \left( a_1 f(t_n,y_n) + a_2 \left( f(t_n,y_n) + p_1 f_t(t_n,y_n) h + p_2 f_y(t_n,y_n) f(t_n,y_n) h  + O(h^2) \right) \right) \\
          & = & y_n + (a_1 + a_2) f(t_n,y_n) h + a_2 \left( p_1 f_t(t_n,y_n) + p_2 f_y(t_n,y_n) f(t_n,y_n) \right) h^2 + O(h^3)
\end{eqnarray*}
which agrees with the expression for \(y(t_n + h)\) to \(O(h^3)\) if \(a_1 + a_2 = 1\), \(p_1 = p_2\), and \(a_2 p_1 = a_2 p_2 = \frac{1}{2}\).

\item If \(f\) is simply a function of \(t\), i.e., \(f(t,y(t)) = f(t)\), then the scheme reduces to (again, applying Taylor's Theorem)
\begin{eqnarray*}
y_{n + 1} & = & y_n + h \left( a_1 f(t_n) + a_2 f(t_n + p_1 h) \right) \\
          & = & y_n + h \left( a_1 f(t_n) + a_2 \left( f(t_n) + p_1 f'(t_n) h + \frac{1}{2} p_1^2 f''(t_n) h^2 + O(h^3) \right) \right) \\
          & = & y_n + (a_1 + a_2) f(t_n) h + a_2 p_1 f'(t_n) h^2 + \frac{1}{2} a_2 p_1^2 f''(t_n) h^3 + O(h^4).
\end{eqnarray*}
Matching coefficients on \(h\) up with the Taylor expansion of \(y(t_n + h)\) about \(t_n\) up to \(h^3\) yields
\begin{eqnarray*}
a_1 + a_2 & = & 1; \\
a_2 p_1 & = & \frac{1}{2}; \\
\frac{1}{2} a_2 p_1^2 & = & \frac{1}{6}.
\end{eqnarray*}
The last two equalities give \(p_1 = \frac{2}{3}\), from which we determine that \(a_2 = \frac{3}{4}\) and \(a_1 = \frac{1}{4}\).  Thus, in this special case, the order can actually exceed two.

\item We analyze stability by applying the method to the model problem \(y'(t) = f(t,y(t)) = \lambda y(t)\):
\begin{eqnarray*}
y_{n + 1} & = & y_n + h f \left( t_n + \frac{1}{2} h, y_n + \frac{1}{2} h f(t_n,y_n) \right) \\
          & = & y_n + h \lambda \left( y_n + \frac{1}{2} h \lambda y_n \right) \\
          & = & \left( 1 + \lambda h + \frac{1}{2} (\lambda h)^2 \right) y_n
\end{eqnarray*}
giving the root of the characteristic polynomial to be
\[\zeta = 1 + \lambda h + \frac{1}{2} (\lambda h)^2.\]
The stability region is the set of complex \(\lambda h\) such that
\[\left| 1 + \lambda h + \frac{1}{2} (\lambda h)^2 \right| < 1.\]
In particular, this is satisfied by \(\lambda h \in (-2,0)\).

\end{enumerate}



\item (10 Pts.) Let \(a(x,y)\) and \(b(x,y)\) be smooth positive functions.  Consider the equation
\[u_t = \left( a(x,y) u_x \right)_x + \left( b(x,y) u_y \right)_y\]
be solved for \(t > 0\), \((x,y) \in [0,1] \times [0,1]\), with smooth initial data \(u(x,y,0) = u_0(x,y)\) and periodic boundary conditions in \(x\) and \(y\): \(u(0,y,t) \equiv u(1,y,t)\), \(u(x,0,t) \equiv u(x,1,t)\).

\begin{enumerate}
\item Construct a second-order accurate unconditionally stable scheme for this equation.  Justify the accuracy and stability properties of your scheme.

\item Construct a second-order accurate unconditionally stable scheme for this equation that only requires the inversion of one dimensional operators.  Justify the accuracy and stability properties of your scheme.

\end{enumerate}

{\bf Solution}

\begin{enumerate}
\item We consider using Crank-Nicolson:
\begin{eqnarray*}
P_{k,h_x,h_y} u^n_{\ell,m}
& = & \frac{1}{k} \left( u^{n+1}_{\ell,m} - u^n_{\ell,m} \right) \\
& - & \frac{1}{2h_x} \left( a \left( x + \frac{1}{2} h_x, y \right) \frac{u^{n+1}_{\ell+1,m} - u^{n+1}_{\ell,m}}{h_x}
                          - a \left( x - \frac{1}{2} h_x, y \right) \frac{u^{n+1}_{\ell,m} - u^{n+1}_{\ell-1,m}}{h_x} \right) \\
& - & \frac{1}{2h_x} \left( a \left( x + \frac{1}{2} h_x, y \right) \frac{u^n_{\ell+1,m} - u^n_{\ell,m}}{h_x}
                          - a \left( x - \frac{1}{2} h_x, y \right) \frac{u^n_{\ell,m} - u^n_{\ell-1,m}}{h_x} \right) \\
& - & \frac{1}{2h_y} \left( b \left( x, y + \frac{1}{2} h_y \right) \frac{u^{n+1}_{\ell,m+1} - u^{n+1}_{\ell,m}}{h_y}
                          - b \left( x, y - \frac{1}{2} h_y \right) \frac{u^{n+1}_{\ell,m} - u^{n+1}_{\ell,m-1}}{h_y} \right) \\
& - & \frac{1}{2h_y} \left( b \left( x, y + \frac{1}{2} h_y \right) \frac{u^n_{\ell,m+1} - u^n_{\ell,m}}{h_y}
                          - b \left( x, y - \frac{1}{2} h_y \right) \frac{u^n_{\ell,m} - u^n_{\ell,m-1}}{h_y} \right) \\
& = & \frac{1}{k} \left( u^{n+1}_{\ell,m} - u^n_{\ell,m} \right) \\
& - & \frac{1}{2h_x^2} \left( a \left( x + \frac{1}{2} h_x, y \right) \left( u^{n+1}_{\ell+1,m} - u^{n+1}_{\ell,m} + u^n_{\ell+1,m} - u^n_{\ell,m} \right) \right. \\
&   & \left.                - a \left( x - \frac{1}{2} h_x, y \right) \left( u^{n+1}_{\ell,m} - u^{n+1}_{\ell-1,m} + u^n_{\ell,m} - u^n_{\ell-1,m} \right) \right) \\
& - & \frac{1}{2h_y^2} \left( b \left( x, y + \frac{1}{2} h_y \right) \left( u^{n+1}_{\ell,m+1} - u^{n+1}_{\ell,m} + u^n_{\ell,m+1} - u^n_{\ell,m} \right) \right. \\
&   & \left.                - b \left( x, y - \frac{1}{2} h_y \right) \left( u^{n+1}_{\ell,m} - u^{n+1}_{\ell,m-1} + u^n_{\ell,m} - u^n_{\ell,m-1} \right) \right); \\
R_{k,h_x,h_y} f^n_{\ell,m}
& = & \frac{1}{2} \left( f^{n+1}_{\ell,m} + f^n_{\ell,m} \right). \\
\end{eqnarray*}
The symbols \(p_{k,h_x,h_y}(s,\xi,\eta)\) and \(r_{k,h_x,h_y}(s,\xi,\eta)\) for these difference operators are given by
\begin{eqnarray*}
p_{k,h_x,h_y}(s,\xi,\eta)
& = & \left. P_{k,h_x,h_y} \left( e^{skn} e^{i(\xi\ell h_x + \eta mh_y)} \right) \right/ e^{skn} e^{i(\xi\ell h_x + \eta mh_y)} \\
& = & \frac{1}{k} \left( e^{sk} - 1 \right) \\
& - & \frac{1}{2h_x^2} \left( e^{sk} + 1 \right) \left( a \left( x + \frac{1}{2} h_x, y \right) \left( e^{i\xi h_x} - 1 \right) - a \left( x - \frac{1}{2} h_x, y \right) \left( 1 - e^{-i\xi h_x} \right) \right) \\
& - & \frac{1}{2h_y^2} \left( e^{sk} + 1 \right) \left( b \left( x, y + \frac{1}{2} h_y \right) \left( e^{i\eta h_y} - 1 \right) - b \left( x, y - \frac{1}{2} h_y \right) \left( 1 - e^{-i\eta h_y} \right) \right); \\
r_{k,h_x,h_y}(s,\xi,\eta)
& = & \left. R_{k,h_x,h_y} \left( e^{skn} e^{i(\xi\ell h_x + \eta mh_y)} \right) \right/ e^{skn} e^{i(\xi\ell h_x + \eta mh_y)} \\
& = & \frac{1}{2} \left( e^{sk} + 1 \right).
\end{eqnarray*}
We can simplify \(p_{k,h_x,h_y}\) to order \(O(k^2) + O(h_x^2) + O(h_y^2)\) as follows.  First, utilizing Taylor's Theorem,
\begin{eqnarray*}
\lefteqn{a \left( x + \frac{1}{2} h_x, y \right) \left( e^{i\xi h_x} - 1 \right)
       - a \left( x - \frac{1}{2} h_x, y \right) \left( 1 - e^{-i\xi h_x} \right)} \\
& = & a \left( x + \frac{1}{2} h_x, y \right) \left( i\xi h_x - \frac{1}{2} \xi^2 h_x^2 - \frac{1}{6} i\xi^3 h_x^3 + O(h_x^4) \right) \\
& - & a \left( x - \frac{1}{2} h_x, y \right) \left( i\xi h_x + \frac{1}{2} \xi^2 h_x^2 - \frac{1}{6} i\xi^3 h_x^3 + O(h_x^4) \right) \\
& = & i \xi a_x(x,y) h_x^2 - \xi^2 a(x,y) h_x^2 + O(h_x^4),
\end{eqnarray*}
and similarly
\begin{eqnarray*}
\lefteqn{b \left( x, y + \frac{1}{2} h_y \right) \left( e^{i\eta h_y} - 1 \right) - b \left( x, y - \frac{1}{2} h_y \right) \left( 1 - e^{-i\eta h_y} \right)} \\
& = & i \eta b_y(x,y) h_y^2 - \eta^2 b(x,y) h_y^2 + O(h_y^4).
\end{eqnarray*}
Also, we have that
\begin{eqnarray*}
\frac{1}{k} \left( e^{sk} - 1 \right)
& = & \frac{1}{k} \left( sk + \frac{1}{2} s^2 k^2 + O(k^3) \right) \\
& = & s + \frac{1}{2} s^2 k + O(k^2) \\
& = & \frac{1}{2} \left( 1 + 1 + sk + O(k^2) \right) s \\
& = & \frac{1}{2} \left( e^{sk} + 1 \right) s + O(k^2)
\end{eqnarray*}
Thus,
\begin{eqnarray*}
p_{k,h_x,h_y}(s,\xi,\eta)
& = & \frac{1}{k} \left( e^{sk} - 1 \right) \\
& - & \frac{1}{2h_x^2} \left( e^{sk} + 1 \right) \left( i \xi a_x h_x^2 - \xi^2 a h_x^2 + O(h_x^4) \right) \\
& - & \frac{1}{2h_y^2} \left( e^{sk} + 1 \right) \left( i \eta b_y h_y^2 - \eta^2 b h_y^2 + O(h_y^4) \right) \\
& = & \frac{1}{2} \left( e^{sk} + 1 \right) \left( s - i \xi a_x + \xi^2 a - i \eta b_y + \eta^2 b \right) + O(k^2) + O(h_x^2) + O(h_y^2).
\end{eqnarray*}
The symbol \(p(s,\xi,\eta)\) of the differential operator \(P = \partial_t - \partial_x(a\partial_x) - \partial_y(b\partial_y)\) is
\begin{eqnarray*}
p(s,\xi,\eta) & = & \left. P \left( e^{sk} e^{i(\xi x + \eta y)} \right) \right/ e^{sk} e^{i(\xi x + \eta y)} \\
              & = & s - i \xi a_x + \xi^2 a - i \eta b_x + \eta^2 b,
\end{eqnarray*}
from which we see that \(p_{k,h_x,h_y}(s,\xi,\eta)\) agrees with \(r_{k,h_x,h_y}(s,\xi,\eta) p(s,\xi,\eta)\) to \(O(k^2) + O(h_x^2) + O(h_y^2)\), as required for second-order accuracy.

Stability is analyzed by replacing \(g = e^{sk}\) in \(p_{k,h_x,h_y}(s,\xi,\eta) = 0\) and solving for \(g\) to determine the roots of the amplification polynomial:
\begin{eqnarray*}
\lefteqn{\frac{1}{k}(g - 1) + \frac{1}{2}(g + 1) \left( \frac{1}{h_x^2} \left( a \left( x + \frac{1}{2} h_x, y \right) \left( 1 - e^{i\xi h_x} \right) + a \left( x - \frac{1}{2} h_x, y \right) \left( 1 - e^{-i\xi h_x} \right) \right) \right.} \\
& + & \left. \frac{1}{h_y^2} \left( b \left( x, y + \frac{1}{2} h_y \right) \left( 1 - e^{i\eta h_y} \right) + b \left( x, y - \frac{1}{2} h_y \right) \left( 1 - e^{-i\eta h_y} \right) \right) \right) = 0 \\
& \Rightarrow & g - 1 + \frac{1}{2} (g + 1) c = 0 \\
& \Rightarrow & g = \frac{2 - c}{2 + c},
\end{eqnarray*}
where we have let
\begin{eqnarray*}
c & = & \mu_x \left( a \left( x + \frac{1}{2} h_x, y \right) \left( 1 - e^{i\xi h_x} \right) + a \left( x - \frac{1}{2} h_x, y \right) \left( 1 - e^{-i\xi h_x} \right) \right) \\
  & + & \mu_y \left( b \left( x, y + \frac{1}{2} h_y \right) \left( 1 - e^{i\eta h_y} \right) + b \left( x, y - \frac{1}{2} h_y \right) \left( 1 - e^{-i\eta h_y} \right) \right).
\end{eqnarray*}
Observing that \(\Re(c) \geq 0\) for all \(\xi,\eta,x,y\) (here the nonnegativity of \(a\) and \(b\) is required), we get that \(|g| \leq 1\), hence the scheme is unconditionally stable.

\item Abbreviate the operators
\[\partial_x \left( a \partial_x \right) = A, \ 
  \partial_y \left( b \partial_y \right) = B.\]
Then \(P = \partial_t - A - B\).  By Taylor's Theorem, if \(u_t = Au + Bu\),
\[\frac{u^{n+1} - u^n}{k}
  = \frac{1}{2} \left( Au^{n+1} + Au^n \right)
  + \frac{1}{2} \left( Bu^{n+1} + Bu^n \right) + O(k^2).\]
Rearranging gives
\[\left( I - \frac{k}{2} A - \frac{k}{2} B \right) u^{n+1}
  = \left( I + \frac{k}{2} A + \frac{k}{2} B \right) u^n + O(k^3),\]
so
\[\left( I - \frac{k}{2} A \right) \left( I - \frac{k}{2} B \right) u^{n+1}
  = \left( I + \frac{k}{2} A \right) \left( I + \frac{k}{2} B \right) u^n
  + \frac{k^2}{4} A B (u^{n+1} - u^n) + O(k^3),\]
and since \(u^{n+1} - u^n \in O(k)\), we obtain
\[\left( I - \frac{k}{2} A \right) \left( I - \frac{k}{2} B \right) u^{n+1}
  = \left( I + \frac{k}{2} A \right) \left( I + \frac{k}{2} B \right) u^n + O(k^3).\]
It follows that if \(A_{h_x}\) and \(B_{h_y}\) approximate \(A\) and \(B\) to \(O(h_x^2)\) and \(O(h_y^2)\), respectively, then
\[\left( I - \frac{k}{2} A_{h_x} \right) \left( I - \frac{k}{2} B_{h_y} \right) u^{n+1}
  = \left( I + \frac{k}{2} A_{h_x} \right) \left( I + \frac{k}{2} B_{h_y} \right) u^n
  + O(k^3) + O(kh_x^2) + O(kh_y^2).\]
This suggests the second-order ADI scheme
\[\left( I - \frac{k}{2} A_{h_x} \right) \left( I - \frac{k}{2} B_{h_y} \right) u^{n+1}
  = \left( I + \frac{k}{2} A_{h_x} \right) \left( I + \frac{k}{2} B_{h_y} \right) u^n.\]
We would use the \(A_{h_x}\) and \(B_{h_y}\) suggested by the scheme in (a), as we have already shown these to be second-order accurate.  The operators may be split according to
\begin{eqnarray*}
\left( I - \frac{k}{2} A_{h_x} \right) \widetilde{u}^{n+1/2}
& = & \left( I + \frac{k}{2} B_{h_y} \right) u^n, \\
\left( I - \frac{k}{2} B_{h_y} \right) u^{n+1}
& = & \left( I + \frac{k}{2} A_{h_x} \right) \widetilde{u}^{n+1/2}.
\end{eqnarray*}
We show stability in a way similar as before.  To simplify the notation, set
\begin{eqnarray*}
c_x & = & \mu_x \left( a \left( x + \frac{1}{2} h_x, y \right) \left( 1 - e^{i\xi h_x} \right) + a \left( x - \frac{1}{2} h_x, y \right) \left( 1 - e^{-i\xi h_x} \right) \right); \\
c_y & = & \mu_y \left( b \left( x, y + \frac{1}{2} h_y \right) \left( 1 - e^{i\eta h_y} \right) + b \left( x, y - \frac{1}{2} h_y \right) \left( 1 - e^{-i\eta h_y} \right) \right).
\end{eqnarray*}
As before, \(\Re(d_x), \Re(d_y) \geq 0\).  We can now quickly compute the amplification factor:
\begin{eqnarray*}
\left( 1 + \frac{1}{2} c_x \right) \widetilde{g} & = & 1 - \frac{1}{2} c_y, \\
\left( 1 + \frac{1}{2} c_y \right) g & = & \left( 1 - \frac{1}{2} c_x \right) \widetilde{g},
\end{eqnarray*}
so
\[g = \frac{\left( 1 - \frac{1}{2} c_x \right) \left( 1 - \frac{1}{2} c_y \right)}
           {\left( 1 + \frac{1}{2} c_x \right) \left( 1 + \frac{1}{2} c_y \right)},\]
from which it is evident that \(|g| \leq 1\), hence the scheme is unconditionally stable.

\end{enumerate}



\item (10 Pts.) Consider the initial boundary value problem
\[u_t + a u_x = 0,\]
where \(a\) is a real number, to be solved for \(x \geq 0\) and \(t \geq 0\), with smooth initial data \(u(x,0) = u_0(x)\).

\begin{enumerate}
\item For a given value of the constant \(a\), what boundary conditions, if any, are needed to solve this problem?

\item Suppose the Lax-Wendroff scheme
\[u^{n+1}_j = u^n_j - \frac{a\lambda}{2} \left( u^n_{j+1} - u^n_{j-1} \right) + \frac{a^2\lambda^2}{2} \left( u^n_{j+1} - 2u^n_j + u^n_{j-1} \right);\]
where \(\lambda = \frac{\Delta t}{\Delta x}\), \(j = 1, 2, \ldots\), and \(n = 0, 1, 2, \ldots\); is used to approximate solutions to this equation.  Give stable boundary conditions for \(u^n_0\).  Justify your statements.

\end{enumerate}

{\bf Solution}

\begin{enumerate}
\item If \(a < 0\), the solution to the boundary value problem is simply \(u(x,t) = u_0(x - at)\).  In this case, the characteristics travel to the left, and no boundary conditions are necessary.  On the other hand, if \(a > 0\), the characteristics travel to the right, and boundary conditions would be necessary to determine \(u(x,t)\) for \(t > x/a\).

\item 

\end{enumerate}



\item The following elliptic problem is approximated by the finite element method:
\begin{eqnarray*}
-\nabla \cdot \left( a(x) \nabla u(x) \right) & = & f(x), \ x \in \Omega \subset \mathbb{R}^2, \\
u(x) & = & u_0(x), \ x \in \Gamma_1, \\
\frac{\partial u}{\partial x_1}(x) + u(x) & = & 0, \ x \in \Gamma_2, \\
\frac{\partial u}{\partial x_2}(x) & = & 0, \ x \in \Gamma_3;
\end{eqnarray*}
where
\begin{eqnarray*}
\Omega & = & \left\{ (x_1,x_2) \ | \ 0 < x_1 < 1, \ 0 < x_2 < 1 \right\}, \\
\Gamma_1 & = & \left\{ (x_1,x_2) \ | \ x_1 = 0, \ 0 \leq x_2 \leq 1 \right\}, \\
\Gamma_2 & = & \left\{ (x_1,x_2) \ | \ x_1 = 1, \ 0 \leq x_2 \leq 1 \right\}, \\
\Gamma_3 & = & \left\{ (x_1,x_2) \ | \ 0 < x_1 < 1, \ x_2 = 0,1 \right\};
\end{eqnarray*}
\[0 < A \leq a(x) \leq B \ \text{for a.e.} \ x \in \Omega, \ f \in L^2(\Omega);\]
and \(u_0|_{\Gamma_1}\) is the trace of a function \(u_0 \in H^1(\Omega)\).

\begin{enumerate}
\item Determine an appropriate weak variational formulation of the problem.

\item Prove conditions on the corresponding linear and bilinear forms which are needed for existence and uniqueness of the solution.

\item Setup a finite element approximation using \(P_1\) elements and a set of basis functions such that the associated linear system is sparse and of band structure.  Discuss the linear system thus obtained, and give the rate of convergence.

\end{enumerate}

{\bf Solution}

\begin{enumerate}
\item We set \(w = u - u_0\) (so \(u = w + u_0\)) and reformulate the problem in terms of \(w\) to obtain homogeneous boundary conditions:
\begin{eqnarray*}
-\nabla \cdot (a(x) \nabla w(x)) & = & f(x) + \nabla \cdot (a(x) \nabla u_0(x)) = g(x), \ x \in \Omega, \\
w(x) & = & 0, \ x \in \Gamma_1, \\
\frac{\partial w}{\partial x_1}(x) + w(x) & = & -\frac{\partial u_0}{\partial x_1}(x) - u_0(x) = h(x), \ x \in \Gamma_2, \\
\frac{\partial w}{\partial x_2}(x) & = & -\frac{\partial u_0}{\partial x_2}(x) = k(x), \ x \in \Gamma_3.
\end{eqnarray*}
Let \(V = \{v \in H^1(\Omega) \ | \ v|_{\Gamma_1} \equiv 0\}\) equipped with the norm \(\|\cdot\|_{H^1(\Omega)}\).  We determine a weak variational formulation by multiplying the differential equation by \(v \in V\), applying integration by parts, and noting that \(v|_{\Gamma_1} \equiv 0\):
\begin{eqnarray*}
\lefteqn{-\nabla \cdot \left( a \nabla w \right) v = g v} \\
& \Rightarrow & \int_{\Omega} -\nabla \cdot \left( a \nabla w \right) v = \int_{\Omega} g v \\
& \Rightarrow & -\int_{\partial\Omega} a v \frac{\partial w}{\partial \nu} + \int_{\Omega} a \nabla w \cdot \nabla v = \int_{\Omega} g v \\
& \Rightarrow & -\int_{\Gamma_2} a v \frac{\partial w}{\partial x_1} + \int_{\Gamma_3, x_2 = 0} a v \frac{\partial w}{\partial x_2} - \int_{\Gamma_3, x_2 = 1} a v \frac{\partial w}{\partial x_2} + \int_{\Omega} a \nabla w \cdot \nabla v = \int_{\Omega} g v \\
& \Rightarrow & -\int_{\Gamma_2} a v \left( h - w \right) + \int_{\Gamma_3, x_2 = 0} a k v - \int_{\Gamma_3, x_2 = 1} a k v + \int_{\Omega} a \nabla w \cdot \nabla v = \int_{\Omega} g v \\
& \Rightarrow & \int_{\Omega} a \nabla w \cdot \nabla v + \int_{\Gamma_2} a w v = \int_{\Omega} g v + \int_{\Gamma_2} a h v - \int_{\Gamma_3, x_2 = 0} a k v + \int_{\Gamma_3, x_2 = 1} a k v.
\end{eqnarray*}
Let
\begin{eqnarray*}
a(w,v) & = & \int_{\Omega} a \nabla w \cdot \nabla v + \int_{\Gamma_2} a w v \\
Lv & = & \int_{\Omega} g v + \int_{\Gamma_2} a h v - \int_{\Gamma_3, x_2 = 0} a k v + \int_{\Gamma_3, x_2 = 1} a k v
\end{eqnarray*}
such that the weak variational formulation is to find \(w \in V\) such that
\[a(w,v) = Lv \ \text{for all} \ v \in V.\]

\item The Lax-Milgram Lemma provides sufficient conditions the bilinear form \(a\) and the linear form \(L\) must satisfy for existence and uniqueness of \(w\):

\begin{itemize}
\item {\em \(a\) is symmetric.}  Clearly \(a(v_1,v_2) = a(v_2,v_1)\) for all \(v_1,v_2 \in V\).

\item {\em \(a\) is continuous.} For \(v_1,v_2 \in V\), by the Cauchy-Schwarz Inequality,
\begin{eqnarray*}
|a(v_1,v_2)| &   =  & \left| \int_{\Omega} a \nabla v_1 \cdot \nabla v_2 + \int_{\Gamma_2} a v_1 v_2 \right| \\
             & \leq & B \|\nabla v_1\|_{L^2(\Omega)} \|\nabla v_2\|_{L^2(\Omega)} + B \|v_1\|_{L^2(\Gamma_2)} \|v_2\|_{L^2(\Gamma_2)} \\
             & \leq & B \|v_1\|_{H^1(\Omega)} \|v_2\|_{H^2(\Omega)} + B \|v_1\|_{L^2(\Gamma_2)} \|v_2\|_{L^2(\Gamma_2)}.
\end{eqnarray*}
But
\[\|v_i\|_{L^2(\Gamma_2)} \leq C \|v_i\|_{H^1(\Omega)}\]
for some \(C > 0\), so, in fact,
\[|a(v_1,v_2)| \leq B(1 + C) \|v_1\|_{H^1(\Omega)} \|v_2\|_{H^2(\Omega)},\]
and we conclude that \(a\) is continuous.

\item {\em \(a\) is \(V\)-elliptic.}  For \(v \in V\),
\begin{eqnarray*}
a(v,v) &   =  & \int_{\Omega} a |\nabla v|^2 + \int_{\Gamma_2} a v^2 \\
       & \geq & A \int_{\Omega} |\nabla v|^2 + A \int_{\Gamma_2} v^2 \\
       &   =  & A \|\nabla v\|_{L^2(\Omega)}^2 + A \|v\|_{L^2(\Gamma_2)}^2 \\
       & \geq & A \|\nabla v\|_{L^2(\Omega)}^2.
\end{eqnarray*}
Now since \(v|_{\Gamma_1} = 0\) and \(\Gamma_1\) has positive length,
\[\|\nabla v\|_{L^2(\Omega)} \geq C' \|v\|_{H^1(\Omega)}\]
for some \(C' > 0\), so
\[a(v,v) \geq A C'^2 \|v\|_{H^1(\Omega)}^2,\]
and so \(a\) is \(V\)-elliptic.

\item {\em \(L\) is continuous.}  For \(v \in V\), by the Cauchy-Schwarz Inequality,
\begin{eqnarray*}
|Lv| &   =  & \left| \int_{\Omega} g v + \int_{\Gamma_2} a h v - \int_{\Gamma_3, x_2 = 0} a k v + \int_{\Gamma_3, x_2 = 1} a k v \right| \\
     & \leq & \|g\|_{L^2(\Omega)} \|v\|_{L^2(\Omega)} + B \|h\|_{L^2(\Gamma_2)} \|v\|_{L^2(\Gamma_2)} + B \|k\|_{L^2(\Gamma_3)} \|v\|_{L^2(\Gamma_3)}.
\end{eqnarray*}
As before,
\[\|v\|_{L^2(\Gamma_3)} \leq C'' \|v\|_{H^1(\Omega)}\]
for some \(C'' > 0\), so that
\[|Lv| \leq \left( \|g\|_{L^2(\Omega)} + B C \|h\|_{L^2(\Gamma_2)} + B C'' \|k\|_{L^2(\Gamma_3)} \right) \|v\|_{H^1(\Omega)},\]
hence \(L\) is continuous.

\end{itemize}

\item (W06.7(c))

\end{enumerate}



\end{enumerate}

\end{document}
